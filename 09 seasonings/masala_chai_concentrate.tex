\documentclass[11pt,letterpaper]{article}
\usepackage{fontspec}
\usepackage{tocloft}
\usepackage{multicol}
\usepackage{fancyhdr}
\usepackage{nicefrac}
\usepackage[left=1.3in,right=1.3in,top=1in,bottom=1in]{geometry}
\setmainfont[Scale=1.3,AutoFakeBold=1.5,AutoFakeSlant=0.3]{Doves Type}

\title{Masala Chai Concentrate}
\author{}
\date{}

% Configure fancy header
\pagestyle{fancy}
\fancyhf{} % Clear all header and footer fields
\fancyhead[C]{\textit{Masala Chai Concentrate}} % Center header with recipe title
\fancyfoot[C]{\thepage} % Center page number at bottom
\renewcommand{\headrulewidth}{0pt} % Remove header line
\renewcommand{\footrulewidth}{0pt} % Remove footer line

\begin{document}

\maketitle
\thispagestyle{empty}

\begin{quote}
\textit{A richly spiced chai concentrate sweetened with condensed milk, designed for effortless preparation. Makes about 6--6\nicefrac{1}{2}~cups concentrate (12 servings of 8~oz chai when diluted 1:1 with hot water or milk). Stores refrigerated up to 2~weeks.}
\end{quote}

\section*{Ingredients}
\setlength{\columnsep}{20pt}
\begin{multicols}{2}
\noindent
    Water \dotfill 5 cups \\
    Green cardamom pods \dotfill 20 \\
    Fresh ginger root \dotfill 3-inch knob \\
    Cinnamon sticks \dotfill 2 (3-inch) \\
    Black peppercorns \dotfill 12 \\
    Whole cloves \dotfill 6 \\
    \columnbreak
    Fennel seeds \dotfill 1\nicefrac{1}{2} tsp. \\
    Star anise \dotfill 3 whole \\
    Black tea, loose leaf \dotfill 5 Tbsp. \\
    Sweetened condensed milk \dotfill 21 oz. \\
    Granulated sugar \dotfill \nicefrac{1}{3} cup \\
\end{multicols}

\section*{Directions}

\noindent
Crush \textbf{cardamom pods}; slice \textbf{ginger} into thin rounds; break \textbf{cinnamon sticks} into smaller pieces ---
Combine \textbf{cardamom}, \textbf{ginger}, \textbf{cinnamon pieces}, \textbf{peppercorns}, \textbf{cloves}, \textbf{fennel seeds}, and \textbf{star anise} in \textit{Medium Bowl~\#1} (toast) ---
Measure \textbf{black tea} and set aside ---
Have \textbf{condensed milk} and \textbf{sugar} ready in \textit{Medium Bowl~\#2}

\begin{enumerate}
    \item In a large, heavy-bottomed saucepan or Dutch oven, add \textbf{cardamom}, \textbf{ginger}, and \textbf{whole spices} (\textit{Medium Bowl~\#1}). Toast over \textit{medium heat} for \textit{2-3~minutes}, stirring frequently, until fragrant and spices begin to release their oils. Do not allow spices to burn. If spices smoke or smell acrid, remove from heat immediately and let cool before discarding; toast a fresh batch.
    
    \item Add 5~cups \textbf{water} to the toasted spices. Increase heat to high and bring to a rolling boil. Once boiling, reduce heat to \textit{medium-low} and maintain a steady simmer for \textit{20~minutes}, stirring occasionally. The liquid should reduce by approximately \nicefrac{3}{4}~cup due to evaporation, concentrating the spice flavors.
    
    \item Remove saucepan from heat. Immediately add 5~Tbsp. \textbf{black tea} to the hot spiced water. Stir gently to submerge all tea leaves, cover with lid, and steep for exactly \textit{5~minutes}. Do not steep longer to avoid bitterness.
    
    \item While tea steeps, prepare a fine-mesh strainer or cheesecloth-lined strainer over a large heat-proof bowl or measuring cup. After \textit{5~minutes}, strain the concentrate through the prepared strainer, pressing firmly on the solids with the back of a spoon to extract maximum liquid. Discard solids. You should have approximately 4~to~4\nicefrac{1}{4}~cups of strained tea concentrate.
    
    \item Return the strained tea concentrate to the saucepan (wiped clean if needed). Place over \textit{medium heat} and add \nicefrac{1}{3}~cup \textbf{sugar} (from \textit{Medium Bowl~\#2}). Stir continuously until sugar is completely dissolved, about \textit{2~minutes}.
    
    \item Reduce heat to \textit{medium-low}. Gradually add 21~oz. \textbf{condensed milk} (\textit{Medium Bowl~\#2}) in a steady stream while stirring constantly to prevent scorching. Continue stirring until the condensed milk is fully integrated into the tea base, creating a smooth, uniform concentrate.
    
    \item Once integrated, increase heat slightly to bring the mixture just to the edge of a simmer---small bubbles should appear around the edges but mixture should not reach a full boil. Maintain this gentle simmer for \textit{2-3~minutes}, stirring constantly, until the concentrate is smooth and creamy with no separation. If the mixture separates, remove from heat and whisk vigorously; if it does not recombine, use as-is and stir well before each serving.
    
    \item Remove from heat and allow concentrate to cool to room temperature, approximately \textit{45~minutes~to~1~hour} (saucepan should feel cool to the touch, not warm). Stir occasionally during cooling to prevent skin formation.
    
    \item Once cooled, transfer concentrate to clean glass bottles or jars using a funnel. Seal tightly and refrigerate immediately. The concentrate will thicken slightly as it cools.
    
    \item To serve: Mix 4~oz. (\nicefrac{1}{2}~cup) \textbf{chai concentrate} with 4~oz. (\nicefrac{1}{2}~cup) hot water or steamed milk. Stir well and serve immediately. Adjust ratio to taste---use more concentrate for stronger chai, more liquid for milder flavor.
\end{enumerate}

\newpage

{\footnotesize
\setlength{\columnsep}{20pt}
\setlength{\multicolsep}{6pt}
\begin{multicols}{2}
\setlength{\parindent}{0pt}
\setlength{\parskip}{2pt}
\setlength{\itemsep}{0pt}
\setlength{\parsep}{0pt}

\subsection*{Yield}
\begin{itemize}
    \item About 6--6\nicefrac{1}{2}~cups concentrate; 12 servings of 8~oz chai (4~oz concentrate, 1:1 dilution)
\end{itemize}

\subsection*{Equipment Required}
\begin{itemize}
    \item Large heavy-bottomed saucepan or 4-quart Dutch oven
    \item Fine-mesh strainer or cheesecloth
    \item Large heat-proof bowl or 8-cup measuring cup
    \item Wooden spoon or heat-resistant silicone spatula
    \item Medium prep bowls (2)
    \item Funnel
    \item Glass bottles or jars with tight-fitting lids (6-8~cups total capacity)
    \item Measuring cups and spoons
    \item Mortar and pestle or heavy knife for crushing cardamom
\end{itemize}

\subsection*{Mise en Place}
\begin{itemize}
    \item Medium Bowl \#1 --- toast: crushed \textbf{cardamom pods} (20), sliced \textbf{ginger} (3-inch knob), broken \textbf{cinnamon sticks}, 12~\textbf{peppercorns}, 6~\textbf{cloves}, 1\nicefrac{1}{2}~tsp. \textbf{fennel seeds}, 3~\textbf{star anise}
    \item Medium Bowl \#2 --- 21~oz. \textbf{condensed milk} (1\nicefrac{1}{2} standard 14-oz. cans) and \nicefrac{1}{3}~cup \textbf{sugar}
    \item Have 5~cups \textbf{water} and 5~Tbsp. \textbf{black tea} measured and ready
\end{itemize}

\subsection*{Ingredient Tips}
\begin{itemize}
    \item \textbf{Green cardamom pods}: Use fresh, plump pods that feel slightly heavy. Avoid pre-ground cardamom as it loses potency quickly. Crush pods just enough to crack them open and expose seeds.
    \item \textbf{Fresh ginger}: Choose firm, unwrinkled ginger with tight skin. No need to peel if thoroughly washed. Slicing into thin rounds maximizes surface area for extraction.
    \item \textbf{Black tea}: Assam CTC (Crush-Tear-Curl) is ideal for authentic flavor and proper strength. Darjeeling works but is more delicate. English Breakfast is acceptable. Avoid Earl Grey (bergamot conflicts with spices).
    \item \textbf{Tea bags alternative}: Use 14-16~standard black tea bags if loose leaf unavailable. Remove after steeping to prevent over-extraction.
    \item \textbf{Star anise}: Use whole star anise, not broken pieces or ground. Contributes subtle licorice notes without overpowering.
    \item \textbf{Fennel seeds}: Adds sweet, slightly anise-like complexity. Use whole seeds, not ground.
    \item \textbf{Condensed milk}: Sweetened condensed milk only (not evaporated milk). Brand matters less than freshness---check expiration date.
    \item \textbf{Spice quality}: Whole spices from bulk bins or specialty stores are fresher than jarred supermarket spices. Store whole spices in airtight containers away from light.
\end{itemize}

\subsection*{Preparation Tips}
\begin{itemize}
    \item \textbf{Toasting spices}: Watch carefully and stir constantly. Spices can burn quickly, which creates bitter flavors. They should smell fragrant and warm, not smoky.
    \item \textbf{Simmering time}: The \textit{20-minute} simmer is critical for full spice extraction. Rushing this step produces weak, one-dimensional flavor.
    \item \textbf{Tea steeping}: Set a timer for exactly \textit{5~minutes}. Over-steeped tea becomes astringent and bitter, especially when concentrated.
    \item \textbf{Straining thoroughly}: Press firmly on solids to extract every drop of flavored liquid. The yield difference between casual straining and thorough pressing is significant.
    \item \textbf{Emulsification technique}: Adding \textbf{condensed milk} gradually while stirring prevents separation. The gentle simmer at the end ensures stable emulsion.
    \item \textbf{Avoiding boiling after milk addition}: Full boiling can cause \textbf{condensed milk} proteins to separate or scorch. Keep at gentle simmer only.
    \item \textbf{Cooling before bottling}: Hot concentrate transferred directly to bottles can crack glass and creates condensation that promotes spoilage.
\end{itemize}

\subsection*{Make Ahead \& Storage}
\begin{itemize}
    \item Concentrate stores refrigerated for up to \textit{2~weeks} in clean, airtight glass containers
    \item Always use clean utensils when portioning concentrate to prevent contamination
    \item Concentrate will thicken when cold---this is normal; shake or stir before using
    \item Check for signs of spoilage before using: off smell, separation that doesn't resolve with stirring, mold, or sour taste
    \item For extended storage, freeze concentrate in ice cube trays (\textit{2~oz. portions}), then transfer cubes to freezer bags. Frozen concentrate keeps \textit{3~months}. Thaw cubes in refrigerator overnight.
    \item Label bottles with preparation date for easy tracking
    \item Spices can be toasted and stored separately in airtight container for \textit{1~week} if you want to streamline future batches
\end{itemize}

\subsection*{Serving Suggestions}
\begin{itemize}
    \item \textbf{Classic chai latte}: Mix 4~oz. concentrate with 4~oz. steamed whole milk or oat milk
    \item \textbf{Simple hot chai}: Mix 4~oz. concentrate with 4~oz. hot water (the \textbf{condensed milk} already provides creaminess)
    \item \textbf{Iced chai}: Mix 4~oz. concentrate with 4~oz. cold milk over ice; stir well
    \item \textbf{Stronger chai}: Use 5~oz. concentrate with 3~oz. liquid
    \item \textbf{Milder chai}: Use 3~oz. concentrate with 5~oz. liquid
    \item \textbf{Dirty chai}: Add 1-2~shots espresso to prepared chai for coffee-chai hybrid
    \item Garnish with ground cinnamon, freshly grated nutmeg, or star anise for presentation
    \item Pair with biscotti, shortbread, or traditional Indian snacks like samosas or pakoras
    \item For special occasions, top with frothed milk and dust with cardamom-cinnamon blend
    \item Concentrate also works as flavoring for baked goods, ice cream base, or overnight oats
\end{itemize}

\end{multicols}
}

\end{document}
