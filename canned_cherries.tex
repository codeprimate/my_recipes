
\documentclass[11pt,letterpaper]{article}
\usepackage{fontspec}
\usepackage{tocloft} % For dotted lines
\usepackage{multicol}
\usepackage[left=1.4in,right=1.5in,top=1in,bottom=1in]{geometry}
\setmainfont[Scale=1.2,AutoFakeBold=1.5,AutoFakeSlant=0.3]{Doves Type}

\title{Cherry Canning Recipe}
\author{}
\date{}

\begin{document}

\maketitle
\thispagestyle{empty}

\section*{Ingredients}
\setlength{\columnsep}{20pt}
\begin{multicols}{2}
\noindent
    Cherries \dotfill 10\# \\
    Bottled lemon juice \dotfill 2½ cups \\
    Sugar \dotfill 5 cups \\
    Water \dotfill 20 cups \\
    Canning jars/lids/bands (quart) \dotfill 7 \\
\end{multicols}

\section*{Instructions}
\begin{enumerate}
    \item \textbf{Prepare the equipment}: Sterilize 7 quart jars and an equal number of lids and bands; check the pressure canner for proper operation, including the seal and vent.
    \item \textbf{Prepare the cherries}: Wash and pit 10\# of \textbf{cherries}. Prepare a soaking solution with 2½ cups of \textbf{bottled lemon juice} and 10 cups of \textbf{water}. Soak the cherries for 10 minutes to help preserve their color and flavor.
    \item \textbf{Prepare the syrup}: Combine 5 cups of \textbf{sugar} with 10 cups of water in a large saucepan. Heat over medium-high until the \textbf{sugar} is completely dissolved, stirring occasionally to prevent sticking.
    \item \textbf{Pack the jars}: Evenly distribute the prepared cherries into the sterilized jars. Pour the hot syrup over the cherries, ensuring each jar is filled while leaving approximately 1 inch of headspace. Use a non-metallic spatula to gently stir inside the jars to remove any trapped air bubbles.
    \item \textbf{Process in canner}: Place the filled jars on the rack inside the pressure canner. Add water as per the canner's instructions, usually around 2-3 inches. Secure the lid and heat until steam flows freely from the vent. Continue to vent for 10 minutes, then close the vent and attach the pressure regulator weight. Process the jars at 10-15 pounds of pressure (adjusted for altitude) for 8-10 minutes.
    \item \textbf{Cool down and store}: Turn off the heat and let the pressure canner cool naturally until the pressure gauge reads zero. Carefully remove the jars using a jar lifter and place them on a towel or cooling rack, avoiding drafty areas. After 12-24 hours, check that each jar is sealed by pressing the center of the lid; it should not flex up or down. Label the jars with the canning date and store in a cool, dark place.
\end{enumerate}


\end{document}
