\documentclass[11pt,letterpaper]{article}
\usepackage[left=1.5in,right=1.5in,top=0.5in,bottom=1in]{geometry}
\usepackage{fontspec}
\usepackage{multicol}

\setmainfont[Scale=1.5,AutoFakeBold=1.5,AutoFakeSlant=0.4]{Doves Type}

\title{Tuscan Zuppa Soup}
\date{}
\author{}

\begin{document}
\maketitle

\section*{Ingredients}
\begin{multicols}{2}
    \begin{itemize}
        \item 1 lb Italian sausage, spicy or mild
        \item 1 medium onion, finely chopped
        \item 3-4 cloves garlic, minced
        \item 6 cups chicken broth
        \item 3-4 large potatoes, diced
        \item 1 can (15 oz) cannellini beans, drained and rinsed
    \end{itemize}
    \columnbreak
    \begin{itemize}
        \item 1 bunch kale or Swiss chard, stems removed, leaves torn
        \item 1/2 cup heavy cream (optional)
        \item Salt and pepper to taste
        \item Olive oil for cooking
        \item Pinch of red pepper flakes (optional)
    \end{itemize}
\end{multicols}

\section*{Directions}
\begin{enumerate}
    \item Brown the Italian sausage over medium heat until fully cooked. Remove and set aside, retaining drippings in the pot.
    \item Sauté onions in the same pot until translucent, then add garlic and cook for an additional minute.
    \item Return sausage to the pot. Add chicken broth and potatoes. Bring to a boil, then simmer until potatoes are tender, about 10-15 minutes.
    \item Stir in kale or Swiss chard and cannellini beans. Cook until greens are wilted and beans are heated through, about 5 minutes.
    \item If desired, stir in heavy cream. Season with salt, pepper, and red pepper flakes to taste.
    \item Serve hot with crusty bread.
\end{enumerate}

\end{document}