\documentclass[11pt,letterpaper]{article}
\usepackage{fontspec}
\usepackage{tocloft} % For dotted lines
\usepackage{multicol}
\usepackage[left=1.4in,right=1.5in,top=1in,bottom=1in]{geometry}
\setmainfont[Scale=1.4,AutoFakeBold=1.5,AutoFakeSlant=0.3]{Doves Type}


\title{Tuscan Zuppa Soup}
\author{}
\date{}

\begin{document}

\maketitle
\thispagestyle{empty}

\begin{quote}
\small
\begin{em}
Tuscan Zuppa Soup is a comforting and hearty dish, perfect for cold days or as a satisfying meal any time. It combines rustic flavors of Italian sausage, potatoes, and kale, simmered in a savory chicken broth, offering a taste of traditional Italian cuisine.
\end{em}
\end{quote}

\section*{Ingredients}
\setlength{\columnsep}{20pt}
\begin{multicols}{2}
\noindent
    Italian sausage \dotfill 1 lb. \\
    Onion, medium \dotfill 1 \\
    Garlic cloves \dotfill 3-4 \\
    Kale \dotfill 1 bunch \\
    Potatoes, large \dotfill 3-4 \\
    \columnbreak
    Chicken broth \dotfill 6 cups \\
    Cannellini beans \dotfill 1 15 oz. can \\
    Heavy cream \dotfill ½ cup \\
    Red pepper flakes \dotfill ¼ tsp. \\
    Salt \dotfill \\
    Pepper \dotfill
\end{multicols}

\section*{Directions}

\noindent
Finely chop \textbf{onions} ---
Mince \textbf{garlic} ---
Remove stems from \textbf{kale} and tear leaves into bite sized pieces ---
Quarter \textbf{potatoes} lengthwise and cut into ¼" slices ---
Drain and rinse \textbf{beans}

\begin{enumerate}
    \item Brown the \textbf{Italian sausage} over medium heat until fully cooked. Remove and set aside, retaining drippings in the pot.
    \item Sauté \textbf{onions} in the same pot until translucent, then add \textbf{garlic} and cook for an additional minute.
    \item Return sausage to the pot. Add \textbf{chicken broth} and \textbf{potatoes}. Bring to a boil, then simmer until potatoes are tender, about 10-15 minutes.
    \item Stir in \textbf{kale or chard} and \textbf{cannellini beans}. Cook until greens are wilted and beans are heated through, about 5 minutes.
    \item Stir in \textbf{heavy cream}. Season with \textbf{salt}, \textbf{pepper}, and \textbf{red pepper flakes} to taste.
\end{enumerate}

\end{document}
