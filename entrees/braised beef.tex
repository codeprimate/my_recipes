\documentclass[11pt,letterpaper]{article}
\usepackage{fontspec}
\usepackage{tocloft}
\usepackage{multicol}
\usepackage[left=1.4in,right=1.5in,top=1in,bottom=1in]{geometry}
\setmainfont[Scale=1.2,AutoFakeBold=1.5,AutoFakeSlant=0.3]{Doves Type}

\title{French Braised Beef Chuck}
\author{}
\date{}

\begin{document}

\maketitle
\thispagestyle{empty}

\section*{Ingredients}
\setlength{\columnsep}{20pt}
\begin{multicols}{2}
\noindent
    Beef chuck shoulder \dotfill 4--5~lbs \\
    Burgundy wine \dotfill 2 cups \\
    Beef stock \dotfill 2 cups \\
    Onions, large \dotfill 2 \\
    Celery stalks \dotfill 3 \\
    Garlic cloves \dotfill 4 \\
    Tomato paste \dotfill 3~Tbsp. \\
    Fresh thyme \dotfill 4--5 sprigs \\
	\columnbreak
    Dried bay leaves \dotfill 2 \\
    Black peppercorns \dotfill 1~tsp. \\
    French dip seasoning mix \dotfill 2~Tbsp. \\
    Kosher salt \dotfill 1½~tsp. \\
    Black pepper \dotfill ½~tsp. \\
    Vegetable oil \dotfill 2~Tbsp. \\
    Pearl onions, peeled \dotfill 2 cups \\
    Cremini mushrooms \dotfill 2 cups \\
    Fresh parsley, chopped \dotfill 2~Tbsp. \\
\end{multicols}

\section*{Searing and Aromatics}

\noindent
Pat dry \textbf{beef chuck} with paper towels ---
Rough chop \textbf{onions} and \textbf{celery} into large chunks ---
Mince \textbf{garlic} ---
Prepare \textbf{bouquet garni}: bundle \textbf{fresh thyme}, \textbf{bay leaves}, and \textbf{peppercorns} in a disposable teabag ---
Preheat oven to \textit{275°F}

\begin{enumerate}
    \item Season \textbf{beef chuck} generously on all sides with \textbf{kosher salt} and \textbf{black pepper}.
    
    \item In a Dutch oven, heat \textbf{vegetable oil} over medium-high heat until shimmering. Working carefully to avoid splatter, sear the \textbf{beef} on all sides until deeply browned, approximately \textit{3--4 minutes} per side. The goal is a rich mahogany crust. Remove \textbf{beef} and set aside.
    
    \item In the same pot, reduce heat to medium. Add rough-chopped \textbf{onions} and \textbf{celery}, stirring occasionally until they begin to soften and caramelize slightly, about \textit{5--7 minutes}. Add minced \textbf{garlic} and cook for \textit{1~minute} until fragrant.
    
    \item Add \textbf{tomato paste} and stir constantly for \textit{1--2 minutes}, allowing it to caramelize slightly and deepen in color. This develops savory complexity.
    
    \item Deglaze the pot with \textbf{Burgundy wine}, scraping up all browned fond from the bottom with a wooden spoon. Simmer for \textit{2--3 minutes} to allow alcohol to burn off slightly.
    
    \item Return seared \textbf{beef} to the pot. Add \textbf{beef stock} and \textbf{French dip seasoning mix}. The liquid should come approximately one-third up the sides of the meat. Add \textbf{bouquet garni}.
\end{enumerate}

\section*{The Braise}

\begin{enumerate}
    \item Bring braising liquid to a bare simmer on the stovetop, approximately \textit{2--3 minutes}.
    
    \item Cover with the lid and transfer to the preheated \textit{275°F} oven. Braise for \textit{3 hours}, then check the meat for tenderness with a fork. It should yield easily but still hold its shape.
    
    \item After \textit{3 hours}, add peeled \textbf{pearl onions} and \textbf{cremini mushrooms} directly to the braising liquid, nestling them among the aromatics and meat. Return to oven, covered, for an additional \textit{1.5--2 hours} until \textbf{beef} is completely fork-tender and vegetables are yielding but not dissolved.
    
    \item Remove from oven. Using tongs or a slotted spoon, carefully transfer the \textbf{beef} to a warm platter, cradling it gently to prevent breaking apart. Distribute \textbf{pearl onions} and \textbf{mushrooms} around the meat.
\end{enumerate}

\section*{Sauce and Service}

\begin{enumerate}
    \item Place the Dutch oven on the stovetop over medium heat. Allow the braising liquid to come to a gentle simmer. Using a skimming ladle, carefully skim the surface fat and impurities, working methodically until the surface is relatively clear. This typically requires \textit{3--5 minutes} of gentle skimming.
    
    \item The sauce should have reduced naturally to approximately one-third of its original volume, yielding a silky, glossy consistency that coats the back of a spoon. If it appears too thin, continue simmering gently until it reaches desired body, approximately \textit{2--3 minutes} more. Taste and adjust seasoning with additional \textbf{salt} and \textbf{pepper} as needed.
    
    \item Remove and discard \textbf{bouquet garni}.
    
    \item Pour sauce over \textbf{beef}, \textbf{pearl onions}, and \textbf{mushrooms}. Garnish generously with fresh \textbf{parsley}. Serve immediately with mashed potatoes and supplemental beef stock gravy on the side.
\end{enumerate}

\newpage

% Begin compact two-column layout
{\small
\setlength{\columnsep}{20pt}
\setlength{\multicolsep}{6pt}
\begin{multicols}{2}
\setlength{\parindent}{0pt}
\setlength{\parskip}{4pt}

\subsection*{Equipment Required}
\begin{itemize}
    \item 5--6 quart enameled Dutch oven with self-basting lid
    \item Large skillet or sauté pan for searing (optional; can use Dutch oven)
    \item Wooden spoon for stirring and scraping fond
    \item Tongs or slotted spoon for handling meat
    \item Skimming ladle with perforations
    \item Sharp knife and cutting board
    \item Measuring cups and spoons
    \item Paper towels for patting dry
    \item Disposable cotton teabags (or cheesecloth bundle)
    \item Warm platter for resting meat
    \item Instant-read thermometer (optional but helpful)
\end{itemize}

\subsection*{Mise en Place}
\begin{itemize}
    \item Ensure \textbf{beef chuck} is at room temperature before searing---remove from refrigerator \textit{30--45~minutes} prior
    \item Peel \textbf{pearl onions} and clean \textbf{mushrooms} well in advance; store in separate containers
    \item Rough chop \textbf{onions} and \textbf{celery} into large, uniform chunks
    \item Mince \textbf{garlic} fresh just before use
    \item Measure all liquids and seasonings
    \item Assemble \textbf{bouquet garni} in teabag
    \item Measure \textbf{French dip seasoning mix}
\end{itemize}

\subsection*{Ingredient Tips}
\begin{itemize}
    \item Select a quality Burgundy wine---Pinot Noir or a rustic Côtes du Bourgogne works well. Avoid heavily oaked wines; the braise will concentrate and intensify the flavors
    \item Chuck shoulder with good marbling ensures richness and tenderness; ask your butcher for a single large roast rather than portioned pieces
    \item Beef stock should be homemade or high-quality store-bought; weak stock will result in thin, unsatisfying sauce
    \item French dip seasoning mix varies by brand; look for onion-forward blends without excessive salt
    \item Pearl onions vary in size; uniformity helps with even cooking
    \item Cremini mushrooms maintain better texture than button mushrooms; avoid portobello, which can become mushy
    \item Fresh thyme is essential for the bouquet garni; dried thyme becomes powdery and unpleasant
\end{itemize}

\subsection*{Preparation Tips}
\begin{itemize}
    \item Pat the \textbf{beef} thoroughly dry before searing; moisture prevents proper browning
    \item Develop a deep mahogany crust during searing---this Maillard reaction creates the foundation of flavor
    \item Do not skip the fond-scraping step; those browned bits contain concentrated savory compounds
    \item Allow \textbf{tomato paste} to caramelize briefly before deglazing; this concentrates and deepens its umami impact
    \item The braising liquid should come only one-third up the meat, not submerge it; this creates a humid oven environment without excessive stewing
    \item Do not stir the braise during cooking; resist the urge to check frequently. The self-basting lid does the work
    \item Add \textbf{pearl onions} and \textbf{mushrooms} late to preserve their distinct texture and flavor
    \item Skim fat methodically and gently; aggressive skimming can cloud the sauce
    \item Taste the sauce before service; the braising aromatics may have rendered differently than expected, requiring seasoning adjustment
\end{itemize}

\subsection*{Make Ahead \& Storage}
\begin{itemize}
    \item The \textbf{beef} can be seared and the braise begun up to \textit{4~hours} ahead; add \textbf{pearl onions} and \textbf{mushrooms} only when you plan to finish
    \item Prepare all vegetables and seasonings the morning of service
    \item Leftover braise keeps refrigerated for up to \textit{4~days}; the flavors actually deepen overnight
    \item To reheat: gently warm in a \textit{325°F} oven, covered, for \textit{20--30~minutes} until heated through. Add a splash of beef stock if sauce has reduced too much during storage
    \item The braise does not freeze well due to the delicate texture of the meat and the nature of the sauce emulsion
\end{itemize}

\subsection*{Serving Suggestions}
\begin{itemize}
    \item Serve over creamy mashed potatoes, allowing them to absorb the silky sauce
    \item Accompany with supplemental beef stock gravy on the side for guests who prefer additional sauce
    \item A simple green salad with vinaigrette provides brightness and cuts through the richness
    \item Crusty bread for soaking up every drop of sauce is essential
    \item The same Burgundy wine used in the braise pairs beautifully for drinking; continue with it at table
    \item Garnish generously with fresh \textbf{parsley} just before service for color and herbaceous freshness
    \item Allow guests to rest the platter for \textit{5~minutes} before serving, preserving the meat's internal juices
\end{itemize}

\end{multicols}
}

\end{document}
