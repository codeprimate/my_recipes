\documentclass[11pt,letterpaper]{article}
\usepackage{fontspec}
\usepackage{tocloft}
\usepackage{multicol}
\usepackage[left=1.4in,right=1.5in,top=1in,bottom=1in]{geometry}
\setmainfont[Scale=1.3,AutoFakeBold=1.5,AutoFakeSlant=0.3]{Doves Type}

\title{Tex-Mex Pulled Chicken •}
\author{}
\date{}

\begin{document}

\maketitle
\thispagestyle{empty}

\section*{Ingredients}
\setlength{\columnsep}{20pt}
\begin{multicols}{2}
\noindent
    Chicken thighs, \dotfill 3-5 lbs \\
    Yellow onion, large \dotfill 1 \\
    Garlic cloves \dotfill 4-6 \\
    Mexican spice blend \dotfill 5 Tbsp \\
    Herdez salsa ranchera \dotfill 7 oz. can \\
    \columnbreak
    Chicken stock \dotfill 2½ cups \\
    Caldo de tomate \dotfill 2 Tbsp \\
    Bay leaves \dotfill 2 \\
    Limes \dotfill 2 \\
    Fresh cilantro, chopped \dotfill ½ cup \\
    Kosher salt \dotfill 2~tsp.
\end{multicols}

\section*{Directions}

\noindent
Preheat oven to \textit{275°F} ---
Pat \textbf{chicken~thighs} completely dry ---
Large dice \textbf{onion} ---
Mince \textbf{garlic} ---
Chop \textbf{cilantro} ---
Juice \textbf{limes} (¼~cup)

\begin{enumerate}
    \item Pat \textbf{chicken~thighs} completely dry with paper towels. Season skin side lightly with \textbf{kosher~salt}. Heat dutch oven over medium-high heat with no added fat. Place \textbf{thighs} skin-side down and do not move them. Sear for \textit{8-10~minutes} until deeply golden-brown and skin releases easily from the pan. Flip briefly for \textit{1-2~minutes}, then remove \textbf{chicken} to a plate. Pour rendered fat into a bowl, reserving 2~Tbsp in the pot along with all the fond.
    
    \item Reduce heat to medium. Add diced \textbf{onion} to the fond, stirring to coat in fat. Cook \textit{5-7~minutes}, stirring occasionally, until edges start browning—not just translucent, we want some color and caramelization.
    
    \item Add 4-5~Tbsp~\textbf{Mexican~spice~blend} to the \textbf{onions}. Cook \textit{45-60~seconds}, stirring constantly, until a toasted fragrance develops. You will smell the distinct shift from raw spice to bloomed aromatic. Add \textbf{salsa} to stop the bloom. Scrape all the fond from the bottom of the pot.
    
    \item Add minced \textbf{garlic} to the tomato mixture and cook for \textit{30~seconds}. Add 2½~cups \textbf{chicken~stock}, 2~Tbsp \textbf{caldo~de~tomate}, and 2~\textbf{bay~leaves}. Stir well and bring to a simmer. Taste the braising liquid—it should be intensely flavored and slightly too salty. If not sufficiently seasoned, add another ½~Tbsp~\textbf{caldo~de~tomate}.
    
    \item Return \textbf{chicken~thighs} to the pot skin-side up. Liquid should come halfway up the \textbf{chicken}; add more \textbf{stock} if needed. Bring to a bare simmer on the stovetop, then cover with a tight-fitting lid. Transfer to the preheated \textit{275°F} oven and braise for \textit{2½-3~hours}.
    
    \item Check at \textit{2½~hours}: the meat should pull from the bone with zero resistance. If still slightly firm, continue braising for another \textit{30~minutes}.
    
    \item Remove \textbf{chicken} from the pot and let cool for \textit{10~minutes} to make handling easier. Remove skin and bones, etc.
    
    \item In a large bowl, shred the \textbf{chicken} into bite-size pieces, maintaining some texture. Add ½~cup of the \textbf{braising~liquid}, 2~Tbsp. \textbf{reserved~chicken~fat}, juice of 2~\textit{limes}, and ½~cup~\textbf{cilantro}. Fold gently to combine. Add more \textbf{braising~liquid} as needed until moist but not soupy.
    
    \item Serve immediately in tacos, or hold warm. This \textbf{chicken} benefits from resting \textit{15-30~minutes} to allow flavors to marry fully.
\end{enumerate}

\newpage

% Begin compact two-column layout
{\small
\setlength{\columnsep}{20pt}
\setlength{\multicolsep}{6pt}
\begin{multicols}{2}
\setlength{\parindent}{0pt}
\setlength{\parskip}{4pt}

\subsection*{Equipment Required}
\begin{itemize}
    \item Dutch oven (5-7 quart capacity, oven-safe to \textit{300°F})
    \item Tight-fitting lid for dutch oven
    \item Large plate for holding seared chicken
    \item Fine mesh strainer
    \item Medium saucepan (for reduction)
    \item Small bowl (for reserved fat)
    \item Cutting board and chef's knife
    \item Measuring cups and spoons
    \item Wooden spoon or silicone spatula
    \item Paper towels
    \item Tongs or fork (for handling chicken)
    \item Ladle or large spoon (for skimming fat)
    \item Two forks (for shredding chicken)
    \item Citrus juicer (optional but helpful)
\end{itemize}

\subsection*{Mise en Place}
\begin{itemize}
    \item Remove \textbf{chicken~thighs} from refrigerator \textit{30~minutes} before cooking to take chill off
    \item Pat \textbf{chicken} completely dry—moisture prevents proper searing
    \item Prepare all vegetables before starting: dice \textbf{onion}, mince \textbf{garlic}, chop \textbf{cilantro}
    \item Measure \textbf{spice~blend} and set near stove for quick addition
    \item Have all liquids measured and ready—braising moves quickly after sear
    \item Open \textbf{salsa} can and have ready to add immediately after spices
    \item Juice \textbf{limes} after \textbf{chicken} goes in oven (keeps juice fresh)
\end{itemize}

\subsection*{Ingredient Tips}
\begin{itemize}
    \item Bone-in, skin-on thighs are essential—boneless will not achieve the same tenderness or gelatin-rich sauce
    \item Choose thighs of similar size for even cooking
    \item \textbf{Herdez~salsa~ranchera} is preferred for its roasted pepper depth; substitute with quality jarred salsa if needed
    \item If your \textbf{Mexican~spice~blend} is older than \textit{6~months}, increase quantity by 1~Tbsp as potency fades
    \item Use fresh \textbf{garlic} only—powdered garlic is already in the spice blend
    \item \textbf{Chicken~stock} quality matters; homemade or low-sodium store-bought preferred
    \item Reserve rendered \textbf{chicken~fat}—it's gold for finishing and adds authentic richness
    \item Fresh \textbf{Mexican~limes} (Key limes) are more authentic but regular Persian limes work well
\end{itemize}

\subsection*{Preparation Tips}
\begin{itemize}
    \item The sear is critical: don't rush it. Deeply golden skin = maximum fond = concentrated flavor
    \item Resist moving the \textbf{chicken} during searing—let it release naturally when ready
    \item The fond (brown bits) contains concentrated Maillard compounds; scrape thoroughly
    \item Brown the \textbf{onion} edges—this adds another layer of caramelization
    \item Bloom spices until fragrant shift occurs, usually \textit{45-60~seconds}—under-blooming leaves raw taste, over-blooming creates bitterness
    \item Add \textbf{salsa} immediately after blooming to halt cooking and prevent burning
    \item Taste braising liquid before adding \textbf{chicken}—should be intensely flavored
    \item During braise, liquid should barely simmer; vigorous bubbling means oven is too hot
    \item Skim fat before reduction—too much fat makes sauce greasy rather than rich
    \item Reduce sauce properly—half volume concentrates flavor exponentially
    \item Add \textbf{lime~juice} and \textbf{cilantro} only at the end to preserve brightness
    \item Shred \textbf{chicken} to bite-size pieces, not fine threads—texture matters
\end{itemize}

\subsection*{Make Ahead \& Storage}
\begin{itemize}
    \item Can be made up to \textit{3~days} ahead through step 10; refrigerate in sauce
    \item Flavor actually improves after \textit{24~hours} as spices continue integrating
    \item If made ahead, add fresh \textbf{lime~juice} and \textbf{cilantro} when reheating
    \item Reheat gently in covered pot over low heat, stirring occasionally
    \item Add splash of \textbf{stock} if sauce has tightened during refrigeration
    \item Can be frozen for up to \textit{3~months}; thaw overnight in refrigerator
    \item After thawing, refresh with additional \textbf{lime~juice} and \textbf{cilantro}
    \item Leftover rendered \textbf{chicken~fat} can be refrigerated for \textit{1~week} and used for sautéing vegetables or making rice
\end{itemize}

\subsection*{Serving Suggestions}
\begin{itemize}
    \item Serve in warm corn or flour tortillas with diced \textbf{onion}, \textbf{cilantro}, and \textbf{lime} wedges
    \item Excellent with pickled jalapeños, sliced radishes, or quick-pickled red onions
    \item Top with crumbled queso fresco, cotija, or shredded Monterey Jack
    \item Pair with Mexican rice, refried beans, or charred street corn
    \item Makes exceptional enchiladas—use as filling with verde or roja sauce
    \item Works beautifully in tortas, quesadillas, or burrito bowls
    \item For tostadas: crisp tortillas, spread refried beans, top with \textbf{chicken}, shredded lettuce, Mexican crema
    \item Leftover \textbf{chicken} makes outstanding chilaquiles for breakfast
    \item Consider crispy skin garnish: remove skin before braising, roast separately until crispy, crumble over tacos
    \item Traditional accompaniments: salsa verde, salsa roja, pickled carrots and jalapeños, lime wedges, fresh tortilla chips
\end{itemize}

\subsection*{Flavor Profile Notes}
\begin{itemize}
    \item This recipe achieves depth through layering: seared proteins, bloomed spices, reduced sauce, bright finish
    \item The \textbf{Mexican~spice~blend} with cinnamon and cloves provides northern Mexican/mole-adjacent complexity
    \item Gelatin from bones creates luxurious mouthfeel and helps sauce cling to meat
    \item Reserved \textbf{chicken~fat} adds authentic richness without greasiness when used judiciously
    \item The acid-fat-salt-umami balance is critical: \textbf{lime} cuts richness, \textbf{salt} amplifies everything, bouillon/caldo provide savory depth
    \item Finishing with fresh \textbf{cilantro} and \textbf{lime} preserves brightness that would be lost during long braise
\end{itemize}

\end{multicols}
}

\end{document}
