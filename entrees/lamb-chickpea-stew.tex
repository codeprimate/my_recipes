\documentclass[11pt,letterpaper]{article}
\usepackage{fontspec}
\usepackage{tocloft}
\usepackage{multicol}
\usepackage[left=1.4in,right=1.5in,top=1in,bottom=1in]{geometry}
\setmainfont[Scale=1.2,AutoFakeBold=1.5,AutoFakeSlant=0.3]{Doves Type}

\title{Lamb and Chickpea Stew}
\author{}
\date{}

\begin{document}

\maketitle
\thispagestyle{empty}

\section*{Ingredients}
\setlength{\columnsep}{20pt}
\begin{multicols}{2}
\noindent
    Dried chickpeas \dotfill 2 cups \\
    Kosher salt (for soaking) \dotfill 3 Tbsp. \\
    Lamb shoulder* \dotfill 3 lbs. \\
    Kosher salt (for lamb) \dotfill 1 Tbsp. \\
    Black pepper (for lamb) \dotfill 1 tsp. \\
    Vegetable oil \dotfill ¼ cup \\
    Onions, large \dotfill 2 \\
    Garlic cloves \dotfill 8-10 \\
    Fresh ginger \dotfill 2 Tbsp. \\
    Ground cumin \dotfill 5 tsp. \\
    Ground coriander \dotfill 5 tsp. \\
    Sweet paprika \dotfill 4 tsp. \\
    \columnbreak
    Smoked paprika \dotfill 2 tsp. \\
    Ground turmeric \dotfill 2 tsp. \\
    Ground cinnamon \dotfill 1 tsp. \\
    Tomato paste \dotfill 6 Tbsp. \\
    Ras el hanout \dotfill 5 tsp. \\
    Kosher salt (for stew) \dotfill 1 Tbsp. \\
    Chicken broth or lamb stock \dotfill 3 cups \\
    Diced tomatoes \dotfill 28 oz. (2 cans) \\
    Dried apricots \dotfill 1½ cups \\
    Fresh cilantro \dotfill 1 cup \\
    Fresh lemon juice \dotfill ½ cup \\
    Honey \dotfill 3 Tbsp. \\
    Harissa \dotfill for table \\
\end{multicols}

{\small\textit{*Beef chuck shoulder may substitute for lamb (same method and timing)*}}

\section*{Directions}

\noindent
\textbf{Night Before:} Combine \textbf{chickpeas}, water, and 3~Tbsp. \textbf{salt} in a large bowl. Cover and refrigerate overnight (\textit{8-12~hours}). ---
Preheat oven to \textit{300°F} ---
Drain and rinse soaked \textbf{chickpeas} ---
Pat dry \textbf{lamb} and cut into 1½" cubes ---
Season \textbf{lamb} with 1~Tbsp. \textbf{salt} and 1~tsp. \textbf{pepper} ---
Dice \textbf{onions} ---
Mince \textbf{garlic} ---
Grate \textbf{ginger} ---
Chop \textbf{apricots} into ¼" pieces ---
Measure all \textbf{spices} ---
Chop \textbf{cilantro}

\begin{enumerate}
    \item Heat 6-quart enameled dutch oven over medium-high heat. Add 2~Tbsp. \textbf{oil}. Working in 3-4~batches to avoid crowding, brown \textbf{lamb~cubes} on multiple sides until deeply caramelized, about \textit{8-10~minutes} per batch, adding more \textbf{oil} as needed. Transfer browned \textbf{lamb} to a large bowl and set aside.
    
    \item Reduce heat to medium. If pot is dry, add final portion of \textbf{oil}. Add diced \textbf{onions} and cook, stirring occasionally, until softened and golden brown, about \textit{8-10~minutes}. Add minced \textbf{garlic} and grated \textbf{ginger}; cook, stirring constantly, for \textit{1-2~minutes} until fragrant.
    
    \item Add \textbf{cumin}, \textbf{coriander}, \textbf{sweet~paprika}, \textbf{smoked~paprika}, \textbf{turmeric}, and \textbf{cinnamon} to the pot. Stir constantly for \textit{45-60~seconds} until spices are darkened and very fragrant. Add \textbf{tomato~paste} and stir constantly, scraping to prevent scorching, for \textit{2-3~minutes} until paste is brick-red and caramelized.
    
    \item Add \textbf{ras~el~hanout} and 1~Tbsp. \textbf{salt}; stir to combine. Immediately add \textbf{chicken~broth} and use a wooden spoon to scrape bottom of pot vigorously, releasing all browned bits. Add \textbf{diced~tomatoes} with their juices. Bring to a simmer.
    
    \item Return browned \textbf{lamb} and any accumulated juices to pot. Add drained \textbf{chickpeas} and 1~cup chopped \textbf{apricots}. Stir to combine. The liquid should come about ¾ of the way up the solids; add additional \textbf{broth} if needed.
    
    \item Bring to a full simmer on stovetop. Cover with tight-fitting lid and transfer to preheated \textit{300°F} oven.
    
    \item Braise for \textit{3-3½~hours}, checking at \textit{2½~hours}. \textbf{Lamb} should be pull-apart tender and \textbf{chickpeas} should be creamy. If liquid level seems low at the \textit{2½~hour} check, add ½-1~cup hot \textbf{broth}. If stew seems too liquidy, crack lid slightly for final \textit{30-45~minutes}.
    
    \item Remove from oven. If sauce needs reducing, place uncovered pot on stovetop over medium heat and simmer for \textit{5-10~minutes} until thickened to coat the back of a spoon. If desired, skim excess fat from surface.
    
    \item Stir in remaining ½~cup chopped \textbf{apricots}, \textbf{fresh~cilantro}, \textbf{lemon~juice}, and \textbf{honey}. Taste and adjust seasoning with additional \textbf{salt} if needed. Let rest for \textit{10-15~minutes} before serving.
    
    \item Serve over \textbf{couscous} or \textbf{cooked~rice} with \textbf{harissa} on the side.
\end{enumerate}

\newpage

% Begin compact two-column layout
{\small
\setlength{\columnsep}{20pt}
\setlength{\multicolsep}{6pt}
\begin{multicols}{2}
\setlength{\parindent}{0pt}
\setlength{\parskip}{4pt}

\subsection*{Equipment Required}
\begin{itemize}
    \item 6-quart enameled dutch oven with tight-fitting lid
    \item Large mixing bowl (for soaking chickpeas)
    \item Large bowl or plate (for browned lamb)
    \item Cutting board and sharp knife
    \item Wooden spoon or heatproof spatula
    \item Measuring cups and spoons
    \item Microplane or fine grater (for ginger)
    \item Ladle
    \item Timer
\end{itemize}

\subsection*{Mise en Place}
\begin{itemize}
    \item Soak \textbf{chickpeas} the night before in salted water
    \item Allow \textit{45-60~minutes} total for prep work on day of cooking
    \item If using whole lamb shoulder, trim excess fat but leave some for flavor
    \item Cut \textbf{lamb} into uniform 1½" cubes for even cooking
    \item Prep all aromatics and measure all spices before beginning—once you start browning, the process moves quickly
    \item Have \textbf{broth} ready and warm for easier deglazing
\end{itemize}

\subsection*{Ingredient Tips}
\begin{itemize}
    \item Lamb shoulder is ideal for braising due to marbling and connective tissue; leg meat is leaner and won't be as tender
    \item Pereg or other quality \textbf{ras~el~hanout} blends work well; avoid dusty, stale spices
    \item Use whole spices and grind fresh for maximum flavor if possible
    \item San Marzano or fire-roasted \textbf{diced~tomatoes} add extra depth
    \item Turkish or California \textbf{apricots} are ideal; avoid overly sweet or sugared varieties
    \item Fresh \textbf{ginger} is essential; powdered won't provide the same brightness
    \item If making homemade \textbf{lamb~stock}, this elevates the dish significantly
\end{itemize}

\subsection*{Preparation Tips}
\begin{itemize}
    \item Don't rush the browning—deep caramelization is the foundation of flavor
    \item Work in small batches; crowding the pot steams meat instead of browning it
    \item The \textbf{tomato~paste} will threaten to scorch; keep stirring and scraping constantly during step 3
    \item Vigorous deglazing in step 4 is critical—every bit of fond adds flavor
    \item Starting the braise at a full simmer on the stovetop ensures immediate cooking when transferred to oven
    \item If your dutch oven lid doesn't seal tightly, cover pot with foil before adding lid to minimize evaporation
    \item Check liquid level at \textit{2½~hours}; ovens and pots vary, so adjustment may be needed
    \item \textbf{Lamb} texture varies by cut quality; check tenderness and extend cooking if needed
    \item The stew will continue to thicken as it rests; it should be slightly looser than desired final consistency when removed from oven
\end{itemize}

\subsection*{Make Ahead \& Storage}
\begin{itemize}
    \item This stew benefits from sitting; make up to \textit{3~days} ahead and refrigerate
    \item Fat will solidify on surface when chilled, making it easy to remove if desired
    \item Reheat gently on stovetop, adding \textbf{broth} if needed to restore consistency
    \item Add finishing ingredients (\textbf{cilantro}, \textbf{lemon~juice}, \textbf{honey}) only when reheating to serve
    \item Freezes well for up to \textit{3~months}; thaw overnight in refrigerator
    \item If freezing, slightly undercook (reduce time by \textit{30~minutes}) as reheating continues cooking
    \item Leftover stew thickens significantly; thin with \textbf{broth} or water when reheating
\end{itemize}

\subsection*{Serving Suggestions}
\begin{itemize}
    \item Serve over basmati rice, couscous, or with crusty bread
    \item Accompany with \textbf{harissa} for heat, plain yogurt for cooling contrast
    \item Garnish with additional \textbf{fresh~cilantro}, toasted sliced almonds, or sesame seeds
    \item A simple cucumber-tomato salad provides refreshing contrast
    \item Pairs beautifully with full-bodied red wines or Moroccan mint tea
    \item Consider topping with a poached or fried egg for brunch service
    \item Leftovers make excellent filling for savory hand pies or empanadas
\end{itemize}

\end{multicols}
}

\end{document}
