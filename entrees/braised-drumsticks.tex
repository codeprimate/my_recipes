\documentclass[11pt,letterpaper]{article}
\usepackage{fontspec}
\usepackage{tocloft}
\usepackage{multicol}
\usepackage[left=1.4in,right=1.5in,top=1in,bottom=1in]{geometry}
\setmainfont[Scale=1.2,AutoFakeBold=1.5,AutoFakeSlant=0.3]{Doves Type}

\title{Classic Braised Chicken Drumsticks with Pan Sauce}
\author{}
\date{}

\begin{document}

\maketitle
\thispagestyle{empty}

\section*{Ingredients}
\setlength{\columnsep}{20pt}
\begin{multicols}{2}
\noindent
    Chicken drumsticks \dotfill 12 (3-4 lbs.) \\
    Vegetable oil \dotfill 2 Tbsp. \\
    Butter \dotfill 6 Tbsp. \\
    Onions, large \dotfill 2 \\
    Carrots, large \dotfill 3-4 \\
    Celery stalks \dotfill 3-4 \\
    Garlic cloves \dotfill 10-12 (1 head) \\
    Tomato paste \dotfill 2 Tbsp. \\
    White wine (optional) \dotfill ½ cup \\
    \columnbreak
    Chicken stock \dotfill 5-6 cups \\
    Canned mushroom slices \dotfill 12 oz. \\
    Dried thyme \dotfill 2 tsp. \\
    Dried rosemary \dotfill 1½ tsp. \\
    Dried sage \dotfill 1 tsp. \\
    Bay leaves \dotfill 3 \\
    Dijon mustard \dotfill 2 Tbsp. \\
    MSG \dotfill ½ tsp. \\
    Lemon, zested \dotfill 1 \\
    All-purpose flour \dotfill 3 Tbsp. \\
    Kosher salt \dotfill 2 tsp. \\
    Black pepper \dotfill 1 tsp. \\
\end{multicols}

\section*{Directions}

\noindent
Preheat oven to \textit{275°F} ---
Peel \textbf{garlic cloves} ---
Cut \textbf{onions} into 8 wedges each ---
Cut \textbf{carrots} into 2-inch pieces ---
Cut \textbf{celery} into 2-inch pieces ---
Drain \textbf{mushrooms} ---
Zest \textbf{lemon} ---
Pat dry \textbf{drumsticks} ---
Season \textbf{drumsticks} generously with \textbf{salt} and \textbf{pepper}

\begin{enumerate}
    \item Heat \textbf{vegetable oil} in a large Dutch oven over medium-high heat. Working in batches, sear \textbf{drumsticks} until golden brown on all sides, about \textit{3-4~minutes} per side. Transfer to a plate and set aside.
    
    \item Reduce heat to medium. Add 4~Tbsp. \textbf{butter} to the Dutch oven. Once melted, add \textbf{onion} wedges, \textbf{carrot} pieces, and \textbf{celery} pieces. Sauté until vegetables begin to soften and develop color, about \textit{8-10~minutes}. Add \textbf{garlic cloves} and cook for \textit{2~minutes} more.
    
    \item Add \textbf{tomato paste} and cook, stirring constantly, until it darkens and becomes fragrant, about \textit{2-3~minutes}.
    
    \item If using \textbf{white wine}, add it now and scrape up any browned bits from the bottom of the pot. Cook until wine is nearly evaporated, about \textit{3-4~minutes}. If not using wine, proceed to next step.
    
    \item Add \textbf{chicken stock}, \textbf{drained mushrooms}, \textbf{dried thyme}, \textbf{dried rosemary}, \textbf{dried sage}, \textbf{bay leaves}, \textbf{Dijon mustard}, and \textbf{MSG}. Stir to combine and bring to a simmer.
    
    \item Return \textbf{drumsticks} to the pot, nestling them into the liquid (liquid should come halfway up the drumsticks). Cover with lid and transfer to oven. Braise for \textit{1½-2~hours}, until \textbf{chicken} is very tender and nearly falling off the bone.
    
    \item Remove pot from oven. Using tongs, carefully transfer \textbf{drumsticks} to a plate. Stir \textbf{lemon zest} into the braising liquid and taste for seasoning, adjusting \textbf{salt} and \textbf{pepper} as needed.
    
    \item To thicken sauce, knead together remaining 2~Tbsp. \textbf{butter} and 3~Tbsp. \textbf{flour} to form a smooth paste (beurre manié). Return pot to stovetop over medium heat. Whisk in beurre manié, a little at a time, until sauce reaches desired consistency. Simmer for \textit{3-5~minutes} to cook out flour taste.
    
    \item For crispy skin: Pat \textbf{drumsticks} dry and air fry at \textit{400°F} for \textit{5-7~minutes} until skin is crispy and caramelized. For freezing: skip this step and proceed directly to storage.
    
    \item Return \textbf{drumsticks} to sauce, or transfer to containers for freezing. Serve hot over rice or with crusty bread.
\end{enumerate}

\newpage

{\small
\setlength{\columnsep}{20pt}
\setlength{\multicolsep}{6pt}
\begin{multicols}{2}
\setlength{\parindent}{0pt}
\setlength{\parskip}{4pt}

\section*{Equipment Required}
\begin{itemize}
    \item Large Dutch oven (6-7 quart capacity)
    \item Large plate or rimmed baking sheet
    \item Tongs
    \item Wooden spoon or spatula
    \item Whisk
    \item Measuring cups and spoons
    \item Sharp knife and cutting board
    \item Microplane or zester
    \item Small bowl (for beurre manié)
    \item Ladle
    \item Air fryer (optional, for finishing)
    \item Freezer-safe containers (if freezing)
\end{itemize}

\subsection*{Mise en Place}
\begin{itemize}
    \item Remove \textbf{drumsticks} from refrigerator \textit{30~minutes} before cooking for even searing
    \item Prep all vegetables before starting - large uniform pieces ensure even cooking
    \item Have \textbf{stock} measured and nearby for quick addition
    \item Separate 2~Tbsp. \textbf{butter} for beurre manié from the 4~Tbsp. used for sautéing
    \item Zest \textbf{lemon} before cooking; reserve zest in small bowl
\end{itemize}

\subsection*{Ingredient Tips}
\begin{itemize}
    \item Choose \textbf{drumsticks} of similar size for even cooking
    \item Whole \textbf{garlic cloves} become sweet and tender when braised - they're delicious eaten whole
    \item Quality \textbf{chicken stock} makes a significant difference; low-sodium preferred for seasoning control
    \item If omitting \textbf{wine}, add 1-2~Tbsp. additional \textbf{lemon juice} at the end for brightness
    \item Canned \textbf{mushrooms} work well here; if using fresh, add them later to prevent overcooking
    \item \textbf{MSG} amplifies savory notes; omit if preferred without other adjustments
\end{itemize}

\subsection*{Preparation Tips}
\begin{itemize}
    \item Don't crowd \textbf{drumsticks} when searing - work in batches for proper browning
    \item Fond development is crucial - those browned bits add deep flavor
    \item Blooming \textbf{tomato paste} until it darkens intensifies umami
    \item Large vegetable pieces stay intact during long braise and provide textural contrast
    \item Check liquid level during braising - add more \textbf{stock} if needed
    \item Beurre manié should be added gradually while whisking to prevent lumps
    \item For stovetop braising: use lowest heat setting, check frequently to maintain bare simmer
    \item Pat \textbf{drumsticks} very dry before air frying for maximum crispness
\end{itemize}

\subsection*{Make Ahead \& Storage}
\begin{itemize}
    \item Can be made up to \textit{2~days} ahead and refrigerated
    \item Flavors improve overnight as they meld
    \item To freeze: cool completely, portion into freezer-safe containers with sauce, freeze up to \textit{3~months}
    \item Freeze flat in zip-top bags for space efficiency
    \item Thaw overnight in refrigerator before reheating
    \item Reheat gently on stovetop or in \textit{325°F} oven until warmed through, about \textit{25-30~minutes}
    \item If reheating from frozen, add \textit{15-20~minutes} to heating time
    \item Air fry for crispy skin after reheating if desired
    \item Sauce may separate when frozen; whisk while reheating to re-emulsify
\end{itemize}

\subsection*{Serving Suggestions}
\begin{itemize}
    \item Serve over jasmine rice cooked in \textbf{chicken stock} with dried soup vegetables and \textbf{bay leaf}
    \item Excellent with mashed potatoes, egg noodles, or crusty bread for sopping up sauce
    \item Garnish with fresh parsley, thyme, or chives if available
    \item Vegetables from the braise are delicious served alongside
    \item Pairs well with simple green salad or roasted green beans
    \item For a complete meal: add roasted root vegetables or sautéed greens
    \item Leftovers make excellent chicken and rice soup - add extra \textbf{stock} and shred \textbf{chicken}
\end{itemize}

\end{multicols}
}

\end{document}