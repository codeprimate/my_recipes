\documentclass[11pt,letterpaper]{article}
\usepackage{fontspec}
\usepackage{tocloft}
\usepackage{multicol}
\usepackage[left=1.4in,right=1.5in,top=1in,bottom=1in]{geometry}
\setmainfont[Scale=1.2,AutoFakeBold=1.5,AutoFakeSlant=0.3]{Doves Type}

\title{Dutch Oven Braised Pork Ribs}
\author{}
\date{}

\begin{document}

\maketitle
\thispagestyle{empty}


\section*{Ingredients}
\setlength{\columnsep}{20pt}
\begin{multicols}{2}
\noindent
    Pork rib rack \dotfill 4 lbs. \\
    Pork or chicken stock \dotfill 3 cups \\
    Apple cider \dotfill 8-12oz \\
    Soy sauce \dotfill ¼ cup \\
    Worcestershire sauce \dotfill 2 Tbsp. \\
    Brown sugar \dotfill 2 Tbsp. \\
    Tomato paste \dotfill 4 Tbsp. \\
    Onion, medium \dotfill 1 \\
    Garlic cloves \dotfill 1 bulb \\
    Bay leaves \dotfill 2 \\
    \columnbreak
    Black peppercorns \dotfill 1 tsp. \\
    Dried thyme \dotfill 1 tsp. \\
    Dried oregano \dotfill ½ tsp. \\
    Kosher salt \dotfill 2 tsp. \\
    Black pepper \dotfill 1 tsp. \\
    Garlic powder \dotfill 1 tsp. \\
    Onion powder \dotfill 1 tsp. \\
    Smoked paprika \dotfill ½ tsp. \\
    Chipotle powder \dotfill ¼ tsp. \\
    Hot paprika \dotfill ¼ tsp. \\
    MSG \dotfill ½ tsp. \\
\end{multicols}

\section*{Directions}

\noindent
Preheat oven to \textit{275°F} ---
Pat dry \textbf{pork~ribs} ---
Quarter \textbf{onion} ---
Smash \textbf{garlic~cloves}

\begin{enumerate}
    \item Cut \textbf{pork~rib~rack} into quarters for easier handling.
    
    \item In a small bowl, combine 2~tsp. \textbf{kosher~salt}, 1~tsp. \textbf{black~pepper}, 1~tsp. \textbf{garlic~powder}, 1~tsp. \textbf{onion~powder}, ½~tsp. \textbf{smoked~paprika}, , ¼~tsp. \textbf{hot~paprika}, ¼~tsp. \textbf{chipotle~powder}, and ½~tsp. \textbf{MSG}. Rub mixture evenly over all surfaces of \textbf{rib~quarters}.
    
    \item Arrange seasoned \textbf{ribs} in Dutch oven (they can overlap slightly), tucking quartered \textbf{onion}, smashed \textbf{garlic}, \textbf{bay~leaves}, \textbf{peppercorns}, \textbf{dried~thyme}, and \textbf{dried~oregano} around and under \textbf{ribs}.
    
    \item In a medium bowl, whisk together \textbf{stock}, \textbf{apple~cider}, \textbf{soy~sauce}, \textbf{Worcestershire~sauce}, \textbf{brown~sugar}, and \textbf{tomato~paste} until well combined. Pour over \textbf{ribs}.
    
    \item Cover Dutch oven tightly with lid and place in oven. Braise for \textit{4-4½~hours} until \textbf{ribs} are fork-tender and meat pulls easily from bone.
    
    \item Remove from oven. Carefully transfer \textbf{ribs} from braising liquid as needed for intended use (BBQ finishing or soup preparation). Reserve braising liquid with rendered fat for soup base.
\end{enumerate}

\newpage

{\small
\setlength{\columnsep}{20pt}
\setlength{\multicolsep}{6pt}
\begin{multicols}{2}
\setlength{\parindent}{0pt}
\setlength{\parskip}{4pt}

\subsection*{Equipment Required}
\begin{itemize}
    \item Dutch oven (6-7 quart capacity minimum)
    \item Small bowl (for dry rub)
    \item Medium bowl (for braising liquid)
    \item Whisk
    \item Measuring cups and spoons
    \item Tongs or slotted spoon
    \item Sharp knife and cutting board
    \item Large plate or platter (for finished ribs)
\end{itemize}

\subsection*{Mise en Place}
\begin{itemize}
    \item Pat \textbf{ribs} completely dry with paper towels before seasoning
    \item Have all liquid ingredients measured and ready to whisk together
    \item Prepare aromatics (quarter \textbf{onion}, smash \textbf{garlic}) before starting
    \item Ensure Dutch oven lid fits tightly to prevent moisture loss
\end{itemize}

\subsection*{Ingredient Tips}
\begin{itemize}
    \item Use homemade or low-sodium \textbf{stock} for better control of final seasoning
    \item \textbf{Apple~cider} should be unfiltered for more flavor; apple juice works as substitute
    \item \textbf{MSG} is optional but adds significant umami depth without altering flavor profile
    \item \textbf{Tomato~paste} should be whisked thoroughly into liquid to prevent clumping
    \item Baby back ribs can substitute for spare ribs; reduce cooking time by \textit{30~minutes}
\end{itemize}

\subsection*{Preparation Tips}
\begin{itemize}
    \item Cut \textbf{rack} into quarters through bone for easier handling and better liquid exposure
    \item Apply dry rub generously but don't let it sit more than \textit{15~minutes} before braising (salt will draw moisture)
    \item \textbf{Ribs} can overlap in Dutch oven but ensure liquid reaches all pieces
    \item Don't lift lid during first \textit{2½~hours} of cooking to maintain consistent temperature
    \item Test doneness by inserting fork between bones - meat should offer minimal resistance
    \item For BBQ finishing: remove \textbf{ribs} gently to preserve meat integrity for air frying
    \item For soup: pull meat directly from bones while warm, return to braising liquid
\end{itemize}

\subsection*{Make Ahead \& Storage}
\begin{itemize}
    \item Entire braise can be completed up to \textit{2~days} ahead and refrigerated in Dutch oven
    \item Fat will solidify on surface when cold; leave intact to protect meat, or remove if desired before reheating
    \item Reheat covered at \textit{300°F} for \textit{30-40~minutes} until warmed through
    \item Braising liquid without \textbf{ribs} can be frozen for up to \textit{3~months}
    \item If freezing liquid, cool completely and skim excess fat before freezing
\end{itemize}

\subsection*{Usage Notes}
\begin{itemize}
    \item This recipe yields \textbf{ribs} suitable for both BBQ finishing and soup preparation
    \item For BBQ: transfer \textbf{ribs} to plate, pat dry, brush with sauce, finish in air fryer at \textit{400°F} for \textit{3-5~minutes}
    \item For soup: leave \textbf{ribs} in liquid, pull meat from bones, proceed with bean soup recipe
    \item Braising liquid contains rendered fat and collagen - ideal soup base requiring no additional stock
    \item Taste braising liquid before seasoning soup; it's already well-salted from the braise
    \item \textbf{Bay~leaves} and \textbf{peppercorns} can be strained out before using liquid for soup
\end{itemize}

\end{multicols}
}

\end{document}
