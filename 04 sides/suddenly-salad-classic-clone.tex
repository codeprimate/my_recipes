\documentclass[11pt,letterpaper]{article}
\usepackage{fontspec}
\usepackage{tocloft}
\usepackage{multicol}
\usepackage{fancyhdr}
\usepackage{nicefrac}
\usepackage[left=1.3in,right=1.3in,top=1in,bottom=1in]{geometry}
\setmainfont[Scale=1.1,AutoFakeBold=1.5,AutoFakeSlant=0.3]{Doves Type}

% Prevent page breaks within list items
\widowpenalty=10000
\clubpenalty=10000
\interlinepenalty=500

\title{Suddenly Salad Classic Clone}
\author{}
\date{}

% Configure fancy header
\pagestyle{fancy}
\fancyhf{} % Clear all header and footer fields
\fancyhead[C]{\textit{Suddenly Salad Classic Clone}} % Center header with recipe title
\fancyfoot[C]{\thepage} % Center page number at bottom
\renewcommand{\headrulewidth}{0pt} % Remove header line
\renewcommand{\footrulewidth}{0pt} % Remove footer line

\begin{document}

\maketitle
\thispagestyle{empty}

\section*{Ingredients}
\setlength{\columnsep}{20pt}
\begin{multicols}{2}
\noindent
    Rotini \dotfill 8 oz. \\
    Dried oregano \dotfill 1 Tbsp. \\
    Dried basil \dotfill 1 tsp. \\
    Dried parsley \dotfill 1 tsp. \\
    Garlic powder \dotfill 1 tsp. \\
    Onion powder \dotfill 1 tsp. \\
    Parmesan cheese, grated \dotfill 2 Tbsp. \\
    \columnbreak
    Citric acid/TruLemon powder \dotfill \nicefrac{1}{2} tsp. \\
    Sugar \dotfill \nicefrac{1}{2} tsp. \\
    Salt \dotfill \nicefrac{3}{4} tsp. \\
    Cornstarch \dotfill \nicefrac{1}{2} tsp. \\
    Dried red bell pepper flakes \dotfill \nicefrac{1}{4} tsp. \\
    Vegetable oil \dotfill 2 Tbsp. \\
    Cold water \dotfill 3 Tbsp. \\
\end{multicols}

\section*{Directions}

\noindent
Combine all seasoning mix ingredients in \textit{Small Bowl~\#1} ---
Bring large pot of salted water to boil for \textbf{pasta} ---
Have \textbf{oil} and \textbf{water} measured and ready

\begin{enumerate}
    \item In \textit{Small Bowl~\#1}, whisk together 1~Tbsp. \textbf{dried oregano}, 1~tsp. \textbf{dried basil}, 1~tsp. \textbf{dried parsley}, 1~tsp. \textbf{garlic powder}, 1~tsp. \textbf{onion powder}, 2~Tbsp. grated \textbf{Parmesan cheese}, \nicefrac{1}{2}~tsp. \textbf{citric acid}, \nicefrac{1}{2}~tsp. \textbf{sugar}, \nicefrac{3}{4}~tsp. \textbf{salt}, \nicefrac{1}{2}~tsp. \textbf{cornstarch}, and \nicefrac{1}{4}~tsp. \textbf{dried red bell pepper flakes} until evenly combined. The mixture should appear uniform with no clumps and have a fragrant herby aroma.
    
    \item Cook 8~oz. \textbf{tri-color rotini} in boiling salted water according to package directions until al dente, typically \textit{8--10~minutes}. Pasta is done when tender but still slightly firm to the bite with no raw flour taste in the center. Drain thoroughly in colander and rinse briefly under cold water to stop cooking and cool slightly, about \textit{30~seconds}. Drain well and transfer to \textit{Large Bowl~\#1}.
    
    \item Add 2~Tbsp. \textbf{vegetable oil} and 3~Tbsp. \textbf{cold water} to prepared seasoning mix (\textit{Small Bowl~\#1}). Whisk vigorously for \textit{30~seconds} until well blended and slightly emulsified. The dressing should appear uniform with no separated oil and have a pourable but slightly thickened consistency.
    
    \item Pour dressing (\textit{Small Bowl~\#1}) over cooked \textbf{pasta} in \textit{Large Bowl~\#1} and toss thoroughly with tongs or a large spoon for \textit{1--2~minutes} until every piece is evenly coated. The pasta should appear glossy and lightly coated with no pools of dressing at the bottom of the bowl.
    
    \item Cover bowl tightly with plastic wrap and refrigerate for at least \textit{2~hours} or up to \textit{24~hours} to allow flavors to meld. The salad is ready when chilled throughout, with pasta having absorbed the dressing and flavors well integrated.
\end{enumerate}

\newpage

{\footnotesize
\setlength{\columnsep}{20pt}
\setlength{\multicolsep}{6pt}
\begin{multicols}{2}
\setlength{\parindent}{0pt}
\setlength{\parskip}{2pt}
\setlength{\itemsep}{0pt}
\setlength{\parsep}{0pt}

\section*{Equipment Required}
\begin{itemize}
    \item Large pot (for cooking pasta)
    \item Colander
    \item Large bowl (1)
    \item Small bowl (1)
    \item Whisk
    \item Measuring spoons
    \item Tongs or large spoon
    \item Plastic wrap
\end{itemize}

\section*{Hints and Notes}

\subsection*{Yield}
\begin{itemize}
    \item Serves 4--6 as a side dish
    \item Equivalent to one standard box mix
\end{itemize}

\subsection*{Mise en Place}
\begin{itemize}
    \item \textit{Small Bowl~\#1} --- seasoning mix: 1~Tbsp. \textbf{oregano}, 1~tsp. \textbf{basil}, 1~tsp. \textbf{parsley}, 1~tsp. \textbf{garlic powder}, 1~tsp. \textbf{onion powder}, 2~Tbsp. \textbf{Parmesan}, \nicefrac{1}{2}~tsp. \textbf{citric acid}, \nicefrac{1}{2}~tsp. \textbf{sugar}, \nicefrac{3}{4}~tsp. \textbf{salt}, \nicefrac{1}{2}~tsp. \textbf{cornstarch}, \nicefrac{1}{4}~tsp. \textbf{red bell pepper flakes}; then becomes dressing when 2~Tbsp. \textbf{oil} and 3~Tbsp. \textbf{water} are added
    \item \textit{Large Bowl~\#1} --- cooked \textbf{pasta} for tossing and chilling (about 4~cups cooked)
    \item Measure \textbf{oil} and \textbf{water} before starting; have ready
    \item No advance prep required beyond measuring ingredients
\end{itemize}

\subsection*{Ingredient Tips}
\begin{itemize}
    \item \textbf{Tri-color rotini}: Use any brand; the spinach and tomato pasta add color but don't significantly affect flavor. Regular rotini works fine if tri-color is unavailable.
    \item \textbf{Citric acid}: Available in the canning or baking section of most grocery stores. Substitute 1~tsp. \textbf{TruLemon} powder if citric acid is unavailable—it provides similar tangy flavor.
    \item \textbf{Parmesan}: Use pre-grated or finely grated fresh Parmesan. Avoid coarse-grated cheese which won't incorporate smoothly into the dressing.
    \item \textbf{Dried red bell pepper flakes}: Look for dried red bell pepper in the spice aisle, or substitute a small pinch of paprika (not smoked).
    \item \textbf{Vegetable oil}: Canola, soybean, or sunflower oil work well. Avoid olive oil which has too strong a flavor for this light dressing.
    \item \textbf{Cornstarch}: Helps thicken and bind the dressing; don't omit.
\end{itemize}

\subsection*{Preparation Tips}
\begin{itemize}
    \item Cook \textbf{pasta} al dente—slightly underdone is better than overcooked, as pasta continues to absorb liquid while chilling
    \item Rinse cooked \textbf{pasta} briefly to cool it and wash away excess starch, which helps the dressing coat evenly
    \item Whisk \textbf{dressing} vigorously to create a slight emulsion; this helps the oil and water blend and prevents separation
    \item Toss \textbf{pasta} thoroughly—every piece should be coated. Use tongs or large spoon and toss for at least \textit{1--2~minutes}
    \item The salad tastes best after \textit{2--4~hours} of chilling when flavors have had time to develop
    \item Pasta absorbs dressing as it sits; if the salad seems dry after chilling, add 1--2~Tbsp. \textbf{water} and toss before serving
    \item Taste before serving—you may want to add a small pinch of \textbf{salt} or more \textbf{citric acid} for extra tang
\end{itemize}

\subsection*{Make Ahead \& Storage}
\begin{itemize}
    \item \textbf{Seasoning mix} can be made in advance and stored in an airtight container at room temperature for up to \textit{3~months}
    \item Prepared \textbf{pasta salad} keeps refrigerated for \textit{3--4~days} in an airtight container
    \item Add a splash of \textbf{water} (1--2~Tbsp.) and toss before serving leftovers, as pasta absorbs dressing over time
    \item Not recommended for freezing—pasta texture suffers
    \item Best served within \textit{24~hours} for optimal texture and flavor
\end{itemize}

\subsection*{Serving Suggestions}
\begin{itemize}
    \item Serve as a simple side dish with grilled meats, burgers, or sandwiches
    \item This is a base recipe—add-ins are up to you: cherry tomatoes, bell peppers, red onion, black olives, mozzarella cubes, salami, or pepperoni are all traditional additions
    \item For a more substantial salad, add diced cooked chicken or canned chickpeas
    \item Garnish with fresh basil or parsley for color and freshness
    \item Serve chilled or at room temperature; flavor is best when not ice-cold
    \item Pairs well with barbecue, picnics, potlucks, and casual gatherings
\end{itemize}

\end{multicols}
}

\end{document}
