\documentclass[11pt,letterpaper]{article}
\usepackage{fontspec}
\usepackage{tocloft}
\usepackage{multicol}
\usepackage[left=1.4in,right=1.5in,top=1in,bottom=1in]{geometry}
\setmainfont[Scale=1.2,AutoFakeBold=1.5,AutoFakeSlant=0.3]{Doves Type}

\title{Caldo de Carnitas y Frijoles Negros}
\author{}
\date{}

\begin{document}

\maketitle
\thispagestyle{empty}

\section*{Ingredients}
\setlength{\columnsep}{20pt}
\begin{multicols}{2}
\noindent
    Carnitas consommé \dotfill 2 cups \\
    Water \dotfill 3 cups \\
    Shredded carnitas meat \dotfill 1½ cups \\
    Black beans \dotfill 2 (15 oz.) cans \\
    Rotel diced tomatoes \dotfill 1 (10 oz.) can \\
    Yellow onion, medium \dotfill 1 \\
    Garlic cloves \dotfill 4 \\
    Celery stalks \dotfill 2 \\
    Carrots, medium \dotfill 2 \\
    \columnbreak
    Vegetable oil \dotfill 2 Tbsp. \\
    Ground cumin \dotfill 1 tsp. \\
    Mexican oregano \dotfill 1 tsp. \\
    Bay leaves \dotfill 2 \\
    Kosher salt \dotfill 1 tsp. \\
    Black pepper \dotfill ½ tsp. \\
    Fresh cilantro \dotfill ½ cup \\
    Lime juice \dotfill 3 Tbsp. \\
    Lime wedges \dotfill for serving \\
\end{multicols}

\section*{Directions}

\noindent
Dice \textbf{onion}, \textbf{celery}, and \textbf{carrots} into ¼-inch pieces ---
Mince \textbf{garlic} ---
Drain and rinse \textbf{black beans} ---
Chop \textbf{cilantro} ---
Juice \textbf{limes} ---
Warm \textbf{carnitas meat} if refrigerated

\begin{enumerate}
    \item Heat \textbf{vegetable oil} in a large Dutch oven or heavy-bottomed pot over medium heat. Add diced \textbf{onion}, \textbf{celery}, and \textbf{carrots}. Cook, stirring occasionally, until vegetables begin to soften, about \textit{8-10~minutes}.
    
    \item Add minced \textbf{garlic}, \textbf{cumin}, and \textbf{Mexican oregano}. Cook, stirring constantly, until fragrant, about \textit{1~minute}.
    
    \item Add \textbf{Rotel tomatoes} with their juice and cook for \textit{3-4~minutes}, allowing some liquid to evaporate and flavors to concentrate.
    
    \item Pour in \textbf{carnitas consommé} and \textbf{water}. Add \textbf{bay leaves}, \textbf{salt}, and \textbf{black pepper}. Bring to a boil, then reduce heat and simmer for \textit{15~minutes}.
    
    \item Add \textbf{black beans} and \textbf{carnitas meat}. Simmer for additional \textit{10-15~minutes} until vegetables are tender and flavors are well integrated.
    
    \item Remove \textbf{bay leaves}. Taste and adjust seasoning with additional \textbf{salt} and \textbf{pepper} as needed.
    
    \item Remove from heat and stir in \textbf{lime juice} and half of the chopped \textbf{cilantro}.
    
    \item Serve hot, garnished with remaining \textbf{cilantro} and \textbf{lime wedges} on the side.
\end{enumerate}

\newpage

% Begin compact two-column layout
{\small
\setlength{\columnsep}{20pt}
\setlength{\multicolsep}{6pt}
\begin{multicols}{2}
\setlength{\parindent}{0pt}
\setlength{\parskip}{4pt}

\subsection*{Equipment Required}
\begin{itemize}
    \item Large Dutch oven or heavy-bottomed pot (6-quart capacity)
    \item Sharp chef's knife
    \item Large cutting board
    \item Measuring cups and spoons
    \item Wooden spoon or silicone spatula
    \item Can opener
    \item Colander for draining beans
    \item Ladle for serving
    \item Timer
\end{itemize}

\subsection*{Mise en Place}
\begin{itemize}
    \item Bring \textbf{carnitas consommé} to room temperature if refrigerated
    \item Warm \textbf{carnitas meat} slightly if cold from refrigeration
    \item Dice all vegetables to uniform ¼-inch pieces for even cooking
    \item Have all spices measured and ready before starting
    \item Open and drain \textbf{black bean} cans just before use
\end{itemize}

\subsection*{Ingredient Tips}
\begin{itemize}
    \item Use \textbf{Mexican oregano} rather than Mediterranean for authentic flavor profile
    \item \textbf{Rotel} provides perfect heat balance - avoid substituting with plain diced tomatoes
    \item Rinse \textbf{black beans} thoroughly to remove excess sodium and starch
    \item If \textbf{carnitas consommé} is very gelatinous, it will thin perfectly when heated
    \item Fresh \textbf{lime juice} is essential - bottled juice lacks the brightness needed
\end{itemize}

\subsection*{Preparation Tips}
\begin{itemize}
    \item Don't rush the vegetable sautéing - proper softening builds flavor foundation
    \item Bloom spices in oil for maximum potency before adding liquid
    \item Add \textbf{lime juice} and \textbf{cilantro} off heat to preserve bright flavors
    \item Taste soup before final seasoning - \textbf{consommé} saltiness varies
    \item If soup seems too thick, add water; if too thin, simmer uncovered longer
\end{itemize}

\subsection*{Make Ahead \& Storage}
\begin{itemize}
    \item Soup improves in flavor after \textit{24~hours} in refrigerator
    \item Store up to \textit{4~days} refrigerated or \textit{3~months} frozen
    \item Add \textbf{lime juice} and fresh \textbf{cilantro} only when reheating to serve
    \item May need additional water when reheating as beans absorb liquid
    \item Freeze in individual portions for easy weeknight meals
\end{itemize}

\subsection*{Serving Suggestions}
\begin{itemize}
    \item Garnish with diced white onion, crumbled \textbf{queso fresco}, or \textbf{Mexican crema}
    \item Serve with warm corn tortillas or crusty bread
    \item Add diced \textbf{avocado} just before serving for richness
    \item Accompany with pickled jalapeños for those wanting extra heat
    \item Makes excellent leftover lunch - flavors continue to develop
    \item Consider serving with Mexican rice as a more substantial meal
\end{itemize}

\end{multicols}
}

\end{document}