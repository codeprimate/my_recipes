\documentclass[11pt,letterpaper]{article}
\usepackage{fontspec}
\usepackage{tocloft}
\usepackage{multicol}
\usepackage[left=1.4in,right=1.5in,top=1in,bottom=1in]{geometry}
\setmainfont[Scale=1.2,AutoFakeBold=1.5,AutoFakeSlant=0.3]{Doves Type}

\title{13-Bean Soup with Braised Pork Ribs}
\author{}
\date{}

\begin{document}

\maketitle
\thispagestyle{empty}

\section*{Ingredients}
\setlength{\columnsep}{20pt}
\begin{multicols}{2}
\noindent
    13-bean mix, dried \dotfill 1 lb. \\
    Braised pork ribs, picked \dotfill 2 lbs. \\
    Pork rib bones \dotfill reserved \\
    Braising liquid, strained \dotfill 3 cups \\
    Chicken stock or water \dotfill 6-8 cups \\
    Bacon fat \dotfill 3 Tbsp. \\
    Onion, medium \dotfill 1 \\
    Carrots, medium \dotfill 2 \\
    Celery stalks \dotfill 3 \\
    Leeks, medium \dotfill 2 \\
    \columnbreak
    Parsnips, medium \dotfill 2 \\
    Garlic cloves \dotfill 6 \\
    Tomato paste \dotfill 3 Tbsp. \\
    San Marzano tomatoes, whole \dotfill 28 oz. can \\
    Dry red wine \dotfill ½ cup \\
    Bay leaves \dotfill 2 \\
    Fresh thyme sprigs \dotfill 4-5 \\
    White pepper \dotfill 1 tsp. \\
    Kosher salt \dotfill 1-2 tsp. \\
    Black pepper \dotfill to taste \\
    Fresh spinach \dotfill 6 oz. \\
    Dried parsley \dotfill 2 Tbsp. \\
\end{multicols}

\section*{Directions}

\noindent
Soak \textbf{beans} overnight in cold water ---
Strain \textbf{braising~liquid}, reserve \textbf{bones} ---
Pick 2~lbs. \textbf{pork} from \textbf{ribs} ---
Dice \textbf{onion}, \textbf{carrots}, \textbf{celery}, and \textbf{parsnips} to ½-inch ---
Clean and slice \textbf{leeks} into ½-inch rounds ---
Mince \textbf{garlic} ---
Hand-crush \textbf{San~Marzano~tomatoes} in bowl

\begin{enumerate}
    \item The night before, rinse \textbf{beans} thoroughly and place in large bowl. Cover with cold water by \textit{3~inches}. Soak for \textit{8-12~hours}. Drain before using.
    
    \item Strain \textbf{braising~liquid}, removing spent aromatics and \textbf{peppercorns}. Reserve \textbf{rib~bones} with any remaining meat attached. Pick \textit{2~lbs.} of \textbf{meat} from \textbf{ribs}, keeping pieces rustic and chunky. Set aside \textbf{meat} and \textbf{bones} separately.
    
    \item In large Dutch oven or heavy pot (\textit{8+~quart}), heat \textbf{bacon~fat} over medium heat. Add diced \textbf{onion}, \textbf{carrots}, \textbf{celery}, \textbf{leeks}, and \textbf{parsnips}. Sauté for \textit{10-12~minutes} until vegetables are softened and beginning to caramelize.
    
    \item Add minced \textbf{garlic} and cook for \textit{1-2~minutes} until fragrant. Add \textbf{tomato~paste} and cook, stirring constantly, for \textit{2-3~minutes} until paste darkens and becomes fragrant.
    
    \item Deglaze with \textbf{red~wine}, scraping up any browned bits from bottom of pot. Simmer until \textbf{wine} reduces by half, about \textit{3-4~minutes}.
    
    \item Add drained \textbf{beans}, hand-crushed \textbf{tomatoes} with juices, \textbf{braising~liquid}, reserved \textbf{rib~bones}, \textbf{bay~leaves}, \textbf{thyme~sprigs}, and \textbf{white~pepper}. Add \textit{6~cups} \textbf{chicken~stock} or water.
    
    \item Bring to boil, then reduce to gentle simmer. Cover and cook for \textit{2-2½~hours}, stirring occasionally, until \textbf{beans} are completely tender. Add additional \textbf{stock} or water as needed to maintain thick soup consistency.
    
    \item Remove \textbf{rib~bones} (they should be clean or nearly so). Add picked \textbf{pork} and \textbf{dried~parsley}. Simmer for \textit{10~minutes} to heat through.
    
    \item Taste and adjust seasoning with \textbf{kosher~salt} and \textbf{black~pepper}. The \textbf{braising~liquid} is already well-seasoned, so add \textbf{salt} conservatively.
    
    \item Stir in \textbf{fresh~spinach} and cook until just wilted, about \textit{2-3~minutes}.
    
    \item Remove \textbf{bay~leaves} and \textbf{thyme~sprigs}. Let soup rest for \textit{10-15~minutes} before serving—it will thicken as it sits.
\end{enumerate}

\newpage

{\small
\setlength{\columnsep}{20pt}
\setlength{\multicolsep}{6pt}
\begin{multicols}{2}
\setlength{\parindent}{0pt}
\setlength{\parskip}{4pt}

\subsection*{Equipment Required}
\begin{itemize}
    \item Dutch oven or heavy pot (8+ quart capacity)
    \item Large bowl (for soaking beans)
    \item Fine mesh strainer or colander
    \item Medium bowl (for crushing tomatoes)
    \item Sharp knife and cutting board
    \item Measuring cups and spoons
    \item Wooden spoon or sturdy spatula
    \item Ladle
    \item Storage containers (for leftovers)
\end{itemize}

\subsection*{Mise en Place}
\begin{itemize}
    \item Soak \textbf{beans} the night before—this is essential for even cooking and proper texture
    \item Strain \textbf{braising~liquid} ahead of time, removing all aromatics and \textbf{peppercorns}
    \item Pick all \textbf{pork} from \textbf{bones} before starting; reserve \textbf{bones} separately
    \item Prepare all vegetables before heating pot—this ensures smooth workflow
    \item Have \textbf{stock} or water measured and ready; you'll add it gradually
    \item Hand-crush \textbf{tomatoes} in bowl before starting to cook
\end{itemize}

\subsection*{Ingredient Tips}
\begin{itemize}
    \item \textbf{Beans}: Overnight soaking is crucial for tender, evenly cooked beans; quick-soak methods don't work as well for this recipe
    \item \textbf{Braising~liquid}: Already contains significant \textbf{salt}, soy sauce, and Worcestershire—taste before adding more \textbf{salt}
    \item \textbf{San~Marzano~tomatoes}: Use certified DOP if possible; hand-crushing gives better texture than pre-crushed varieties
    \item \textbf{Bacon~fat}: Rendered bacon fat adds tremendous flavor; substitute with olive oil or butter if needed
    \item \textbf{Leeks}: Clean thoroughly—slice lengthwise and rinse between layers to remove sand
    \item \textbf{Stock}: Homemade or low-sodium \textbf{chicken~stock} preferred; water works well given the rich \textbf{braising~liquid}
    \item \textbf{Wine}: Use a dry red you'd drink; avoid "cooking wine" which contains added \textbf{salt}
\end{itemize}

\subsection*{Preparation Tips}
\begin{itemize}
    \item Take time with vegetable sauté—caramelization builds foundational sweetness and depth
    \item Bloom \textbf{tomato~paste} until it darkens and smells sweet; this removes raw taste and concentrates flavor
    \item Don't rush the \textbf{wine} reduction—you want to cook off harsh alcohol while preserving acidity
    \item Keep \textbf{rib~bones} in during bean cooking—they continue releasing gelatin for body
    \item Stir occasionally during long simmer to prevent sticking, but don't over-stir or \textbf{beans} will break down
    \item Add liquid gradually—different bean mixes and \textbf{braising~liquid} concentrations affect absorption
    \item Test \textbf{beans} for doneness by tasting several; they should be completely tender with no chalky center
    \item Add \textbf{pork} near end to prevent it from becoming tough through extended simmering
    \item \textbf{White~pepper} blooms throughout cooking for integrated heat; \textbf{black~pepper} at end for fresh bite
    \item Add \textbf{spinach} last—it wilts quickly and loses color if overcooked
    \item Let soup rest before serving—flavors meld and consistency thickens naturally
\end{itemize}

\subsection*{Make Ahead \& Storage}
\begin{itemize}
    \item This soup improves significantly overnight as flavors marry; make \textit{1-2~days} ahead if possible
    \item Store in refrigerator for up to \textit{4~days}; soup will thicken considerably when cold
    \item To freeze: cool completely, portion into containers, freeze up to \textit{3~months}
    \item Don't add \textbf{spinach} if freezing—add fresh when reheating
    \item Reheat gently on stovetop, adding water or \textbf{stock} to achieve desired consistency
    \item Soup thickens as it sits; thin with water or \textbf{stock} when reheating
    \item Taste and re-season after reheating—flavors can mellow
\end{itemize}

\subsection*{Serving Suggestions}
\begin{itemize}
    \item Serve with crusty artisan bread, cornbread, or buttermilk biscuits for sopping
    \item Offer hot sauce or red pepper flakes at table for those who want heat
    \item Drizzle with high-quality extra virgin olive oil just before serving
    \item Top with grated Parmesan, Pecorino Romano, or aged cheddar
    \item Garnish with fresh parsley, thyme leaves, or chopped \textbf{green~onions}
    \item Pairs beautifully with the same dry red \textbf{wine} used in cooking
    \item Serve with simple green salad dressed with vinaigrette to cut richness
    \item Makes excellent next-day lunch—flavors continue developing
\end{itemize}

\end{multicols}
}

\end{document}
