\documentclass[11pt,letterpaper]{article}
\usepackage{fontspec}
\usepackage{tocloft}
\usepackage{multicol}
\usepackage[left=1.4in,right=1.5in,top=1in,bottom=1in]{geometry}
\usepackage{textcomp}
\usepackage{nicefrac}
\setmainfont[Scale=1.3,AutoFakeBold=1.5,AutoFakeSlant=0.3]{Doves Type}

\title{Berbere Lamb and Chickpea Stew •}
\author{}
\date{}

\begin{document}

\maketitle
\thispagestyle{empty}

\section*{Ingredients}
\setlength{\columnsep}{20pt}
\begin{multicols}{2}
\noindent
    Dried chickpeas \dotfill 2 cups \\
    Kosher salt \dotfill 5 Tbsp. \\
    Lamb shoulder/shank* \dotfill 3 lbs. \\
    Black pepper (for lamb) \dotfill 1 tsp. \\
    Vegetable oil \dotfill \nicefrac{1}{4} cup \\
    Onions, large \dotfill 2 \\
    Garlic cloves \dotfill 8-10 \\
    Berbere spice blend \dotfill 5 Tbsp. \\
	\columnbreak
    Tomato paste \dotfill 6 Tbsp. \\
    Chicken broth or lamb stock \dotfill 3 cups \\
    Diced tomatoes \dotfill 28 oz. (2 cans) \\
    Dried apricots \dotfill 1\nicefrac{1}{2} cups \\
    Fresh cilantro \dotfill 1 cup \\
    Fresh lemon juice \dotfill \nicefrac{1}{2} cup \\
    Honey \dotfill 3 Tbsp. \\
    Pomegranate arils (optional) \dotfill \nicefrac{1}{2} cup \\
    Tzatziki (optional) \dotfill for serving \\
    Harissa (optional) \dotfill for table \\
\end{multicols}

{\small\textit{*Beef chuck shoulder may substitute for lamb (same method and timing)*}}

\section*{Directions}

\noindent
\textbf{Night Before:} Combine 2~cups \textbf{chickpeas}, water, and 3~Tbsp. \textbf{salt} in a large bowl. Cover and refrigerate overnight (\textit{8-12~hours}). \\
\textbf{Day of Preparation:}
Preheat oven to \textit{300°F} ---
Drain and rinse soaked 2~cups \textbf{chickpeas} ---
Pat dry \textbf{lamb} and cut into 1\nicefrac{1}{2}" cubes ---
Season \textbf{lamb} with 1~Tbsp. \textbf{salt} and 1~tsp. \textbf{pepper} ---
Dice 2~large \textbf{onions} and mince 8-10 \textbf{garlic} cloves; combine in \textit{Small Bowl~\#1} (aromatics) ---
Chop 1\nicefrac{1}{2}~cups \textbf{apricots} into \nicefrac{1}{4}" pieces; set aside in \textit{Medium Bowl~\#1} ---
Chop 1~cup \textbf{cilantro} ---
Combine chopped \textbf{cilantro}, \nicefrac{1}{2}~cup \textbf{lemon~juice}, and 3~Tbsp. \textbf{honey} in \textit{Small Bowl~\#2} (finishing)

\begin{enumerate}
    \item Heat 6-quart enameled dutch oven over medium-high heat. Add 2~Tbsp. \textbf{oil}. Working in 3-4~batches to avoid crowding, brown \textbf{lamb~cubes} on multiple sides until deeply caramelized, about \textit{8-10~minutes} per batch, adding more \textbf{oil} as needed (up to \nicefrac{1}{4}~cup total). Transfer browned \textbf{lamb} to \textit{Large Bowl~\#1} and set aside.
    
    \item Reduce heat to medium. If pot is dry, add final portion of \textbf{oil} (remaining from \nicefrac{1}{4}~cup). Add \textbf{onions} and \textbf{garlic} (\textit{Small Bowl~\#1}) and cook, stirring occasionally, until softened and golden brown, about \textit{8-10~minutes}.
    
    \item Add 3~Tbsp. \textbf{berbere} to the pot. Stir constantly for \textit{45-60~seconds} until spice is darkened and very fragrant. Add 6~Tbsp. \textbf{tomato~paste} and stir constantly, scraping to prevent scorching, for \textit{2-3~minutes} until paste is brick-red and caramelized.
    
    \item Add remaining 2~Tbsp. \textbf{berbere} and 1~Tbsp. \textbf{salt}; stir to combine. Immediately add 3~cups \textbf{chicken~broth} and use a wooden spoon to scrape bottom of pot vigorously, releasing all browned bits. Add 28~oz. (2~cans) \textbf{diced~tomatoes} with their juices. Bring to a simmer.
    
    \item Return browned \textbf{lamb} and any accumulated juices (\textit{Large Bowl~\#1}) to pot. Add drained \textbf{chickpeas} and 1~cup chopped \textbf{apricots} (\textit{Medium Bowl~\#1}). Stir to combine. The liquid should come about \nicefrac{3}{4} of the way up the solids; add additional \textbf{broth} if needed.
    
    \item Bring to a full simmer on stovetop. Cover with tight-fitting lid and transfer to preheated \textit{300°F} oven.
    
    \item Braise for \textit{4~hours}, checking at \textit{3\nicefrac{1}{2}~hours}. \textbf{Lamb} should be pull-apart tender and 2~cups \textbf{chickpeas} should be creamy. If liquid level seems low at the \textit{3\nicefrac{1}{2}~hour} check, add \nicefrac{1}{2}-1~cup hot \textbf{broth}. If stew seems too thin, crack lid slightly for final \textit{30~minutes}.
    
    \item Remove from oven. If sauce needs reducing, place uncovered pot on stovetop over medium heat and simmer for \textit{5-10~minutes} until thickened to coat the back of a spoon. If desired, skim excess fat from surface.
    
    \item Stir in remaining \nicefrac{1}{2}~cup chopped \textbf{apricots} (\textit{Medium Bowl~\#1}) and \textbf{cilantro}, \textbf{lemon~juice}, and \textbf{honey} (\textit{Small Bowl~\#2}). Taste and adjust seasoning with additional \textbf{salt} if needed. Let rest for \textit{10-15~minutes} before serving.
    
    \item Serve over \textbf{couscous} or \textbf{cooked~rice}, topped with \textbf{pomegranate~arils} and \textbf{tzatziki}, with \textbf{harissa} on the side.
\end{enumerate}

\newpage

% Begin compact two-column layout
{\small
\setlength{\columnsep}{20pt}
\setlength{\multicolsep}{6pt}
\begin{multicols}{2}
\setlength{\parindent}{0pt}
\setlength{\parskip}{4pt}

\subsection*{Equipment Required}
\begin{itemize}
    \item 6-quart enameled dutch oven with tight-fitting lid
    \item Large mixing bowl (for soaking chickpeas)
    \item Small prep bowls (2)
    \item Medium prep bowl (1)
    \item Large prep bowl (1)
    \item Cutting board and sharp knife
    \item Wooden spoon or heatproof spatula
    \item Measuring cups and spoons
    \item Ladle
    \item Timer
\end{itemize}

\subsection*{Mise en Place}
\begin{itemize}
    \item Small Bowl \#1 — aromatics: diced 2~large \textbf{onions}, minced 8-10 \textbf{garlic} cloves
    \item Medium Bowl \#1 — chopped \textbf{apricots} (1\nicefrac{1}{2}~cups total, used in two stages)
    \item Small Bowl \#2 — finishing: 1~cup chopped \textbf{cilantro}, \nicefrac{1}{2}~cup \textbf{lemon~juice}, 3~Tbsp. \textbf{honey}
    \item Large Bowl \#1 — browned \textbf{lamb} (set aside after browning, about 3~lbs., with accumulated juices)
    \item Soak 2~cups \textbf{chickpeas} the night before in salted water (\textit{8-12~hours})
    \item Drain and rinse \textbf{chickpeas} before beginning active cooking
    \item Pat dry and cut \textbf{lamb} into uniform 1\nicefrac{1}{2}" cubes; season with \textbf{salt} and \textbf{pepper} before browning
    \item If using whole lamb shoulder, trim excess fat but leave some for flavor
    \item Prep all aromatics and measure all spices before beginning—once you start browning, the process moves quickly
    \item Have \textbf{broth} ready and warm for easier deglazing
    \item Allow \textit{45-60~minutes} total for prep work on day of cooking
\end{itemize}

\subsection*{Ingredient Tips}
\begin{itemize}
    \item Lamb shoulder is ideal for braising due to marbling and connective tissue; leg meat is leaner and won't be as tender
    \item Quality \textbf{berbere} blends vary significantly in heat level and complexity; taste yours to gauge intensity before adding
    \item San Marzano or fire-roasted \textbf{diced~tomatoes} add extra depth
    \item Turkish or California \textbf{apricots} are ideal; avoid overly sweet or sugared varieties
    \item Homemade \textbf{lamb~stock} elevates the dish significantly
\end{itemize}

\subsection*{Preparation Tips}
\begin{itemize}
    \item Don't rush the browning—deep caramelization is the foundation of flavor
    \item Work in small batches; crowding the pot steams meat instead of browning it
    \item The \textbf{tomato~paste} will threaten to scorch; keep stirring and scraping constantly during step 3
    \item Vigorous deglazing in step 4 is critical—every bit of fond adds flavor
    \item Starting the braise at a full simmer on the stovetop ensures immediate cooking when transferred to oven
    \item If your dutch oven lid doesn't seal tightly, cover pot with foil before adding lid to minimize evaporation
    \item Check liquid level at \textit{3\nicefrac{1}{2}~hours}; ovens and pots vary, so adjustment may be needed
    \item \textbf{Lamb} texture varies by cut quality; check tenderness and extend cooking if needed
    \item The stew will continue to thicken as it rests; it should be slightly looser than desired final consistency when removed from oven
\end{itemize}

\subsection*{Make Ahead \& Storage}
\begin{itemize}
    \item This stew benefits from sitting; make up to \textit{3~days} ahead and refrigerate
    \item Add finishing ingredients (\textbf{cilantro}, \textbf{lemon~juice}, \textbf{honey}) only when reheating to serve
    \item If freezing, slightly undercook (reduce time by \textit{30~minutes}) as reheating continues cooking
    \item Leftover stew thickens significantly; thin with \textbf{broth} or water when reheating
\end{itemize}

\end{multicols}
}

\end{document}
