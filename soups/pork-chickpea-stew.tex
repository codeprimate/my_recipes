\documentclass[11pt,letterpaper]{article}
\usepackage{fontspec}
\usepackage{tocloft}
\usepackage{multicol}
\usepackage[left=1.4in,right=1.5in,top=1in,bottom=1in]{geometry}
\setmainfont[Scale=1.2,AutoFakeBold=1.5,AutoFakeSlant=0.3]{Doves Type}

\title{Pork and Chickpea Stew with North African Spices}
\author{}
\date{}

\begin{document}

\maketitle
\thispagestyle{empty}

\section*{Ingredients}
\setlength{\columnsep}{20pt}
\begin{multicols}{2}
\noindent
    Dried chickpeas \dotfill 2 cups \\
    Kosher salt (for soaking) \dotfill 3 Tbsp. \\
    Pork shoulder, 1½" cubes \dotfill 3 lbs. \\
    Kosher salt (for pork) \dotfill 1 Tbsp. \\
    Black pepper (for pork) \dotfill 1 tsp. \\
    Vegetable oil \dotfill ¼ cup \\
    Onions, large \dotfill 2 \\
    Garlic cloves \dotfill 8-10 \\
    Fresh ginger \dotfill 2 Tbsp. \\
    Ground cumin \dotfill 5 tsp. \\
    Ground coriander \dotfill 5 tsp. \\
    Sweet paprika \dotfill 3 tsp. \\
    Smoked paprika \dotfill 1 Tbsp. \\
    \columnbreak
    Ground turmeric \dotfill 2 tsp. \\
    Ground cinnamon \dotfill 1 tsp. \\
    Fennel seeds \dotfill 1 tsp. \\
    Tomato paste \dotfill 6 Tbsp. \\
    Ras el hanout \dotfill 5 tsp. \\
    Kosher salt (for stew) \dotfill 1 Tbsp. \\
    Chicken broth \dotfill 3 cups \\
    Diced tomatoes \dotfill 28 oz. (2 cans) \\
    Dried apricots \dotfill 1½ cups \\
    Fresh cilantro \dotfill 1 cup \\
    Fresh lemon juice \dotfill ½ cup \\
    Honey \dotfill 2 Tbsp. \\
    Cooked rice \dotfill for serving \\
    Harissa \dotfill for table \\
\end{multicols}

\section*{Directions}

\noindent
\textbf{Night Before:} Combine \textbf{chickpeas}, water, and 3~Tbsp. \textbf{salt} in a large bowl. Cover and refrigerate overnight (\textit{8-12~hours}). ---
Preheat oven to \textit{300°F} ---
Drain and rinse soaked \textbf{chickpeas} ---
Pat dry \textbf{pork} and cut into 1½" cubes ---
Trim excess fat from \textbf{pork}, leaving some for flavor ---
Season \textbf{pork} with 1~Tbsp. \textbf{salt} and 1~tsp. \textbf{pepper} ---
Dice \textbf{onions} ---
Mince \textbf{garlic} ---
Grate \textbf{ginger} ---
Chop \textbf{apricots} into ¼" pieces ---
Measure all \textbf{spices} ---
Chop \textbf{cilantro}

\begin{enumerate}
    \item Heat 6-quart enameled dutch oven over medium-high heat. Add 2~Tbsp. \textbf{oil}. Working in 3-4~batches to avoid crowding, brown \textbf{pork~cubes} on multiple sides until deeply caramelized, about \textit{8-10~minutes} per batch, adding more \textbf{oil} as needed. Transfer browned \textbf{pork} to a large bowl and set aside.
    
    \item Carefully pour off all but 2~Tbsp. of rendered fat from pot (reserve excess for another use). Add diced \textbf{onions} and cook over medium heat, stirring occasionally, until softened and golden brown, about \textit{8-10~minutes}. Add minced \textbf{garlic} and grated \textbf{ginger}; cook, stirring constantly, for \textit{1-2~minutes} until fragrant.
    
    \item Add \textbf{cumin}, \textbf{coriander}, \textbf{sweet~paprika}, \textbf{smoked~paprika}, \textbf{turmeric}, \textbf{cinnamon}, and \textbf{fennel~seeds} to the pot. Stir constantly for \textit{45-60~seconds} until spices are darkened and very fragrant. Add \textbf{tomato~paste} and stir constantly, scraping to prevent scorching, for \textit{2-3~minutes} until paste is brick-red and caramelized.
    
    \item Add \textbf{ras~el~hanout} and 1~Tbsp. \textbf{salt}; stir to combine. Immediately add \textbf{chicken~broth} and use a wooden spoon to scrape bottom of pot vigorously, releasing all browned bits. Add \textbf{diced~tomatoes} with their juices. Bring to a simmer.
    
    \item Return browned \textbf{pork} and any accumulated juices to pot. Add drained \textbf{chickpeas} and 1~cup chopped \textbf{apricots}. Stir to combine. The liquid should come about ¾ of the way up the solids; add additional \textbf{broth} if needed.
    
    \item Bring to a full simmer on stovetop. Cover with tight-fitting lid and transfer to preheated \textit{300°F} oven.
    
    \item Braise for \textit{2½-3~hours}, checking at \textit{2~hours}. \textbf{Pork} should be pull-apart tender and \textbf{chickpeas} should be creamy. If liquid level seems low at the \textit{2~hour} check, add ½-1~cup hot \textbf{broth}. If stew seems too liquidy, crack lid slightly for final \textit{30-45~minutes}.
    
    \item Remove from oven. Skim excess fat from surface—\textbf{pork} renders more fat than lamb, and removing excess prevents greasiness. If sauce needs reducing, place uncovered pot on stovetop over medium heat and simmer for \textit{5-10~minutes} until thickened to coat the back of a spoon.
    
    \item Stir in remaining ½~cup chopped \textbf{apricots}, \textbf{fresh~cilantro}, \textbf{lemon~juice}, and \textbf{honey}. Taste and adjust seasoning with additional \textbf{salt} if needed. Let rest for \textit{10-15~minutes} before serving.
    
    \item Serve over \textbf{cooked~rice} with \textbf{harissa} on the side.
\end{enumerate}

\newpage

% Begin compact two-column layout
{\small
\setlength{\columnsep}{20pt}
\setlength{\multicolsep}{6pt}
\begin{multicols}{2}
\setlength{\parindent}{0pt}
\setlength{\parskip}{4pt}

\subsection*{Equipment Required}
\begin{itemize}
    \item 6-quart enameled dutch oven with tight-fitting lid
    \item Large mixing bowl (for soaking chickpeas)
    \item Large bowl or plate (for browned pork)
    \item Cutting board and sharp knife
    \item Wooden spoon or heatproof spatula
    \item Measuring cups and spoons
    \item Microplane or fine grater (for ginger)
    \item Ladle
    \item Fat separator or large spoon (for skimming)
    \item Timer
\end{itemize}

\subsection*{Mise en Place}
\begin{itemize}
    \item Soak \textbf{chickpeas} the night before in salted water
    \item Allow \textit{45-60~minutes} total for prep work on day of cooking
    \item Trim excess fat from \textbf{pork~shoulder} but leave some marbling—it adds flavor and moisture
    \item Cut \textbf{pork} into uniform 1½" cubes for even cooking
    \item Prep all aromatics and measure all spices before beginning—once you start browning, the process moves quickly
    \item Have \textbf{broth} ready and warm for easier deglazing
    \item Keep a heatproof container nearby for collecting rendered pork fat
\end{itemize}

\subsection*{Ingredient Tips}
\begin{itemize}
    \item Pork shoulder (also called pork butt or Boston butt) is ideal; avoid loin, which is too lean for braising
    \item Look for well-marbled \textbf{pork} with visible fat running through the meat
    \item Pereg or other quality \textbf{ras~el~hanout} blends work well; avoid dusty, stale spices
    \item Use whole \textbf{fennel~seeds} rather than ground for better texture and flavor; they add aromatic sweetness that complements pork
    \item San Marzano or fire-roasted \textbf{diced~tomatoes} add extra depth
    \item Turkish or California \textbf{apricots} are ideal; avoid overly sweet or sugared varieties
    \item Fresh \textbf{ginger} is essential; powdered won't provide the same brightness
    \item Reduced \textbf{honey} (compared to lamb version) accounts for pork's natural sweetness
    \item Increased \textbf{smoked~paprika} balances pork's richness and adds depth
\end{itemize}

\subsection*{Preparation Tips}
\begin{itemize}
    \item Don't rush the browning—deep caramelization is the foundation of flavor
    \item Work in small batches; crowding the pot steams meat instead of browning it
    \item \textbf{Pork} renders significantly more fat than lamb; pour off excess after browning to prevent greasy stew
    \item The \textbf{tomato~paste} will threaten to scorch; keep stirring and scraping constantly during step 3
    \item \textbf{Fennel~seeds} add subtle licorice notes that pair beautifully with pork and apricots
    \item Vigorous deglazing in step 4 is critical—every bit of fond adds flavor
    \item Starting the braise at a full simmer on the stovetop ensures immediate cooking when transferred to oven
    \item If your dutch oven lid doesn't seal tightly, cover pot with foil before adding lid to minimize evaporation
    \item Check tenderness at \textit{2~hours}; \textbf{pork} often reaches pull-apart texture faster than lamb
    \item Skimming fat in step 8 is more important with pork than lamb—be thorough
    \item The stew will continue to thicken as it rests; it should be slightly looser than desired final consistency when removed from oven
\end{itemize}

\subsection*{Make Ahead \& Storage}
\begin{itemize}
    \item This stew improves with time; make up to \textit{3~days} ahead and refrigerate
    \item Pork fat will solidify into a thick layer when chilled, making it very easy to remove
    \item Remove solidified fat layer before reheating for best results
    \item Reheat gently on stovetop, adding \textbf{broth} if needed to restore consistency
    \item Add finishing ingredients (\textbf{cilantro}, \textbf{lemon~juice}, \textbf{honey}) only when reheating to serve
    \item Freezes well for up to \textit{3~months}; thaw overnight in refrigerator
    \item If freezing, slightly undercook (reduce time by \textit{30~minutes}) as reheating continues cooking
    \item Leftover stew thickens significantly; thin with \textbf{broth} or water when reheating
    \item The rendered pork fat can be saved and used for sautéing vegetables or making cornbread
\end{itemize}

\subsection*{Serving Suggestions}
\begin{itemize}
    \item Serve over basmati rice, couscous, polenta, or with crusty bread
    \item Accompany with \textbf{harissa} for heat, plain yogurt or sour cream for cooling contrast
    \item Garnish with additional \textbf{fresh~cilantro}, toasted sliced almonds, or toasted pine nuts
    \item A simple arugula salad with lemon vinaigrette provides refreshing contrast
    \item Pickled vegetables (turnips, carrots, or red onions) cut through the richness
    \item Pairs beautifully with medium-bodied red wines, amber ales, or spiced cider
    \item Consider topping with a poached or fried egg for brunch service
    \item Leftovers make excellent filling for tacos, empanadas, or savory hand pies
    \item Also delicious served over creamy polenta or mashed potatoes
\end{itemize}

\subsection*{Variation Notes}
\begin{itemize}
    \item This is a North African-\textit{inspired} recipe adapted for pork; traditional North African cuisine does not use pork
    \item For a more traditional approach, substitute lamb or beef shoulder using the same technique
    \item For extra richness, add ½~cup golden raisins along with the \textbf{apricots}
    \item Orange zest (1~Tbsp.) added with finishing ingredients complements pork beautifully
    \item For a spicier version, add 1-2~tsp. cayenne pepper or Aleppo pepper with other spices
    \item Substitute bone-in pork shoulder for even more flavor; increase cooking time to \textit{3-3½~hours}
\end{itemize}

\end{multicols}
}

\end{document}
