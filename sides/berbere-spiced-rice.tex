\documentclass[11pt,letterpaper]{article}
\usepackage{fontspec}
\usepackage{tocloft}
\usepackage{multicol}
\usepackage[left=1.4in,right=1.5in,top=1in,bottom=1in]{geometry}
\setmainfont[Scale=1.4,AutoFakeBold=1.5,AutoFakeSlant=0.3]{Doves Type}

\title{Berbere-Spiced Rice with Vegetables}
\author{}
\date{}

\begin{document}

\maketitle
\thispagestyle{empty}

\section*{Ingredients}
\setlength{\columnsep}{20pt}
\noindent
    Long grain white rice \dotfill 3 cups \\
    Hot water \dotfill 3 cups \\
    Better Than Bouillon - Chicken \dotfill 1 Tbsp. \\
    Bay leaf \dotfill 1 \\
    Berbere spice mix \dotfill 2 tsp. \\
    Dehydrated soup vegetables \dotfill 2 Tbsp. \\
    Butter \dotfill \textonehalf{} Tbsp.

\section*{Directions}

\noindent
Wash \textbf{rice} thoroughly ---
Prepare \textbf{stock} with \textbf{Better Than Bouillon}

\begin{enumerate}
    \item Wash \textbf{rice} thoroughly: Place \textbf{rice} in a fine-mesh strainer or bowl. Rinse under cold running water, gently agitating with your fingers, until the water runs clear (typically \textit{3-4~rinses}). Drain well.
    
    \item Combine \textbf{hot water} and \textbf{Better Than Bouillon} in a measuring cup, stirring to dissolve completely.
    
    \item Transfer washed \textbf{rice} to rice maker pot. Add prepared \textbf{stock}, \textbf{bay leaf}, \textbf{Berbere spice mix}, and \textbf{dehydrated soup vegetables}.
    
    \item Cook according to rice maker instructions (long grain white rice setting).
    
    \item Once cooking is complete, let rice rest for \textit{5~minutes} before opening.
    
    \item Open rice maker, remove \textbf{bay leaf}, and gently fluff rice with a fork or rice paddle.
    
    \item Add \textbf{butter} and gently fold into rice until evenly distributed. Serve immediately.
\end{enumerate}

\newpage

% Begin compact two-column layout
{\small
\setlength{\columnsep}{20pt}
\setlength{\multicolsep}{6pt}
\begin{multicols}{2}
\setlength{\parindent}{0pt}
\setlength{\parskip}{4pt}

\subsection*{Equipment Required}
\begin{itemize}
    \item Rice maker
    \item Fine-mesh strainer or large bowl
    \item Measuring cups and spoons
    \item Large measuring cup or bowl (for stock preparation)
    \item Rice paddle or fork (for fluffing)
\end{itemize}

\subsection*{Mise en Place}
\begin{itemize}
    \item Measure \textbf{rice}
    \item Measure \textbf{hot water} and \textbf{Better Than Bouillon}
    \item Have \textbf{bay leaf} ready
    \item Measure \textbf{Berbere spice mix}
    \item Measure \textbf{dehydrated soup vegetables}
    \item Have \textbf{butter} at room temperature for easier incorporation
\end{itemize}

\subsection*{Ingredient Tips}
\begin{itemize}
    \item Long-grain white rice works best for this method
    \item Thorough washing removes excess starch and prevents gummy texture
    \item \textbf{Better Than Bouillon} adds depth without overwhelming the rice
    \item Quality \textbf{butter} makes a noticeable difference in final flavor
    \item \textbf{Bay leaf} should be removed before serving
    \item \textbf{Berbere spice mix} provides warm, aromatic Ethiopian-inspired flavors
    \item \textbf{Dehydrated soup vegetables} rehydrate during cooking and add texture
\end{itemize}

\subsection*{Preparation Tips}
\begin{itemize}
    \item Ensure \textbf{Better Than Bouillon} is fully dissolved before adding to rice maker
    \item Don't skip the washing step - it's essential for proper texture
    \item Let rice rest after cooking to allow steam to finish the process
    \item Fluff rice gently to avoid breaking grains
    \item Add \textbf{butter} while rice is still hot for best incorporation
    \item The \textbf{Berbere spice mix} will infuse the rice during cooking
    \item \textbf{Dehydrated soup vegetables} will soften and rehydrate during the cooking process
    \item If rice seems too wet, let it sit with lid open for a few minutes
    \item If rice seems too dry, add a tablespoon of hot water and fluff again
\end{itemize}

\subsection*{Make Ahead \& Storage}
\begin{itemize}
    \item Cooked \textbf{rice} keeps refrigerated for \textit{3-4~days}
    \item Reheat gently with a splash of water or broth to restore moisture
    \item Freeze cooked \textbf{rice} for up to \textit{2~months}
    \item Cool to room temperature before refrigerating
    \item Store in airtight container
    \item Label container with date
\end{itemize}

\subsection*{Serving Suggestions}
\begin{itemize}
    \item Serve as side dish with grilled meats or roasted vegetables
    \item Excellent accompaniment to Ethiopian-inspired dishes
    \item Pairs well with braised meats and stews
    \item Use as base for grain bowls with additional vegetables
    \item Perfect with saucy dishes that benefit from spiced rice
    \item Use leftover rice for fried rice (best with day-old rice)
    \item Add to soups and casseroles
    \item Reheat gently with additional broth for extra flavor
\end{itemize}

\end{multicols}
}

\end{document}

