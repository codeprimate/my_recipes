\documentclass[11pt,letterpaper]{article}
\usepackage{fontspec}
\usepackage{tocloft}
\usepackage{multicol}
\usepackage[left=1.4in,right=1.5in,top=1in,bottom=1in]{geometry}
\setmainfont[Scale=1.4,AutoFakeBold=1.5,AutoFakeSlant=0.3]{Doves Type}

\title{Steamed Cabbage (Sweet and Sour)}
\author{}
\date{}

\begin{document}

\maketitle
\thispagestyle{empty}

\section*{Ingredients}
\setlength{\columnsep}{20pt}
\begin{multicols}{2}
\noindent
    Round cabbage \dotfill 1 head \\
    Rice vinegar \dotfill 3½ Tbsp. \\
    Sake \dotfill 2 Tbsp. \\
    Mirin \dotfill 2 Tbsp. \\
    Light soy sauce \dotfill 1 tsp. \\
    \columnbreak
    Grapeseed oil \dotfill 1 Tbsp. \\
    Ginger powder \dotfill ½ tsp. \\
    Sesame oil \dotfill ½ tsp. \\
    Fresh lime juice \dotfill 1 tsp. \\
    White pepper \dotfill pinch \\
    Candied ginger, minced \dotfill 2 pieces \\
    Salt \dotfill if needed \\
\end{multicols}

\section*{Directions}

\noindent
Remove outer leaves from \textbf{cabbage} and cut into 1-inch strips ---
Discard core ---
Mince \textbf{candied~ginger} fine ---
Set up steamer

\begin{enumerate}
    \item Steam \textbf{cabbage} strips for \textit{6-8~minutes} until tender but with slight bite.
    
    \item While \textbf{cabbage} steams, heat \textbf{grapeseed~oil} in wok or skillet over medium heat. Add \textbf{ginger~powder} and bloom for \textit{30~seconds} until fragrant.
    
    \item Add \textbf{sake}, \textbf{rice~vinegar}, and \textbf{mirin}. Bring to simmer and reduce by about one-third to concentrate flavors and mellow acidity, about \textit{3-4~minutes}.
    
    \item Stir in \textbf{light~soy~sauce}, \textbf{white~pepper}, and minced \textbf{candied~ginger}. Cook for another \textit{1-2~minutes} to integrate the \textbf{candied~ginger} flavors.
    
    \item Remove from heat and stir in \textbf{sesame~oil} and \textbf{fresh~lime~juice}.
    
    \item Add steamed \textbf{cabbage} to the pan with the sauce. Toss gently for \textit{1~minute} to coat evenly.
    
    \item Serve warm or at room temperature.
\end{enumerate}

\newpage

% Begin compact two-column layout
{\small
\setlength{\columnsep}{20pt}
\setlength{\multicolsep}{6pt}
\begin{multicols}{2}
\setlength{\parindent}{0pt}
\setlength{\parskip}{4pt}

\subsection*{Equipment Required}
\begin{itemize}
    \item Steamer setup (bamboo steamer, electric steamer, or large pot with steaming rack)
    \item Wok or large skillet for sauce
    \item Sharp knife and cutting board
    \item Measuring spoons and cups
    \item Wooden spoon or spatula
    \item Serving platter
    \item Small bowl for mincing candied ginger
    \item Fine-mesh strainer (optional, for lime juice)
\end{itemize}

\subsection*{Mise en Place}
\begin{itemize}
    \item Set up steamer and bring water to boil before starting
    \item Have all sauce ingredients measured and ready
    \item Mince \textbf{candied~ginger} finely before cooking begins
    \item Cut \textbf{cabbage} just before steaming to prevent oxidation
    \item Juice \textbf{lime} fresh and strain if desired
\end{itemize}

\subsection*{Ingredient Tips}
\begin{itemize}
    \item Choose firm, heavy \textbf{cabbage} heads with tight, crisp leaves
    \item \textbf{Grapeseed~oil} can be substituted with vegetable or canola oil
    \item Quality \textbf{sake} makes a difference - avoid cooking sake if possible
    \item \textbf{Candied~ginger} should be soft and pliable, not dried out
    \item \textbf{White~pepper} provides clean heat without competing with other flavors
    \item Use fresh \textbf{lime~juice} only - bottled lacks the volatile oils needed for palate cleansing
\end{itemize}

\subsection*{Preparation Tips}
\begin{itemize}
    \item Cut \textbf{cabbage} strips uniformly for even cooking
    \item Watch steaming time carefully - smaller pieces cook faster than wedges
    \item Don't over-reduce the sauce - it should remain light and bright
    \item Bloom \textbf{ginger~powder} carefully to avoid burning
    \item Add \textbf{sesame~oil} and \textbf{lime~juice} off heat to preserve delicate aromatics
    \item Taste sauce before final seasoning - \textbf{candied~ginger} adds natural sweetness
    \item Toss gently to avoid breaking the tender \textbf{cabbage} strips
\end{itemize}

\subsection*{Make Ahead \& Storage}
\begin{itemize}
    \item \textbf{Cabbage} can be cut up to \textit{2~hours} ahead and stored covered
    \item Sauce can be made up to \textit{1~day} ahead and gently rewarmed
    \item Add \textbf{lime~juice} only when ready to serve for maximum brightness
    \item Best served fresh, but leftovers keep \textit{2~days} refrigerated
    \item Reheat gently or serve at room temperature as a cold salad
    \item Do not freeze - texture will be compromised
\end{itemize}

\subsection*{Serving Suggestions}
\begin{itemize}
    \item Perfect as palate cleanser alongside spam musubi or other rich dishes
    \item Serve between every 2-3 pieces of musubi for optimal effect
    \item Excellent with grilled meats or fried foods
    \item Can be served warm or at room temperature
    \item Provide small chopsticks or forks for easy pickup
    \item Pairs well with steamed rice and other Asian-inspired sides
\end{itemize}

\end{multicols}
}

\end{document}