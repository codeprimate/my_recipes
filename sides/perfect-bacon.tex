\documentclass[11pt,letterpaper]{article}
\usepackage{fontspec}
\usepackage{tocloft}
\usepackage{multicol}
\usepackage[left=1.4in,right=1.5in,top=1in,bottom=1in]{geometry}
\setmainfont[Scale=1.4,AutoFakeBold=1.5,AutoFakeSlant=0.3]{Doves Type}

\title{Perfect Bacon •}
\author{}
\date{}

\begin{document}

\maketitle
\thispagestyle{empty}

\begin{quote}
\small
\begin{em}
A cold-start technique for consistently crisp, flat bacon every time. Starting in an unheated oven allows the fat to render slowly and evenly while the meat cooks: no curling, no burning, no fuss.
\end{em}
\end{quote}

\section*{Ingredients}
\noindent
Bacon \dotfill as needed

\section*{Directions}

\noindent
Remove \textbf{bacon} from refrigerator \textit{1~hour} before cooking and allow to come to room temperature ---
\textbf{DO NOT} preheat oven. A cool oven is the key to success.


\begin{enumerate}
    \item Place wire cooling rack(s) on baking sheet(s)and arrange \textbf{bacon} strips in single layer, not touching.
    
    \item Place baking sheets with \textbf{bacon} on rack(s) in a cold oven.
    
    \item Set oven to \textit{405°F} and start timer for \textit{20~minutes}.
    
    \item Check \textbf{bacon} at \textit{20~minutes}. It is done when golden brown and stiff. Bacon will crisp further as it cools. Depending on oven and cut, another \textit{5–10~minutes} may be necessary.
    
    \item Transfer \textbf{bacon} to a paper towel-lined plate, (optionally) layering paper towels between strips to absorb excess grease.
\end{enumerate}

\end{document}

