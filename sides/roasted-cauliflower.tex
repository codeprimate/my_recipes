\documentclass[11pt,letterpaper]{article}
\usepackage{fontspec}
\usepackage{tocloft}
\usepackage{multicol}
\usepackage[left=1.4in,right=1.5in,top=1in,bottom=1in]{geometry}
\setmainfont[Scale=1,AutoFakeBold=1.5,AutoFakeSlant=0.3]{Doves Type}

\title{Roasted Cauliflower}
\author{}
\date{}

\begin{document}

\maketitle
\thispagestyle{empty}

\section*{Ingredients}
\setlength{\columnsep}{20pt}
\begin{multicols}{2}
\noindent
    Cauliflower, whole head \dotfill 1 large (2-3 lbs.) \\
    Olive oil \dotfill ¼ cup \\
    Garlic cloves, minced \dotfill 4 \\
    Fresh thyme leaves \dotfill 2 Tbsp. \\
    Fresh rosemary, chopped \dotfill 1 Tbsp. \\
    Garlic powder \dotfill ½ tsp. \\
    Onion powder \dotfill ½ tsp. \\
    Smoked paprika \dotfill ¼ tsp. \\
    \columnbreak
    Lemon zest \dotfill 1 Tbsp. \\
    Lemon juice \dotfill 2 Tbsp. \\
    Salt \dotfill 1 tsp. \\
    Black pepper \dotfill ½ tsp. \\
    Parmesan cheese, grated \dotfill ½ cup \\
    Pork rinds, crushed \dotfill 1 cup \\
    Gluten-free flour blend \dotfill 2 Tbsp. \\
    Butter, melted \dotfill 2 Tbsp. \\
\end{multicols}

\section*{Directions}

\noindent
Preheat oven to \textit{425°F} ---
Line a baking sheet with parchment paper ---
Bring a large pot of salted water to boil ---
Prepare an ice bath ---
Crush \textbf{pork rinds} in food processor or by hand ---
Mince \textbf{garlic} and chop herbs ---
Zest and juice \textbf{lemon}

\begin{enumerate}
    \item Remove the outer leaves from the \textbf{cauliflower} and trim the stem, keeping the head intact. Carefully cut out the core with a small knife, ensuring the head remains whole.
    
    \item Blanch the \textbf{cauliflower} by carefully lowering the head into the boiling water, stem-side up. Boil for \textit{3-4~minutes}, then transfer to the ice bath for \textit{1~minute} to stop the cooking process. Drain thoroughly and pat dry with paper towels.
    
    \item In a small bowl, mix \textbf{olive oil}, minced \textbf{garlic}, \textbf{thyme}, \textbf{rosemary}, \textbf{garlic powder}, \textbf{onion powder}, \textbf{smoked paprika}, \textbf{lemon zest}, 1 Tbsp. \textbf{lemon juice}, \textbf{salt}, and \textbf{pepper}.
    
    \item Place the \textbf{cauliflower} stem-side down on the prepared baking sheet. Brush the entire surface generously with the herb oil mixture, ensuring it gets between the florets.
    
    \item Roast in the preheated oven for \textit{40~minutes}, brushing with more herb oil mixture halfway through cooking.
    
    \item While roasting, prepare the crust mixture: In a medium bowl, combine crushed \textbf{pork rinds}, grated \textbf{Parmesan}, \textbf{gluten-free flour}, remaining \textbf{lemon juice}, and melted \textbf{butter}. Mix until it forms a coarse, slightly moist mixture.
    
    \item After \textit{40~minutes} of roasting, remove the \textbf{cauliflower} and carefully press the crust mixture onto the top and sides of the cauliflower. Return to the oven and roast for an additional \textit{15-20~minutes} until the crust is golden brown and the cauliflower is tender when pierced with a knife.
    
    \item Let rest for \textit{5~minutes} before transferring to a serving platter. Slice into wedges and serve hot.
\end{enumerate}

\newpage

% Begin compact two-column layout
{\small
\setlength{\columnsep}{20pt}
\setlength{\multicolsep}{6pt}
\begin{multicols}{2}
\setlength{\parindent}{0pt}
\setlength{\parskip}{4pt}

\subsection*{Equipment Required}
\begin{itemize}
    \item Large pot (6-8 quart) for blanching
    \item Large bowl for ice bath
    \item Rimmed baking sheet
    \item Parchment paper
    \item Small mixing bowl for herb oil
    \item Medium mixing bowl for crust mixture
    \item Pastry brush or spoon for coating
    \item Measuring cups and spoons
    \item Sharp knife and cutting board
    \item Food processor or zip-top bag and rolling pin (for crushing pork rinds)
    \item Microplane or zester
    \item Citrus juicer
    \item Kitchen towel or paper towels
    \item Serving platter
\end{itemize}

\subsection*{Mise en Place}
\begin{itemize}
    \item Prepare the \textbf{cauliflower} by removing leaves and trimming stem before preheating the oven
    \item Crush \textbf{pork rinds} and store in airtight container until needed
    \item Prepare herb oil mixture before blanching the cauliflower
    \item Set up ice bath before bringing water to boil
    \item Have all ingredients measured and ready before beginning cooking process
\end{itemize}

\subsection*{Ingredient Tips}
\begin{itemize}
    \item Choose a firm, compact \textbf{cauliflower} with tight florets and no brown spots
    \item For maximum flavor, use freshly grated \textbf{Parmesan} rather than pre-grated
    \item Plain, unflavored \textbf{pork rinds} work best as a neutral crispy base
    \item Verify your \textbf{gluten-free flour blend} contains xanthan gum; if not, add ¼ tsp.
    \item Use high-quality cold-pressed \textbf{olive oil} for best flavor
    \item Fresh herbs provide superior flavor, but dried can be substituted (1 tsp. dried thyme, ½ tsp. dried rosemary)
    \item For extra richness, substitute some of the olive oil with duck fat or schmaltz
    \item Check that your \textbf{Parmesan} is truly gluten-free (some brands use anti-caking agents)
\end{itemize}

\subsection*{Preparation Tips}
\begin{itemize}
    \item The blanching step is crucial for ensuring the cauliflower cooks evenly
    \item Thoroughly dry the cauliflower after blanching to ensure proper browning
    \item Apply herb oil generously between florets for maximum flavor penetration
    \item If crust browns too quickly, cover loosely with foil
    \item Test for doneness by inserting a knife into the thickest part - it should enter easily
    \item For extra browning on the crust, broil for the final \textit{1-2~minutes} (watch carefully)
    \item Adjust roasting time based on cauliflower size - larger heads may need an additional \textit{10-15~minutes}
    \item Let the cauliflower rest before cutting to allow juices to redistribute
\end{itemize}

\subsection*{Make Ahead \& Storage}
\begin{itemize}
    \item Herb oil can be prepared up to \textit{24~hours} in advance and refrigerated
    \item \textbf{Pork rind} mixture can be prepared \textit{4~hours} ahead and stored at room temperature
    \item The entire dish can be blanched and prepared up to the roasting step \textit{4~hours} in advance
    \item Leftover cauliflower can be refrigerated for up to \textit{3~days}
    \item Reheat leftovers in a \textit{350°F} oven for \textit{10-15~minutes} until warmed through
\end{itemize}

\subsection*{Serving Suggestions}
\begin{itemize}
    \item Present whole on a serving platter and carve at the table for dramatic effect
    \item Serve alongside \textbf{Dad's Brussels Sprouts} for a vegetable-forward meal
    \item Pairs beautifully with roasted meats, particularly lamb or beef
    \item Drizzle with extra herb oil just before serving for added freshness
    \item Garnish with additional fresh herbs and lemon wedges
    \item For a complete meal, serve with a gluten-free grain like quinoa or millet
    \item Accompany with a crisp white wine such as Pinot Grigio or Sauvignon Blanc
\end{itemize}

\end{multicols}
}

\end{document}