\documentclass[11pt,letterpaper]{article}
\usepackage{fontspec}
\usepackage{tocloft}
\usepackage{multicol}
\usepackage[left=1.4in,right=1.5in,top=1in,bottom=1in]{geometry}
\setmainfont[Scale=1.2,AutoFakeBold=1.5,AutoFakeSlant=0.3]{Doves Type}

\title{Perfect Oven-Baked Crumbly Bacon}
\author{}
\date{}

\begin{document}

\maketitle
\thispagestyle{empty}

\begin{quote}
\small
\begin{em}
This technique produces exceptionally crisp, crumbly bacon with a honeycomb-like texture. The two-phase cooking method ensures thorough fat rendering while maintaining structural integrity, resulting in bacon that shatters pleasingly when bitten rather than bending.
\end{em}
\end{quote}

\section*{Ingredients}
\setlength{\columnsep}{20pt}
\begin{multicols}{2}
\noindent
    Thick-cut, dry-cured bacon \dotfill 1 lb. \\
    Ground black pepper \dotfill ¼ tsp. (optional) \\
    Brown sugar \dotfill 1 Tbsp. (optional) \\
\end{multicols}

\section*{Directions}

\noindent
Unwrap \textbf{bacon} and refrigerate uncovered for \textit{2-4~hours} ---
Preheat oven to \textit{350°F} ---
Line half-sheet pan with aluminum foil (optional) ---
Place wire rack on baking sheet

\begin{enumerate}
    \item Remove \textbf{bacon} from refrigerator and allow to reach approximately \textit{60°F} (slightly below room temperature), about \textit{15-20~minutes}.
    
    \item Pat \textbf{bacon} dry with paper towels to remove excess moisture. If using, lightly sprinkle with \textbf{black pepper} or \textbf{brown sugar}.
    
    \item Arrange \textbf{bacon} strips perpendicular to the wire grid of the cooling rack, maintaining ¼" to ½" spacing between strips to promote proper air circulation.
    
    \item Place baking sheet with \textbf{bacon} on the middle rack of a fully preheated oven. Cook at \textit{350°F} for \textit{15~minutes} to initiate fat rendering.
    
    \item Reduce oven temperature to \textit{325°F} and continue cooking for \textit{10-15~minutes}, until \textbf{bacon} develops a mahogany color with small bubbles throughout the meat portions.
    
    \item Remove \textbf{bacon} from oven when it appears slightly less done than desired. It should bend slightly but not immediately break when lifted at one end.
    
    \item Transfer \textbf{bacon} to a fresh wire rack (not paper towels) and allow to rest for \textit{3-5~minutes} in a low-humidity environment.
    
    \item Once cooled slightly, break or cut \textbf{bacon} into desired lengths and serve immediately.
\end{enumerate}

\newpage

% Begin compact two-column layout
{\small
\setlength{\columnsep}{20pt}
\setlength{\multicolsep}{6pt}
\begin{multicols}{2}
\setlength{\parindent}{0pt}
\setlength{\parskip}{4pt}

\subsection*{Equipment Required}
\begin{itemize}
    \item Half-sheet (18" × 13") heavy-gauge aluminum baking sheet
    \item 304 stainless steel wire cooling rack with grid pattern
    \item Aluminum foil (optional, for easier cleanup)
    \item Paper towels
    \item Oven thermometer
    \item Kitchen tongs
    \item Timer
    \item Cutting board and knife (optional, for portioning)
\end{itemize}

\subsection*{Mise en Place}
\begin{itemize}
    \item Refrigerate \textbf{bacon} unwrapped for \textit{2-4~hours} before cooking to create a pellicle
    \item Verify oven temperature with thermometer before cooking
    \item Allow \textit{10~minutes} of oven stabilization after reaching temperature
    \item Have all equipment ready before removing \textbf{bacon} from refrigerator
    \item Prepare a fresh wire rack for the cooling phase
\end{itemize}

\subsection*{Ingredient Tips}
\begin{itemize}
    \item Select center-cut bacon with 60-70\% visual fat distribution for optimal results
    \item Thickness should be approximately 1/8" to 3/16" thick
    \item Artisanal or butcher-shop bacon often provides superior results compared to mass-market alternatives
    \item For a subtle flavor variation, try applewood or hickory-smoked varieties
    \item If adding \textbf{brown sugar}, apply very lightly to avoid burning
\end{itemize}

\subsection*{Preparation Tips}
\begin{itemize}
    \item Position bacon strips perpendicular to wire grid for optimal support
    \item The transition from perfect to overdone occurs rapidly in the final \textit{2-3~minutes}
    \item Visual cues for completion: mahogany color, translucent fat, small bubbles in meat
    \item If bacon bends slightly but doesn't immediately break when lifted, it's ready
    \item For maximum crispness, avoid all contact between strips when arranging on rack
\end{itemize}

\subsection*{Make Ahead \& Storage}
\begin{itemize}
    \item Best served immediately after the \textit{3-5~minute} resting period
    \item Can be stored in refrigerator for up to \textit{4~days} in airtight container
    \item To reheat, place on wire rack in \textit{325°F} oven for \textit{3-5~minutes}
    \item Avoid microwave reheating, which will compromise the crisp texture
    \item For meal prep, partially cook bacon for \textit{15~minutes}, cool, refrigerate, and finish cooking later
\end{itemize}

\subsection*{Serving Suggestions}
\begin{itemize}
    \item Ideal as a standalone breakfast side
    \item Crumble into salads for textural contrast
    \item Use as a topping for baked potatoes or creamy soups
    \item Incorporate into sandwiches or burgers
    \item Pair with maple syrup or honey for a sweet-savory combination
\end{itemize}

\end{multicols}
}

\end{document}