\documentclass[11pt,letterpaper]{article}
\usepackage{fontspec}
\usepackage{tocloft}
\usepackage{multicol}
\usepackage{nicefrac}
\usepackage[left=1.4in,right=1.5in,top=1in,bottom=1in]{geometry}
\setmainfont[Scale=1.4,AutoFakeBold=1.5,AutoFakeSlant=0.3]{Doves Type}

\widowpenalty=10000
\clubpenalty=10000
\interlinepenalty=500

\title{Mayonnaise}
\author{}
\date{}

\begin{document}

\maketitle
\thispagestyle{empty}

\begin{quote}
\textit{A classic cold emulsion of egg yolk and neutral oil, stabilized with mustard and brightened with acid. Made by hand with a whisk, the mayo develops into a thick, glossy, versatile condiment with fresh, bright flavor. Yields about 1\nicefrac{1}{4}~cups.}
\end{quote}

\section*{Ingredients}

\setlength{\columnsep}{20pt}
\begin{multicols}{2}
\noindent
    Large egg yolk \dotfill 1 \\
    Dijon mustard \dotfill 1 tsp. \\
    Fresh lemon juice \dotfill 1 Tbsp. \\
    White wine vinegar \dotfill 1 tsp. \\
    \columnbreak
    Fine sea salt \dotfill \nicefrac{1}{4} tsp. \\
    Granulated sugar \dotfill \nicefrac{1}{4} tsp. \\
    Neutral oil \dotfill 1 cup
\end{multicols}

{\small\textit{*Neutral oil: canola, vegetable, or grapeseed oil. For richer flavor, use half neutral oil and half light olive oil.*}}

\section*{Directions}

\noindent
Bring \textbf{egg yolk} to room temperature --- Measure \textbf{lemon juice} and \textbf{vinegar}; divide \textbf{lemon juice} in half --- Have \textbf{oil} ready in a vessel with a pour spout

\begin{enumerate}
    \item In a medium bowl, combine 1~\textbf{egg yolk} (room temperature), 1~tsp. \textbf{Dijon mustard}, half the \textbf{lemon juice} (\nicefrac{1}{2}~Tbsp.), \nicefrac{1}{4}~tsp. \textbf{salt}, and \nicefrac{1}{4}~tsp. \textbf{sugar}. Whisk vigorously for \textit{30~seconds} until mixture is smooth, slightly thickened, and lightened in color. The \textbf{mustard} and \textbf{egg yolk} should be fully integrated with no streaks.
    
    \item Begin adding \textbf{oil} extremely slowly—literally drop by drop. Add 3-4~drops of \textbf{oil}, then whisk vigorously for \textit{5-10~seconds} until fully incorporated. Repeat this process, adding a few drops at a time and whisking thoroughly between each addition. This is the critical emulsion-building phase. Continue for the first \nicefrac{1}{4}~cup of \textbf{oil}, about \textit{3-5~minutes}. The mixture will begin to thicken noticeably and turn pale yellow, indicating the emulsion has formed successfully.
    
    \item Once the emulsion has formed and the mixture appears thick and glossy (after about \nicefrac{1}{4}~cup \textbf{oil}), you can add \textbf{oil} more quickly. Pour \textbf{oil} in a thin, steady stream (about the thickness of a pencil lead) while whisking constantly and vigorously. Move the whisk in rapid circular motions, ensuring the \textbf{oil} is incorporated as it's added. The mayonnaise will continue to thicken and become very pale, creamy, and glossy. If the mixture becomes extremely thick and difficult to whisk (like stiff frosting), pause and whisk in 1~tsp. water or \textbf{lemon juice} to thin slightly, then continue adding \textbf{oil}.
    
    \item Once all \textbf{oil} is incorporated, add remaining \nicefrac{1}{2}~Tbsp. \textbf{lemon juice} and 1~tsp. \textbf{white wine vinegar}. Whisk thoroughly until the acid is fully integrated. The mayonnaise should be thick, glossy, and hold soft peaks when the whisk is lifted. It should have a smooth, creamy texture with no visible oil separation.
    
    \item Taste and adjust seasoning. The mayonnaise should be well-balanced: mildly tangy from the acid, subtly sweet, and properly salted. Add more \textbf{salt}, \textbf{lemon juice}, or \textbf{sugar} as needed. Remember that flavors will be slightly muted when cold, so season assertively.
    
    \item Transfer to an airtight container and refrigerate for at least \textit{30~minutes} before using to allow flavors to meld and texture to firm slightly. The mayonnaise is ready when it's cold throughout, very thick, and spreadable.
\end{enumerate}

\textbf{If Emulsion Breaks:} If the mayonnaise suddenly becomes thin, greasy, and separated (oil pools on top), the emulsion has broken. To fix: In a clean bowl, add 1~tsp. cold water or fresh \textbf{lemon juice}. Whisk briefly. Slowly whisk in the broken mayonnaise, treating it like the \textbf{oil}—start with drops, then increase to a thin stream as the emulsion reforms. The fresh liquid provides new surface area for re-emulsification. Alternatively, use 1~fresh \textbf{egg yolk} instead of water for even more reliable results.

\newpage

% Begin compact two-column layout
{\small
\setlength{\columnsep}{20pt}
\setlength{\multicolsep}{6pt}
\begin{multicols}{2}
\setlength{\parindent}{0pt}
\setlength{\parskip}{4pt}

\subsection*{Equipment Required}
\begin{itemize}
    \item Medium bowl (stable and wide)
    \item Whisk (balloon whisk preferred)
    \item Measuring cups and spoons
    \item Small vessel with pour spout for oil (liquid measuring cup or small pitcher)
    \item Rubber spatula
    \item Airtight container for storage
    \item Optional: damp kitchen towel (to stabilize bowl while whisking)
\end{itemize}

\subsection*{Yield}
\begin{itemize}
    \item Makes about 1\nicefrac{1}{4}~cups mayonnaise
    \item Approximately 10~servings (2~Tbsp. each)
\end{itemize}

\subsection*{Mise en Place}
\begin{itemize}
    \item Bring \textbf{egg yolk} to room temperature (\textit{30-60~minutes} on counter, or place whole egg in warm water for \textit{5~minutes})—room temperature eggs emulsify more easily
    \item Measure \nicefrac{1}{2}~Tbsp. \textbf{lemon juice} in \textit{Small Bowl~\#1} (for initial mixture)
    \item Measure remaining \nicefrac{1}{2}~Tbsp. \textbf{lemon juice} and 1~tsp. \textbf{vinegar} in \textit{Small Bowl~\#2} (for finishing)
    \item Measure 1~cup \textbf{oil} into vessel with pour spout—have ready
    \item Set damp towel under bowl to prevent spinning while whisking
    \item Have clean bowl and extra \textbf{lemon juice} ready in case emulsion breaks
\end{itemize}

\subsection*{Ingredient Tips}
\begin{itemize}
    \item \textbf{Egg yolk} must be fresh and room temperature for best emulsification
    \item \textbf{Dijon mustard} acts as an emulsifier and adds subtle flavor—do not omit
    \item \textbf{Neutral oils} (canola, vegetable, grapeseed) create classic Hellmann's-style flavor
    \item Light or pure \textbf{olive oil} can replace half the neutral oil for richer flavor; avoid extra virgin olive oil (too assertive and can taste bitter)
    \item \textbf{White wine vinegar} is milder than distilled white vinegar; both work
    \item \textbf{Lemon juice} must be fresh-squeezed—bottled juice has off-flavors
    \item \textbf{Sugar} is optional but mimics commercial mayo sweetness
    \item Fine \textbf{sea salt} or table salt dissolves easily; kosher salt works but may not dissolve completely
\end{itemize}

\subsection*{Preparation Tips}
\begin{itemize}
    \item Room temperature ingredients are critical—cold \textbf{egg yolks} won't emulsify properly and cold \textbf{oil} is harder to incorporate
    \item The drop-by-drop phase for the first \nicefrac{1}{4}~cup \textbf{oil} is essential—rushing this step causes the emulsion to break
    \item Vigorous, constant whisking creates the mechanical action needed to break \textbf{oil} into tiny droplets
    \item Once emulsion forms (mixture thickens noticeably), you can add \textbf{oil} faster, but never in a flood—thin stream only
    \item If mixture gets too thick before all \textbf{oil} is added, thin with water or \textbf{lemon juice}—this provides more liquid phase for the remaining \textbf{oil}
    \item A damp towel under the bowl prevents it from spinning while you whisk
    \item Adding acid in two stages (some at start, rest at finish) helps stabilize emulsion and provides final flavor adjustment
    \item If \textbf{arm} gets tired during whisking, take brief breaks—the partially formed emulsion will hold for \textit{30-60~seconds}
\end{itemize}

\subsection*{Make Ahead \& Storage}
\begin{itemize}
    \item Homemade mayonnaise contains raw \textbf{egg}—refrigerate immediately
    \item Store in airtight container in refrigerator for \textit{3-5~days}
    \item Bring to room temperature \textit{10~minutes} before using for best spreadability and flavor
    \item If mayonnaise separates slightly during storage, whisk vigorously to re-emulsify
    \item Do not freeze—emulsion breaks and texture degrades
    \item Flavor is best within \textit{48~hours} of making
    \item Commercial mayonnaise lasts months due to preservatives and pasteurized eggs; homemade has shorter shelf life
\end{itemize}

\subsection*{Serving Suggestions}
\begin{itemize}
    \item Use as base for sandwich spreads, potato salad, egg salad, tuna salad, and coleslaw
    \item Transform into aioli: add 2-4~cloves crushed \textbf{garlic} with \textbf{egg yolk}, use half \textbf{olive oil}
    \item Make tartar sauce: add chopped pickles, capers, fresh dill, and \textbf{lemon juice}
    \item Create remoulade: add Dijon, capers, cornichons, herbs, and paprika
    \item Make spicy mayo: whisk in sriracha, sambal oelek, or chipotle in adobo
    \item Make herb mayo: blend with fresh basil, tarragon, or chives
    \item Serve with french fries, roasted vegetables, grilled fish, or sandwiches
    \item Use as binder for crab cakes or as spread for burgers
\end{itemize}

\end{multicols}
}

\end{document}
