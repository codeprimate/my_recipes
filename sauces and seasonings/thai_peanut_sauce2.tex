\documentclass[11pt,letterpaper]{article}
\usepackage{fontspec}
\usepackage{tocloft}
\usepackage{multicol}
\usepackage{nicefrac}
\usepackage[left=1.4in,right=1.5in,top=1in,bottom=1in]{geometry}
\setmainfont[Scale=1.4,AutoFakeBold=1.5,AutoFakeSlant=0.3]{Doves Type}

\title{Thai Peanut Sauce (Nam Jim Thua)}
\author{}
\date{}

\begin{document}

\maketitle
\thispagestyle{empty}

\section*{Ingredients}
\setlength{\columnsep}{20pt}
\begin{multicols}{2}
\noindent
    Coconut cream \dotfill 1 cup \\
    Thai red curry paste \dotfill 2 Tbsp. \\
    Natural peanut butter \dotfill 1 cup \\
    Palm sugar (or brown sugar) \dotfill \nicefrac{1}{3} cup \\
    Fish sauce \dotfill 2 Tbsp. \\
    Tamarind paste \dotfill 1 Tbsp. \\
    \columnbreak
    Fresh ginger, minced \dotfill 1 Tbsp. \\
    Garlic cloves, minced \dotfill 3 \\
    Fresh lime juice \dotfill 2 Tbsp. \\
    Roasted peanuts, crushed \dotfill \nicefrac{1}{4} cup \\
    Salt \dotfill \nicefrac{1}{2} tsp. \\
    Water \dotfill \nicefrac{1}{4}-\nicefrac{1}{2} cup \\
\end{multicols}

\section*{Directions}

\noindent
Mince \textbf{ginger} and \textbf{garlic}; combine in \textit{Small Bowl~\#1} ---
Crush \textbf{roasted peanuts}; set aside in \textit{Small Bowl~\#2} ---
Juice \textbf{lime}; set aside in \textit{Small Bowl~\#3} ---
Chop \textbf{palm sugar} if using block form; set aside in \textit{Small Bowl~\#4}

\begin{enumerate}
    \item In a heavy-bottomed saucepan over medium heat, cook \textbf{coconut cream} until it begins to separate and the oil rises to the surface, about \textit{3-4~minutes}.
    
    \item Add \textbf{Thai red curry paste} to the separated coconut cream and fry until fragrant and the oil turns slightly red, about \textit{2~minutes}.
    
    \item Reduce heat to medium-low. Add \textbf{peanut butter} and stir constantly until well combined and smooth, about \textit{2~minutes}.
    
    \item Add \textbf{palm sugar}, \textbf{fish sauce}, and \textbf{tamarind paste}. Stir until sugar dissolves completely, about \textit{2~minutes}.
    
    \item Add minced \textbf{ginger} and \textbf{garlic} (\textit{Small Bowl~\#1}). Cook for \textit{1~minute} until fragrant.
    
    \item Add \nicefrac{1}{4}~cup \textbf{water} and simmer for \textit{3-4~minutes}, stirring occasionally. Add more \textbf{water} if needed to reach desired consistency (up to \nicefrac{1}{2}~cup total).
    
    \item Remove from heat and stir in \textbf{lime juice} (2~Tbsp., \textit{Small Bowl~\#3}), \textbf{crushed peanuts} (\nicefrac{1}{4}~cup, \textit{Small Bowl~\#2}), and \nicefrac{1}{2}~tsp. \textbf{salt}. Taste and adjust seasoning if needed.
    
    \item Let cool for \textit{10~minutes} before serving. Sauce will thicken as it cools.
\end{enumerate}

\newpage

% Begin compact two-column layout
{\small
\setlength{\columnsep}{20pt}
\setlength{\multicolsep}{6pt}
\begin{multicols}{2}
\setlength{\parindent}{0pt}
\setlength{\parskip}{4pt}

\subsection*{Equipment Required}
\begin{itemize}
    \item Heavy-bottomed saucepan (2-quart)
    \item Wooden spoon or silicone spatula
    \item Measuring cups and spoons
    \item Microplane or fine grater (for ginger)
    \item Garlic press (optional)
    \item Mortar and pestle (or food processor for peanuts)
    \item Fine-mesh strainer (optional)
    \item Glass storage container with lid
    \item Citrus juicer
    \item Sharp knife and cutting board
\end{itemize}

\subsection*{Mise en Place}
\begin{itemize}
    \item Have all ingredients measured and ready before starting
    \item Bring \textbf{peanut butter} to room temperature
    \item Mince aromatics just before cooking
    \item Crush \textbf{peanuts} ahead of time
    \item If using block \textbf{palm sugar}, chop finely
\end{itemize}

\subsection*{Ingredient Tips}
\begin{itemize}
    \item Use natural, unsweetened \textbf{peanut butter} for best results
    \item Coconut cream, not milk, provides proper thickness
    \item Mae Ploy or Maesri \textbf{curry paste} recommended
    \item Palm sugar preferred, but brown sugar works well
    \item Fresh \textbf{lime juice} only - never bottled
    \item Use Thai fish sauce (Nam Pla) for authentic flavor
\end{itemize}

\subsection*{Preparation Tips}
\begin{itemize}
    \item Watch coconut cream carefully - it should separate but not burn
    \item Stir constantly when adding \textbf{peanut butter} to prevent sticking
    \item Sauce will thicken significantly as it cools
    \item For extra smooth sauce, strain through fine-mesh strainer
    \item Add \textbf{water} gradually to control consistency
    \item Reserve some \textbf{crushed peanuts} for garnish
\end{itemize}

\subsection*{Make Ahead \& Storage}
\begin{itemize}
    \item Keeps refrigerated for up to \textit{1~week}
    \item Bring to room temperature before serving
    \item Reheat gently over low heat, stirring frequently
    \item Add warm water to thin if needed after refrigeration
    \item Freeze for up to \textit{3~months}
\end{itemize}

\subsection*{Serving Suggestions}
\begin{itemize}
    \item Perfect for chicken satay or grilled meats
    \item Serve with fresh spring rolls
    \item Use as a dip for raw vegetables
    \item Thin with coconut milk for salad dressing
    \item Drizzle over grilled chicken or shrimp
    \item Garnish with extra \textbf{crushed peanuts} and cilantro
\end{itemize}

\end{multicols}
}

\end{document}
