\documentclass[11pt,letterpaper]{article}
\usepackage{fontspec}
\usepackage{tocloft}
\usepackage{multicol}
\usepackage[left=1.4in,right=1.5in,top=1in,bottom=1in]{geometry}
\setmainfont[Scale=1.4,AutoFakeBold=1.5,AutoFakeSlant=0.3]{Doves Type}

\title{Tzatziki (Cucumber-Yogurt Salad)}
\author{}
\date{}

\begin{document}

\maketitle
\thispagestyle{empty}

\section*{Ingredients}
\setlength{\columnsep}{20pt}
\begin{multicols}{2}
\noindent
    English cucumber \dotfill 1 \\
    Full-fat goat milk yogurt \dotfill 2 cups \\
    Garlic cloves \dotfill 5-6 \\
    Red wine vinegar \dotfill 1½-2 Tbsp. \\
    \columnbreak
    Extra virgin olive oil \dotfill 3-4 Tbsp. \\
    Fresh dill, chopped \dotfill 2-3 Tbsp. \\
    Fresh mint, chopped \dotfill 1-2 Tbsp. \\
    Kosher salt \dotfill 1½ tsp. \\
    Red pepper flakes \dotfill ¼ tsp.
\end{multicols}

{\small\textit{*Full-fat Greek yogurt (cow's milk) may substitute for goat milk yogurt*}}

\section*{Directions}

\noindent
Grate \textbf{cucumber} on large holes of box grater ---
Grate or very finely mince \textbf{garlic} ---
Chop \textbf{dill} and \textbf{mint} ---
Slice additional \textbf{cucumbers} into spears or rounds for serving

\begin{enumerate}
    \item Grate \textbf{cucumber} on the large holes of a box grater. Transfer to a fine-mesh strainer set over a bowl. Toss with 1~tsp. \textbf{kosher~salt} and let drain for \textit{30-40~minutes}, stirring 2-3~times to encourage drainage.
    
    \item After draining, gather \textbf{cucumber} in a clean kitchen towel (or multiple layers of cheesecloth) and squeeze aggressively to extract maximum liquid. Wring and twist until you cannot extract any more. You should remove ¾-1~cup liquid. The \textbf{cucumber} should feel almost dry when finished.
    
    \item Grate or very finely mince \textbf{garlic}. For more even distribution and less harsh bite, grate on a microplane. For stronger garlic punch with some texture, mince very fine with a knife. Optionally, crush \textbf{garlic} with a pinch of \textbf{salt} and \textbf{pepper} to create a paste.
    
    \item In a medium bowl, add \textbf{goat~milk~yogurt} and prepared \textbf{garlic}. Begin adding \textbf{olive~oil} gradually, 1~tablespoon at a time, while stirring vigorously. Alternate with small additions of \textbf{red~wine~vinegar}. This slow incorporation prevents the oil from pooling on top and creates a smooth, emulsified texture. Add ½~tsp. \textbf{kosher~salt} and mix thoroughly.
    
    \item Add thoroughly drained \textbf{cucumber}, chopped \textbf{dill}, and \textbf{mint} to \textbf{yogurt} mixture. Fold together until evenly combined. The mixture should be very thick.
    
    \item Taste and adjust aggressively. The tzatziki should be punchy with \textbf{garlic}, tangy with \textbf{vinegar}, and well-salted. Remember that flavors will meld and mellow slightly during rest time. Add more \textbf{salt}, \textbf{vinegar}, or \textbf{garlic} as needed.
    
    \item Cover and refrigerate for at least \textit{2-3~hours}, preferably overnight. This allows the \textbf{garlic} to permeate, herbs to hydrate, and flavors to marry. The mixture will thicken slightly as it chills.
    
    \item About \textit{30~minutes} before serving, remove from refrigerator and let sit at cool room temperature—this improves flavor expression and makes the texture more scoopable.
    
    \item Before serving, stir well and taste again. Adjust final seasoning if needed. Transfer to serving bowl, create a shallow well in the center with the back of a spoon, and drizzle with \textbf{olive~oil}. Sprinkle with \textbf{red~pepper~flakes} and garnish with fresh \textbf{dill} leaves.
\end{enumerate}

\newpage

% Begin compact two-column layout
{\small
\setlength{\columnsep}{20pt}
\setlength{\multicolsep}{6pt}
\begin{multicols}{2}
\setlength{\parindent}{0pt}
\setlength{\parskip}{4pt}

\subsection*{Equipment Required}
\begin{itemize}
    \item Box grater with large holes
    \item Fine-mesh strainer
    \item Medium bowl (for straining)
    \item Clean kitchen towel or multiple layers of cheesecloth
    \item Medium mixing bowl
    \item Whisk or sturdy spoon
    \item Microplane or garlic press (optional)
    \item Measuring cups and spoons
    \item Cutting board and sharp knife
    \item Serving bowl
    \item Rubber spatula
\end{itemize}

\subsection*{Mise en Place}
\begin{itemize}
    \item Begin \textbf{cucumber} draining \textit{30-40~minutes} before assembly
    \item Allow \textit{2-3~hours} minimum for resting before serving (overnight is ideal)
    \item Remove from refrigerator \textit{30~minutes} before service
    \item All ingredients should be ready before beginning—once assembly starts, it moves quickly
    \item Prepare serving \textbf{cucumbers} during rest time
\end{itemize}

\subsection*{Ingredient Tips}
\begin{itemize}
    \item English (hothouse) \textbf{cucumbers} have fewer seeds and less water than standard cucumbers
    \item \textbf{Goat~milk~yogurt} provides authentic tangy flavor; traditional Greek tzatziki uses sheep or goat milk
    \item Full-fat yogurt is essential—low-fat versions are too thin and tangy
    \item If using cow's milk \textbf{Greek~yogurt}, choose Fage Total 5\% or similar thick, strained yogurt
    \item \textbf{Red~wine~vinegar} is traditional in Greece; lemon juice is an American adaptation
    \item Fresh \textbf{garlic} is critical; pre-minced or powdered won't provide the same punch
    \item Use young, fresh \textbf{dill} and \textbf{mint}—older herbs can be bitter
    \item \textbf{Dill} is the primary herb; \textbf{mint} is secondary or optional
    \item High-quality, fruity \textbf{extra~virgin~olive~oil} makes a significant difference
    \item \textbf{Aleppo~pepper} provides mild heat and fruity notes; substitute with mild red pepper flakes if unavailable
    \item For serving \textbf{cucumbers}, Persian varieties provide best crunch and minimal seeds
\end{itemize}

\subsection*{Preparation Tips}
\begin{itemize}
    \item Aggressive \textbf{cucumber} squeezing is the single most critical step—insufficient draining yields watery tzatziki
    \item Salt the grated \textbf{cucumber} generously to draw out moisture through osmosis
    \item Grating \textbf{garlic} on microplane creates smoother distribution; mincing creates pockets of stronger flavor
    \item Crushing \textbf{garlic} with \textbf{salt} into a paste mellows the bite slightly
    \item Slow incorporation of \textbf{olive~oil} is essential—add gradually while stirring to prevent separation
    \item Alternate \textbf{oil} and \textbf{vinegar} additions for proper emulsification
    \item The mixture should taste quite assertive when first made—flavors mellow significantly during rest
    \item Overnight resting allows \textbf{garlic} to fully infuse and creates more unified flavor
    \item Bringing to cool room temperature before serving is essential—cold dulls flavor perception
\end{itemize}

\subsection*{Make Ahead \& Storage}
\begin{itemize}
    \item Optimal make-ahead time is overnight in refrigerator
    \item Can be made up to \textit{2~days} ahead; \textbf{garlic} intensifies over time
    \item Traditional Greek cooks make tzatziki in the morning to serve at lunch or dinner
    \item Store covered in refrigerator
    \item If making more than \textit{1~day} ahead, reserve some \textbf{olive~oil} and fresh \textbf{dill} for refreshing before serving
    \item Drain any accumulated liquid before serving if made well ahead
    \item \textbf{Garlic} acts as a preservative—tzatziki keeps up to \textit{1-2~weeks} refrigerated
    \item Does not freeze well—\textbf{yogurt} separates and texture degrades
    \item Flavor becomes more garlicky over time; some prefer this aged flavor
\end{itemize}

\subsection*{Serving Suggestions}
\begin{itemize}
    \item Serve alongside rich, spiced stews and grilled meats for cooling contrast
    \item Essential accompaniment to souvlaki, gyros, and all grilled meats
    \item Excellent with warm flatbread, pita, or crusty bread
    \item Traditional pairing with fried foods like fried potatoes, zucchini fritters, or eggplant
    \item Persian or English \textbf{cucumber} spears make ideal dippers
    \item Can be used as a sauce for grilled vegetables or fish
    \item Garnish serving bowl with additional fresh \textbf{dill}, \textbf{olive~oil}, and \textbf{Aleppo~pepper}
    \item Pairs well with other mezze-style dishes
    \item Drizzle with \textbf{olive~oil} and top with an olive for traditional presentation
\end{itemize}

\end{multicols}
}

\end{document}
