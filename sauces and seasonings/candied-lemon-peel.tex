\documentclass[11pt,letterpaper]{article}
\usepackage{fontspec}
\usepackage{tocloft}
\usepackage{multicol}
\usepackage{nicefrac}
\usepackage[left=1.4in,right=1.5in,top=1in,bottom=1in]{geometry}
\setmainfont[Scale=1.2,AutoFakeBold=1.5,AutoFakeSlant=0.3]{Doves Type}

\title{Candied Lemon Peel}
\author{}
\date{}

\begin{document}

\maketitle
\thispagestyle{empty}

\begin{quote}
\textit{Elegant candied citrus ribbons with concentrated lemon flavor and tender-chewy texture. Perfect as garnish for desserts, folded into baked goods, or enjoyed on their own. Using a dehydrator ensures consistent results and extended shelf life up to 6 months.}
\end{quote}

\section*{Ingredients}
\setlength{\columnsep}{20pt}
\begin{multicols}{2}
\noindent
    Lemons, large \dotfill 3 \\
    Granulated sugar \dotfill 1\nicefrac{1}{2} cups \\
    \columnbreak
    Water \dotfill 1\nicefrac{1}{2} cups \\
    Superfine sugar \dotfill \nicefrac{1}{3}--\nicefrac{1}{2} cup \\
    Kosher salt \dotfill \nicefrac{1}{8} tsp.
\end{multicols}

{\small\textit{*Yields approximately 1--1\nicefrac{1}{4} cups candied peel. Meyer lemons work beautifully and candy faster due to thinner, sweeter peel. Use leftover peels from zested/juiced lemons for zero waste.*}}

\section*{Directions}

\noindent
Cut \textbf{lemons} into quarters lengthwise ---
Remove flesh and white pith (reserve for other use); set peels aside in \textit{Large Bowl~\#1} ---
Slice peel into \nicefrac{1}{8}~inch strips ---
Prepare dehydrator with lined trays

\begin{enumerate}
    \item Quarter \textbf{lemons} lengthwise. Use a spoon or small knife to scoop out all flesh and juice (reserve for another use). Use a paring knife to carefully remove most of the white pith from the inside of the peel, leaving a thin layer attached to the yellow zest. This reduces bitterness while maintaining structure. Cut peels into \nicefrac{1}{8}~inch × 2~inch strips. Transfer to \textit{Large Bowl~\#1}.
    
    \item Bring a large pot of water to a rolling boil. Add \textbf{lemon peel} strips (\textit{Large Bowl~\#1}) and boil for \textit{2~minutes}. Drain completely in a colander. This is the first blanch.
    
    \item Return peels to the pot with fresh cold water. Bring to a boil again and cook for \textit{2~minutes}. Drain completely. This is the second blanch.
    
    \item Repeat the blanching process one more time with fresh cold water—bring to boil, cook \textit{2~minutes}, drain completely. This third blanch removes most of the bitterness from the pith while softening the peel. The strips should be tender but not falling apart. Transfer drained peels to \textit{Medium Bowl~\#1}.
    
    \item In a large, wide saucepan or deep skillet, combine 1\nicefrac{1}{2}~cups \textbf{granulated~sugar}, 1\nicefrac{1}{2}~cups \textbf{water}, and \nicefrac{1}{8}~tsp. \textbf{kosher~salt}. Stir over medium heat until sugar dissolves completely, about \textit{2-3~minutes}.
    
    \item Add blanched \textbf{lemon peels} (\textit{Medium Bowl~\#1}) to the simple syrup. The peels should be mostly submerged. Bring to a gentle simmer, then reduce heat to maintain a bare simmer—small bubbles breaking the surface occasionally, not a rolling boil.
    
    \item Simmer gently, stirring occasionally, for \textit{40--50~minutes}. The peels will gradually become translucent and glossy, taking on a jewel-like appearance. The syrup will thicken slightly but should not caramelize. Monitor heat carefully—if syrup begins to color or thicken rapidly, reduce heat and add 2--3~Tbsp. water. The peels are ready when they're completely translucent, tender but still intact, and the syrup has reduced by about one-third.
    
    \item Remove from heat. Allow peels to cool in the syrup for \textit{10~minutes}. This helps them absorb more sugar and develop better texture.
    
    \item Using a slotted spoon or spider, transfer \textbf{lemon peels} to a wire cooling rack set over a baking sheet. Let drain for \textit{5~minutes}, allowing excess syrup to drip off. Reserve the syrup in \textit{Medium Bowl~\#2}—it's excellent in cocktails, tea, or drizzled over desserts.
    
    \item While peels are still warm and slightly tacky, transfer to a shallow bowl. Add \nicefrac{1}{3}--\nicefrac{1}{2}~cup \textbf{superfine sugar} and toss gently but thoroughly to coat all surfaces. The slight tackiness helps the sugar adhere. For less sweet peels, skip this step or use less sugar.
    
    \item Line dehydrator trays with non-stick sheets, silicone mats, or parchment paper (if your dehydrator allows). Arrange sugar-coated \textbf{lemon peels} in a single layer with space between strips—they should not touch or overlap.
    
    \item Set dehydrator to \textit{135°F}. Dehydrate for \textit{5--7~hours}, checking progress every hour after the 4-hour mark. For \textbf{tender-chewy peels} (ideal for garnish and eating): Remove when peels are pliable and slightly tacky but no longer wet, about \textit{5--6~hours}. They should bend without breaking and have some give when squeezed. For \textbf{crisp-candied peels} (ideal for chopping into baked goods): Continue dehydrating until completely dry and brittle, about \textit{6--7~hours}. They should snap cleanly when bent.
    
    \item Remove trays from dehydrator and let peels cool completely on the trays, about \textit{20~minutes}. As they cool, they'll firm up slightly and the sugar coating will set.
    
    \item Transfer cooled peels to an airtight container, layering with parchment paper between layers to prevent sticking. Store at room temperature in a cool, dry place for up to \textit{4--6~months}.
\end{enumerate}

\newpage

% Begin compact two-column layout
{\small
\setlength{\columnsep}{20pt}
\setlength{\multicolsep}{6pt}
\begin{multicols}{2}
\setlength{\parindent}{0pt}
\setlength{\parskip}{4pt}

\subsection*{Equipment Required}
\begin{itemize}
    \item Food dehydrator with temperature control
    \item Non-stick dehydrator sheets, silicone mats, or parchment paper
    \item Large pot (4-quart capacity) for blanching
    \item Large, wide saucepan or deep skillet (3-4 quart capacity) for candying
    \item Colander for draining
    \item Wire cooling rack
    \item Rimmed baking sheet (for catching drips)
    \item Slotted spoon or spider
    \item Sharp paring knife
    \item Sharp chef's knife for slicing
    \item Cutting board
    \item Measuring cups
    \item Shallow bowl for sugar coating
    \item Large prep bowl (for cut peels)
    \item Medium prep bowl (for blanched peels)
    \item Airtight storage container
    \item Parchment paper for layering
\end{itemize}

\subsection*{Mise en Place}
\begin{itemize}
    \item \textit{Large Bowl~\#1} --- raw \textbf{lemon peel} strips (about 2--2\nicefrac{1}{2}~cups)
    \item \textit{Medium Bowl~\#1} --- triple-blanched \textbf{lemon peels} (ready for simmering)
    \item \textit{Medium Bowl~\#2} --- reserved lemon-infused simple syrup (about 1--1\nicefrac{1}{4}~cups)
    \item Prepare dehydrator location with access to power; process takes 5--7~hours
    \item Set up draining station: wire rack over rimmed baking sheet
    \item This recipe requires active time for prep and monitoring, plus 5--7~hours unattended dehydrator time
    \item Work in batches if your dehydrator has limited tray space
\end{itemize}

\subsection*{Ingredient Tips}
\begin{itemize}
    \item Choose \textbf{lemons} with thick, unblemished skin—more peel means higher yield
    \item \textbf{Meyer lemons} have thinner, sweeter peel with less pith—they candy faster and are less bitter
    \item Eureka or Lisbon \textbf{lemons} (standard supermarket) have thicker pith—remove more white pith when prepping
    \item Organic \textbf{lemons} are ideal if eating the peel, but wash all lemons thoroughly regardless
    \item Use leftover peels from recipes requiring only zest or juice—zero waste approach
    \item \textbf{Superfine sugar} (also called caster sugar) adheres better than granulated; make your own by pulsing granulated sugar in a food processor for \textit{30~seconds}
    \item The reserved lemon-infused simple \textbf{syrup} keeps refrigerated for \textit{2~weeks}—use in cocktails, lemonade, iced tea, or drizzle over cakes and pound cake
    \item \textbf{Salt} in the syrup enhances lemon flavor and helps preserve the peels
\end{itemize}

\subsection*{Preparation Tips}
\begin{itemize}
    \item Leave a thin layer of white \textbf{pith} attached—complete removal makes peels fragile; too much pith makes them bitter
    \item Cut strips uniformly for even candying and dehydrating
    \item Don't skip the triple blanch—it's essential for removing bitterness while maintaining structure
    \item Use fresh cold water for each blanch to maximize bitterness removal
    \item Keep syrup at a bare simmer, not a rolling boil—high heat can toughen peels or caramelize the syrup
    \item Stir occasionally during simmering to ensure even coverage and prevent sticking
    \item Watch for translucency as your doneness indicator—peels should look jewel-like and glossy
    \item If syrup reduces too quickly or begins to color, add water 2--3~Tbsp. at a time
    \item Toss in sugar while peels are still warm—the tackiness helps coating adhere
    \item For less-sweet peels, skip the sugar coating or use only 2--3~Tbsp.
    \item Space peels on dehydrator trays—touching peels will stick together
    \item Check progress after 4~hours, then every hour—dehydration time varies by humidity, dehydrator model, and peel thickness
    \item \textbf{Meyer lemon} peels dehydrate faster (4--5~hours) than standard lemons (5--7~hours)
    \item Don't over-dehydrate for garnish use—tender-chewy is the goal
    \item Let cool completely before storing—warm peels create condensation and reduce shelf life
\end{itemize}

\subsection*{Make Ahead \& Storage}
\begin{itemize}
    \item Properly dehydrated \textbf{candied lemon peels} store at room temperature for \textit{4--6~months}
    \item Store in airtight container with tight-fitting lid
    \item Layer with parchment paper to prevent sticking
    \item Keep in cool, dry place away from direct sunlight and heat
    \item If peels become sticky during storage (humid climate), re-toss in fresh \textbf{superfine sugar}
    \item For extra-long storage, refrigerate in airtight container for up to \textit{1~year}
    \item Can be frozen for up to \textit{18~months}—freeze in single layer on tray, then transfer to freezer bag
    \item The blanched peels (before candying) can be frozen for up to \textit{3~months}—candy from frozen when ready
    \item Reserved lemon-infused \textbf{syrup} keeps refrigerated for \textit{2~weeks}; freeze in ice cube trays for longer storage
    \item Make large batches when lemons are in season (winter/spring) for year-round supply
\end{itemize}

\subsection*{Serving Suggestions}
\begin{itemize}
    \item \textbf{As dessert garnish}: Place small curls on lemon bars, tarts, cakes, panna cotta, or ice cream
    \item \textbf{For teatime}: Serve alongside shortbread, scones, or petit fours
    \item \textbf{In baking}: Chop crisp-candied peels and fold into scone dough, muffin batter, or cookie dough
    \item \textbf{For gifting}: Package in clear cellophane bags or small glass jars with ribbon—makes elegant homemade gifts
    \item \textbf{In cocktails}: Use as edible stirrer or muddle into drinks for bright citrus flavor
    \item \textbf{With chocolate}: Dip one end in melted dark chocolate and let set on parchment—classic French confection
    \item \textbf{For charcuterie}: Add to cheese boards with soft cheeses, marcona almonds, and dark chocolate
    \item \textbf{As snack}: Enjoy tender-chewy peels on their own as a refined sweet treat
    \item \textbf{In granola or trail mix}: Chop and add for bright citrus notes
    \item \textbf{For ice cream}: Fold chopped peels into softened vanilla ice cream and refreeze
    \item Pair with other candied citrus—orange and grapefruit peels use identical method
\end{itemize}

\end{multicols}
}

\end{document}

