\documentclass[11pt,letterpaper]{article}
\usepackage{fontspec}
\usepackage{tocloft} % For dotted lines
\usepackage{multicol}
\usepackage[left=1.4in,right=1.5in,top=1in,bottom=1in]{geometry}
\setmainfont[Scale=1.4,AutoFakeBold=1.5,AutoFakeSlant=0.3]{Doves Type}

\title{Thai Peanut Sauce}
\author{}
\date{}

\begin{document}

\maketitle
\thispagestyle{empty}

\section*{Ingredients}
\setlength{\columnsep}{20pt}
\begin{multicols}{2}
\noindent
    Garlic cloves \dotfill 2 \\
    Lime \dotfill 1 \\
    Creamy peanut butter \dotfill ¾ cup \\
    Thai red curry paste \dotfill 2-4 Tbsp. \\
    \columnbreak
    Coconut milk \dotfill 13.5 oz. can \\
    Turbinado sugar \dotfill ½ cup \\
    Tamarind paste \dotfill 2 Tbsp. \\
    Water \dotfill ½ cup \\
    Salt \dotfill to taste \\
\end{multicols}

\section*{Directions}

\noindent
Mince \textbf{garlic} ---
Juice \textbf{lime}

\begin{enumerate}
    \item In a medium saucepan, whisk together \textbf{coconut milk}, \textbf{peanut butter}, \textbf{curry paste}, \textbf{turbinado sugar}, \textbf{tamarind paste}, \textbf{water}, minced \textbf{garlic}, and \textbf{lime juice}.
    
    \item Heat over medium-low heat, whisking constantly until smooth and well combined, about \textit{5~minutes}.
    
    \item Taste and adjust seasoning with \textbf{salt}, additional \textbf{curry paste} for heat, or \textbf{turbinado sugar} for sweetness.
    
    \item Store in an airtight container in the refrigerator for up to \textit{1~week}. Reheat gently before serving, adding water if needed to thin.
\end{enumerate}

Makes approximately \textit{2½~cups}

Note: Adjust \textbf{curry paste} amount based on desired spice level.

Serve as a dipping sauce for spring rolls, satay, or toss with noodles and vegetables.

\end{document}