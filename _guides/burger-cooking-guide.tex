\documentclass[11pt,letterpaper]{article}
\usepackage{fontspec}
\usepackage{booktabs}
\usepackage{tabularx}
\usepackage{enumitem}
\usepackage{fancyhdr}
\usepackage[left=1.2in,right=1.2in,top=1in,bottom=1in]{geometry}
\setmainfont[Scale=1.2,AutoFakeBold=1.5,AutoFakeSlant=0.3]{Doves Type}

\widowpenalty=10000
\clubpenalty=10000
\interlinepenalty=500

\title{Beef Burgers (Indoors)}
\author{}
\date{}

\pagestyle{fancy}
\fancyhf{}
\fancyhead[C]{\textit{Beef Burgers (Indoors)}}
\fancyfoot[C]{\thepage}
\renewcommand{\headrulewidth}{0pt}
\renewcommand{\footrulewidth}{0pt}

\begin{document}

\maketitle
\thispagestyle{empty}

\section*{Doneness}

\textbf{Medium:} center around \textasciitilde150--155°F (warm pink). \\
\noindent\textbf{Medium-well to well:} 155--160°F or higher (little or no pink).
\\
\\
\textit{Check in the thickest part of the patty when you think they're close.}

\section*{Pans and heat levels}

To keep smoke down, use the oven (no stovetop smoke at all) or cook on a \textbf{medium} burner in a nonstick or enameled cast iron pan.

\begin{description}[labelwidth=4.8cm, leftmargin=5.2cm, style=nextline, itemsep=2pt, parsep=0pt]
\item[Oven]
  \noindent\textbf{Use for burgers:} Bake on a rack set over a sheet pan.
  \par\noindent\textbf{Heat:} 375--400°F.
  \par\noindent\textbf{Notes:} No smoke from the stovetop; you won't get a crust unless you give the burgers a quick finish in a pan.
\item[Nonstick pan or griddle]
  \noindent\textbf{Use for burgers:} Stovetop cooking (pan) or electric griddle.
  \par\noindent\textbf{Heat:} Medium on burner; griddle 350--375°F.
  \par\noindent\textbf{Notes:} Stays relatively smoke-free; high heat can damage the coating and create more smoke.
\item[Enameled cast iron]
  \noindent\textbf{Use for burgers:} Stovetop or shallow frying.
  \par\noindent\textbf{Heat:} Medium.
  \par\noindent\textbf{Notes:} Holds heat evenly and gives you a controlled sear without as much smoke.
\item[Cast iron]
  \noindent\textbf{Use for burgers:} Stovetop.
  \par\noindent\textbf{Heat:} Medium or medium-high.
  \par\noindent\textbf{Notes:} High heat produces a lot of smoke, so stick to medium if the alarm is a concern.
\item[Stovetop grill (flat or ridged)]
  \noindent\textbf{Use for burgers:} Optional sear.
  \par\noindent\textbf{Heat:} Medium.
  \par\noindent\textbf{Notes:} The ridged side lets fat drip and smoke more; the flat side is easier to manage.
\end{description}

\noindent
A reliable low-smoke approach: thaw the patties, then bake them on a rack over a sheet pan until done (around 375--400°F). If you want a bit of crust, finish them in a nonstick or enameled cast iron pan over \textbf{medium} heat for 1--2~minutes per side.

\section*{Burger Types}

Thaw the patties in the fridge for at least \textit{1~hour} before cooking. Season with salt and black pepper on the \textbf{surface only}, right before they go in the pan. Working salt into the meat makes the texture tight and bouncy, so avoid that. \textit{Times below assume typical 4--6~oz patties; use a thermometer and cook to your target temp---thicker patties need longer.}

\begin{description}[labelwidth=4.8cm, leftmargin=5.2cm, style=nextline, itemsep=2pt, parsep=0pt]
\item[Ground chuck (butcher)]
  \noindent\textbf{Fat and behavior:} Usually $\sim$80/20; juicy but more shrink.
  \par\noindent\textbf{Seasoning:} S\&P on surface.
  \par\noindent\textbf{Handling:} Shape gently and dimple the center.
  \par\noindent\textbf{Temp and time:} Stovetop medium or medium-high: 3--4~min per side. Oven: 375--400°F, \textasciitilde10--14~min to 155°F.
  \par\noindent\textbf{Cooking:} No oil nonstick/griddle; oil in cast iron optional. Flip once; don't press.
\item[Ground premium (butcher)]
  \noindent\textbf{Fat and behavior:} Often leaner (85/15--90/10).
  \par\noindent\textbf{Seasoning:} S\&P on surface.
  \par\noindent\textbf{Handling:} Handle very gently; overworking shows up quickly.
  \par\noindent\textbf{Temp and time:} Stovetop medium: 3--4~min per side. Oven: 375°F, \textasciitilde12--16~min.
  \par\noindent\textbf{Cooking:} No oil nonstick/griddle; oil or butter in cast iron. Lower heat than chuck; let rest after cooking.
\item[Cheap frozen chuck patties]
  \noindent\textbf{Fat and behavior:} Higher fat, sometimes extra water or binders.
  \par\noindent\textbf{Seasoning:} S\&P; a touch of garlic or onion powder if they taste bland.
  \par\noindent\textbf{Handling:} Thaw in the fridge when you can; if cooking from frozen, go low and slow.
  \par\noindent\textbf{Temp and time:} Thawed: medium stovetop 3--4~min per side, or oven 375--400°F \textasciitilde10--14~min. Frozen: oven 375°F \textasciitilde16--22~min, then optional pan finish 1--2~min per side.
  \par\noindent\textbf{Cooking:} No oil nonstick/griddle; oil in cast iron optional. They'll shrink more; oven plus quick pan finish works well.
\item[Premium lean frozen]
  \noindent\textbf{Fat and behavior:} Lean and easy to dry out.
  \par\noindent\textbf{Seasoning:} S\&P only.
  \par\noindent\textbf{Handling:} Thaw in the fridge and handle as little as possible.
  \par\noindent\textbf{Temp and time:} Stovetop medium-low to medium: 3--4~min per side. Oven: 375°F, \textasciitilde12--16~min.
  \par\noindent\textbf{Cooking:} No oil nonstick/griddle; oil or butter in cast iron. No pressing, then rest; a little mayo or butter on the bun helps.
\item[Premium Angus frozen]
  \noindent\textbf{Fat and behavior:} Often 80/20 with good balance.
  \par\noindent\textbf{Seasoning:} S\&P on surface.
  \par\noindent\textbf{Handling:} Thaw in the fridge when you can.
  \par\noindent\textbf{Temp and time:} Same as butcher chuck: stovetop 3--4~min per side or oven 375--400°F \textasciitilde10--14~min.
  \par\noindent\textbf{Cooking:} No oil nonstick/griddle; oil in cast iron optional. Medium heat, one flip, then rest.
\end{description}

\section*{Universal rules (all types)}

\begin{center}
\small
\begin{tabularx}{\textwidth}{@{}XX@{}}
\toprule
\textbf{Do} & \textbf{Don't} \\
\midrule
Season the surface with salt and pepper right before cooking & Work salt or heavy spice blends into the meat \\
Shape patties gently and dimple the center & Squeeze or overwork the meat \\
Flip once and avoid pressing & Smash the patties or flip repeatedly \\
Let burgers rest 2--3~minutes after cooking when you & Slice into them right away \\
Use a bit more heat for fattier meat and lower heat for lean & Crank the same high heat for every type \\
\bottomrule
\end{tabularx}
\end{center}

\end{document}
