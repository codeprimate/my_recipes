\documentclass[11pt,letterpaper]{article}
\usepackage[left=1.5in,right=1.5in,top=0.5in,bottom=1in]{geometry}
\usepackage{fontspec}

\setmainfont[Scale=1.4,AutoFakeBold=1.5,AutoFakeSlant=0.4]{Doves Type}

\title{Tuscan Zuppa Soup}
\author{}
\date{}

\begin{document}
\maketitle

\begin{table}[h]
  \centering
  \setlength{\tabcolsep}{1em}
  \renewcommand{\arraystretch}{1.4}
  \begin{tabular}{|l|l|}
    \hline
    \textbf{Ingredient} & \textbf{Quantity} \\
    \hline
    Italian sausage & 1 lb \\
    Onion, medium & 1 \\
    Garlic cloves & 3-4 \\
    Chicken broth & 6 cups \\
    Potatoes, large & 3-4 \\
    Cannellini beans, canned & 1 (15 oz) can \\
    Kale or Swiss chard & 1 bunch \\
    Heavy cream (optional) & 1/2 cup \\
    Salt & To taste \\
    Pepper & To taste \\
    Olive oil & For cooking \\
    Red pepper flakes (optional) & A pinch \\
    \hline
  \end{tabular}
\end{table}

\section*{Directions}
\begin{enumerate}
    \item Brown the Italian sausage over medium heat until fully cooked. Remove and set aside, retaining drippings in the pot.
    \item Sauté onions in the same pot until translucent, then add garlic and cook for an additional minute.
    \item Return sausage to the pot. Add chicken broth and potatoes. Bring to a boil, then simmer until potatoes are tender, about 10-15 minutes.
    \item Stir in kale or Swiss chard and cannellini beans. Cook until greens are wilted and beans are heated through, about 5 minutes.
    \item If desired, stir in heavy cream. Season with salt, pepper, and red pepper flakes to taste.
\end{enumerate}

\end{document}
