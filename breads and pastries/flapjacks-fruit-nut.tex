\documentclass[11pt,letterpaper]{article}
\usepackage{fontspec}
\usepackage{tocloft}
\usepackage{multicol}
\usepackage[left=1.4in,right=1.5in,top=1in,bottom=1in]{geometry}
\setmainfont[Scale=1.4,AutoFakeBold=1.5,AutoFakeSlant=0.3]{Doves Type}

\title{Flapjacks with Fruit and Nuts}
\author{}
\date{}

\begin{document}

\maketitle
\thispagestyle{empty}

\section*{Ingredients}
\setlength{\columnsep}{20pt}
\begin{multicols}{2}
\noindent
    Rolled oats \dotfill 3 cups \\
    Nuts, chopped \dotfill 1 cup \\
    Dried fruit, chopped \dotfill 1 cup \\
    Unsalted butter \dotfill ¾ cup \\
    Light corn syrup \dotfill 2/3 cup \\
    \columnbreak
    Unsulphured molasses \dotfill 2 Tbsp. \\
    Light brown sugar, packed \dotfill ½ cup \\
    Mexican vanilla \dotfill 1 tsp. \\
    Ground cinnamon \dotfill 1½ tsp. \\
    Ground cardamom \dotfill ½ tsp. \\
    Ground nutmeg \dotfill ¼ tsp. \\
    Fine sea salt \dotfill ½ tsp. \\
\end{multicols}

\section*{Directions}

\noindent
Preheat oven to \textit{350°F} ---
Grease and line a 9x9 inch baking pan with parchment paper (leaving overhang for easy removal) ---
Toast \textbf{nuts} at \textit{350°F} for \textit{8-10~minutes} until fragrant, cool, rub in towel to remove skins if applicable, then finely chop ---
Finely chop \textbf{dried~fruit} ---
Place chopped \textbf{fruit} in small bowl, cover with warm water, soak \textit{10~minutes}, drain thoroughly, and pat dry with paper towels

\begin{enumerate}
    \item In a large bowl, combine \textbf{oats}, chopped \textbf{nuts}, \textbf{cinnamon}, \textbf{cardamom}, \textbf{nutmeg}, and \textbf{salt}. Mix well and set aside.
    
    \item In a medium saucepan over medium-low heat, combine \textbf{butter}, \textbf{corn~syrup}, \textbf{molasses}, and \textbf{brown~sugar}. Stir constantly until \textbf{butter} is melted and \textbf{sugar} is dissolved, about \textit{3-4~minutes}. Do not allow mixture to boil.
    
    \item Remove from heat and stir in \textbf{vanilla}.
    
    \item Pour the \textbf{butter} mixture over the \textbf{oat} mixture and stir thoroughly until all \textbf{oats} are evenly coated and mixture is well combined.
    
    \item Add the drained, chopped \textbf{fruit} and fold in gently but thoroughly to distribute evenly throughout.
    
    \item Transfer mixture to prepared baking pan. Press down very firmly and evenly with the back of a spatula or measuring cup to compact the mixture --- this is critical for achieving cohesive bars.
    
    \item Bake for \textit{30~minutes} until golden brown throughout. The edges will be slightly darker and the center should be set but still give slightly when pressed.
    
    \item Remove from oven and allow to cool in pan for \textit{10~minutes}. Using a sharp knife, cut into bars while still warm (12 or 16 bars depending on desired size).
    
    \item Allow bars to cool completely in the pan, at least \textit{2~hours}, before removing. The bars will firm up significantly as they cool.
    
    \item Once completely cool, use parchment overhang to lift from pan. Separate bars along pre-cut lines and store in an airtight container.
\end{enumerate}

\subsection*{Fruit and Nut Pairings}
\begin{itemize}
    \item \textbf{Cherry + Hazelnut} --- Classic pairing with deep, complementary flavors
    \item \textbf{Apricot + Almond} --- Mediterranean combination, honeyed and nutty
    \item \textbf{Cranberry + Pecan} --- Tart-sweet with buttery richness
    \item \textbf{Fig + Walnut} --- Sophisticated, earthy sweetness
    \item \textbf{Blueberry + Macadamia} --- Subtle berry with rich, creamy nuts
    \item \textbf{Raisin + Cashew} --- Traditional, mild pairing
\end{itemize}

\newpage

% Begin compact two-column layout
{\small
\setlength{\columnsep}{20pt}
\setlength{\multicolsep}{6pt}
\begin{multicols}{2}
\setlength{\parindent}{0pt}
\setlength{\parskip}{4pt}

\subsection*{Equipment Required}
\begin{itemize}
    \item 9x9 inch baking pan
    \item Parchment paper
    \item Large mixing bowl (at least 4-quart capacity)
    \item Medium saucepan (2-3 quart)
    \item Small bowl (for soaking fruit)
    \item Rimmed baking sheet (for toasting nuts)
    \item Measuring cups and spoons
    \item Sharp knife and cutting board
    \item Rubber spatula or wooden spoon
    \item Measuring cup or flat-bottomed glass (for pressing)
    \item Kitchen towel (for removing nut skins)
    \item Paper towels (for drying fruit)
\end{itemize}



\subsection*{Mise en Place}
\begin{itemize}
    \item Toast and chop \textbf{nuts} first, as they need time to cool
    \item Soak \textbf{dried~fruit} while preparing other ingredients
    \item Measure all spices and have them ready before starting
    \item Cut parchment paper to fit pan with overhang on two sides
    \item Have all ingredients at room temperature for easier mixing
\end{itemize}

\subsection*{Ingredient Tips}
\begin{itemize}
    \item Use certified gluten-free \textbf{oats} if needed; regular rolled oats work well otherwise
    \item For \textbf{nuts}: toast until fragrant to intensify flavor; hazelnuts and almonds benefit from skin removal
    \item For \textbf{dried~fruit}: choose unsweetened or lightly sweetened varieties; tart cherries, unsulphured apricots, and less-sweetened cranberries provide best flavor balance
    \item Light \textbf{corn~syrup} (Karo) combined with \textbf{molasses} approximates British golden syrup
    \item European-style \textbf{butter} with higher fat content yields richer flavor
\end{itemize}

\subsection*{Preparation Tips}
\begin{itemize}
    \item Chop \textbf{nuts} and \textbf{fruit} finely for even distribution and cohesive texture
    \item Don't skip soaking the \textbf{fruit} --- this prevents hard, dry bits in final bars
    \item Press mixture very firmly into pan; inadequate pressing results in crumbly bars
    \item Watch carefully during final \textit{5~minutes} of baking to prevent over-browning
    \item Cut while warm but allow complete cooling for clean cuts and proper texture
    \item If bars seem too soft after cooling, they may need \textit{5~minutes} more baking time next batch
\end{itemize}

\subsection*{Make Ahead \& Storage}
\begin{itemize}
    \item Flapjacks keep at room temperature in airtight container for up to \textit{1~week}
    \item Layer bars between parchment paper to prevent sticking
    \item Freeze for up to \textit{3~months}; thaw at room temperature for \textit{30~minutes}
    \item Texture firms slightly during storage; this is normal
    \item Do not refrigerate as this can make bars overly hard
\end{itemize}

\subsection*{Serving Suggestions}
\begin{itemize}
    \item Serve as breakfast bars, snacks, or with afternoon tea
    \item Pair with coffee, tea, or cold milk
    \item For dessert, serve warm with vanilla ice cream or whipped cream
    \item Pack individually in parchment paper for portable snacks
    \item Drizzle with melted chocolate for special occasions
\end{itemize}

\end{multicols}
}

\end{document}
