\documentclass[11pt,letterpaper]{article}
\usepackage{fontspec}
\usepackage{tocloft}
\usepackage{multicol}
\usepackage[left=1.4in,right=1.5in,top=1in,bottom=1in]{geometry}
\setmainfont[Scale=1.4,AutoFakeBold=1.5,AutoFakeSlant=0.3]{Doves Type}

\title{Injera Bread}
\author{}
\date{}

\begin{document}

\maketitle
\thispagestyle{empty}

\section*{Ingredients}
\setlength{\columnsep}{20pt}
\begin{multicols}{2}
\noindent
    Teff flour, whole grain \dotfill 2 cups \\
    Water, filtered/bottled \dotfill 3 cups \\
    Active dry yeast \dotfill ¼ tsp. \\
    \columnbreak
    Salt \dotfill ½ tsp. \\
    Baking powder \dotfill ¼ tsp. \\
    Vegetable oil \dotfill 1 tsp. \\
\end{multicols}

\section*{Directions}

\noindent
Measure \textbf{teff~flour} ---
Dissolve \textbf{yeast} in ¼~cup warm \textbf{water} (\textit{105-110°F})

\subsection*{Day 1 (Morning)}

\begin{enumerate}
    \item In a large non-reactive bowl, dissolve \textbf{active~dry~yeast} in ¼~cup warm \textbf{water} (\textit{105-110°F}). Let stand \textit{5~minutes} until foamy. \textit{(Water must be chlorine-free to allow fermentation)}
    
    \item Add \textbf{teff~flour} and 2¼~cups room temperature \textbf{water} to yeast mixture. Whisk vigorously until completely smooth with no lumps. The batter should be thin, similar to crepe batter, and coat a spoon in a thin layer.
    
    \item Cover bowl loosely with clean kitchen towel or cheesecloth (do not seal—fermentation produces CO\textsubscript{2} that must escape). Place in warm location (\textit{68-75°F}) away from direct sunlight.
\end{enumerate}

\subsection*{Days 2-3 (Fermentation)}

\begin{enumerate}
    \setcounter{enumi}{3}
    \item Monitor fermentation progress \textit{twice daily}. Healthy fermentation indicators:
    \begin{itemize}
        \item Surface bubbling and active foaming
        \item Pleasant sour aroma (lactic fermentation, similar to yogurt)
        \item Thin layer of clear liquid (hooch) on surface—stir this back in
        \item Batter will thin slightly as enzymatic activity breaks down starches
    \end{itemize}
    
    \item Check for contamination signs (see detailed notes on page~2). If contamination is suspected, discard and restart.
    
    \item Batter is ready when it shows active bubbling, smells pleasantly sour, and has fermented for \textit{48-72~hours}. In cooler environments, full fermentation may require \textit{3~days}.
\end{enumerate}

\subsection*{Cooking Day}

\begin{enumerate}
    \setcounter{enumi}{6}
    \item Thin batter with remaining ½~cup \textbf{water} if needed—it should pour easily in a thin stream and coat a spoon lightly. Gently stir in \textbf{salt} and \textbf{baking~powder} just before cooking (do not overmix).
    
    \item Heat a 10-12~inch non-stick skillet over medium heat until a drop of \textbf{water} dances on the surface. Very lightly oil the pan with \textbf{vegetable~oil} for the first \textbf{injera} only.
    
    \item Pour approximately ½~cup \textbf{batter} into the pan in a rapid spiral motion from outside edge to center, tilting pan to ensure even coverage. The entire pour should take \textit{3-4~seconds}—speed is critical for uniform thickness.
    
    \item Immediately cover with a tight-fitting lid. Cook for \textit{60-90~seconds} without lifting lid. Surface should be completely dry with numerous small holes (aynet) covering the entire surface. Bottom should be lightly set but not browned.
    
    \item Remove \textbf{injera} without flipping (it cooks on one side only) and place on clean kitchen towel to cool. Do not stack while hot.
    
    \item Repeat with remaining \textbf{batter}, adjusting heat as needed. No additional oil should be necessary after the first \textbf{injera}.
    
    \item Once cooled to room temperature, stack \textbf{injera} between layers of parchment paper or clean towels. Use immediately or store.
\end{enumerate}

\newpage

% Begin compact two-column layout
{\small
\setlength{\columnsep}{20pt}
\setlength{\multicolsep}{6pt}
\begin{multicols}{2}
\setlength{\parindent}{0pt}
\setlength{\parskip}{4pt}

\subsection*{Equipment Required}
\begin{itemize}
    \item Large non-reactive bowl (glass, ceramic, or stainless steel—minimum 3-quart capacity)
    \item Clean kitchen towel or cheesecloth for covering
    \item Wire whisk
    \item 10-12 inch non-stick skillet with tight-fitting lid (essential for steam)
    \item Ladle or measuring cup (½ cup capacity)
    \item Clean kitchen towels for cooling and storage
    \item Instant-read thermometer (for water temperature)
    \item Parchment paper for stacking
\end{itemize}

\subsection*{Mise en Place}
\begin{itemize}
    \item Prepare chlorine-free \textbf{water} at least \textit{24~hours} in advance if using tap water (let stand uncovered to allow chlorine to evaporate)
    \item Ensure fermentation location maintains \textit{68-75°F}—cooler temperatures slow fermentation significantly
    \item Use room temperature ingredients for initial mixing
    \item Have \textbf{salt} and \textbf{baking~powder} measured and ready to add on cooking day
    \item Test skillet with water drop before beginning to cook
\end{itemize}

\subsection*{Ingredient Tips}
\begin{itemize}
    \item \textbf{Teff flour}: Use 100\% whole grain teff for authentic flavor and texture. Brown teff produces more robust flavor; white teff creates milder, lighter-colored \textbf{injera}. Store \textbf{teff~flour} in refrigerator or freezer to prevent rancidity.
    \item Chlorine in tap water inhibits fermentation—use filtered water or let tap water stand uncovered for \textit{24~hours}
    \item \textbf{Yeast} acts as fermentation starter to ensure reliable results; traditional methods rely solely on wild yeasts but are less predictable
    \item \textbf{Baking~powder} added on cooking day provides additional lift and insurance for hole formation
\end{itemize}

\subsection*{Fermentation Monitoring}
\begin{itemize}
    \item \textbf{Healthy fermentation}: Pleasant sour smell (lactic acid, yogurt-like), active bubbling, thin clear liquid on top, batter thins over time
    \item \textbf{CONTAMINATION SIGNS—DISCARD IF OBSERVED}:
    \begin{itemize}
        \item Pink, orange, or red discoloration (bacterial contamination)
        \item Fuzzy growth on surface (mold—white, green, black, or gray)
        \item Putrid, rotten, or ammonia-like smell (spoilage bacteria)
        \item Slimy or ropy texture when stirred (certain bacteria produce polysaccharides)
        \item Complete separation with clear \textbf{water} on top and thick paste on bottom (failed fermentation)
    \end{itemize}
    \item Stir batter \textit{once daily} to redistribute microorganisms and prevent surface drying
    \item Temperature critical: below \textit{65°F} fermentation is very slow; above \textit{80°F} increases contamination risk
    \item If fermentation seems sluggish after \textit{48~hours}, move to warmer location or extend fermentation time
\end{itemize}

\subsection*{Preparation Tips}
\begin{itemize}
    \item Batter consistency is critical—too thick produces dense \textbf{injera} without holes; too thin creates fragile bread that tears
    \item Pour rapidly in continuous spiral motion—hesitation creates uneven thickness and poor hole distribution
    \item \textbf{Lid must fit tightly}—steam trapped under lid creates the characteristic holes. Without steam, surface remains flat
    \item First \textbf{injera} is often imperfect as pan temperature stabilizes—consider it a test piece
    \item Do not flip or cook second side—\textbf{injera} has distinct top (holey) and bottom (smooth) surfaces
    \item If holes fail to form: batter may need more fermentation time, \textbf{baking~powder} may be old, or lid may not seal properly
    \item Adjust heat between batches—pan should be hot enough for immediate bubbling but not so hot that bottom burns before top sets
    \item Stack only when completely cool to prevent steaming and sogginess
\end{itemize}

\subsection*{Make Ahead \& Storage}
\begin{itemize}
    \item \textbf{Batter} can be refrigerated after fermentation for up to \textit{24~hours}—bring to room temperature before cooking
    \item Cooked \textbf{injera} stores at room temperature for \textit{2-3~days} wrapped in clean towel, then placed in plastic bag
    \item Refrigerate for up to \textit{1~week}—layer with parchment paper and store in airtight container
    \item Freeze for up to \textit{3~months}—layer with parchment, wrap tightly, and thaw at room temperature
    \item Best served at room temperature or slightly warm
    \item \textbf{Injera} becomes more pliable and easier to tear after \textit{24~hours} as starches retrograde
\end{itemize}

\subsection*{Serving Suggestions}
\begin{itemize}
    \item Traditionally used as both plate and utensil—tear pieces with right hand to scoop stews and sauces
    \item Pairs excellently with Ethiopian wot (spiced stews), Moroccan tagines, or any rich, sauce-based dish
    \item The sour notes complement fatty meats (lamb, beef) and cut through richness of cream-based sauces
    \item Spongy texture absorbs sauces while maintaining structural integrity for scooping
    \item Serve at room temperature—cold \textbf{injera} is less pliable
    \item For traditional presentation, line large platter with \textbf{injera}, arrange stews on top, and serve additional \textbf{injera} rolled on the side
    \item Leftover \textbf{injera} can be torn and mixed with scrambled eggs for Ethiopian firfir breakfast dish
\end{itemize}

\end{multicols}
}

\end{document}
