
\documentclass[11pt,letterpaper]{article}
\usepackage{fontspec}
\usepackage{tocloft} % For dotted lines
\usepackage{multicol}
\usepackage[left=1.5in,right=1.5in,top=1in,bottom=1in]{geometry}
\setmainfont[Scale=1.2,AutoFakeBold=1.5,AutoFakeSlant=0.3]{Doves Type}

\title{Canned Cherries with Light Syrup}
\author{}
\date{}

\begin{document}

\maketitle
\thispagestyle{empty}

\section*{Ingredients}
\setlength{\columnsep}{20pt}
\begin{multicols}{2}
\noindent
    Cherries \dotfill 10\# \\
    Bottled lemon juice \dotfill 2½ cups \\
    Sugar \dotfill 5 cups \\
    Water \dotfill 25 cups \\
    Vinegar \dotfill ½ cup \\
    Canning jars/lids/bands (quart) \dotfill 7 \\
\end{multicols}

\section*{Instructions}
\begin{enumerate}
    \item \textbf{Prepare the equipment}: Sterilize 7 quart jars and an equal number of lids and bands in boiling water. Check the pressure canner for proper operation, including the seal and vent.
    \item \textbf{Prepare the cherries}: Wash and pit \textbf{10\# of cherries}. Prepare a solution with \textbf{2½ cups of bottled lemon juice} and \textbf{10+ cups of water}. Soak the cherries for \textbf{10 minutes} to help preserve their color and flavor.
    \item \textbf{Prepare the syrup}: Combine \textbf{5 cups of sugar} with \textbf{10 cups of water} in a large saucepan. Heat at \textbf{medium-high} until the sugar is completely dissolved, \textbf{stirring occasionally} to prevent sticking.
    \item \textbf{Pack the jars}: Evenly distribute the prepared cherries into the sterilized jars. Pour the \textbf{hot syrup} over the \textbf{cherries}, ensuring each jar is filled while leaving approximately \textbf{one inch of headspace}. Use a non-metallic spatula to gently stir inside the jars to \textbf{remove any trapped air bubbles}.
    \item \textbf{Place lids}: Wet a clean lint-free with \textbf{vinegar} to clean and dry the jar rims. Apply lids and rings, then tighten lightly with fingertips. (Always use new lids!)
    \item \textbf{Process in canner}: Place the filled jars on the rack inside the pressure canner. Add water as per the canner's instructions, usually around 2-3 inches. Secure the lid and heat until steam flows freely from the vent. Continue to \textbf{vent for 10 minutes}, then close the vent and attach the pressure regulator weight. Process the jars at \textbf{10-15 pounds of pressure} (adjusted for altitude) for \textbf{10 minutes}.
    \item \textbf{Cool down and store}: Turn off the heat and let the pressure canner \textbf{cool naturally} until the pressure gauge reads zero. Carefully remove the jars using a jar lifter and place them on a towel or cooling rack, avoiding drafty areas. After 12-24 hours, check that each jar is sealed by pressing the center of the lid; it should not flex up or down. Label the jars with the canning date and store in a cool, dark place.
\end{enumerate}

\end{document}
