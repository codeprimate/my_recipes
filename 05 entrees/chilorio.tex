\documentclass[11pt,letterpaper]{article}
\usepackage{fontspec}
\usepackage{tocloft}
\usepackage{multicol}
\usepackage[left=1.4in,right=1.5in,top=1in,bottom=1in]{geometry}
\setmainfont[Scale=1.4,AutoFakeBold=1.5,AutoFakeSlant=0.3]{Doves Type}

\title{Chilorio Tacos}
\author{}
\date{}

\begin{document}

\maketitle
\thispagestyle{empty}

\section*{Ingredients}
\setlength{\columnsep}{20pt}
\begin{multicols}{2}
\noindent
    Boneless pork shoulder \dotfill 3 lbs. \\
    Fresh orange juice \dotfill 2 cups \\
    Dried ancho chiles \dotfill 8 medium \\
    Garlic cloves \dotfill 8 \\
    Mexican oregano \dotfill 2 tsp. \\
    Black pepper, ground \dotfill 1 tsp. \\
    \columnbreak
    Ground cumin \dotfill ½ tsp. \\
    Cider vinegar \dotfill ¼ cup \\
    Kosher salt \dotfill 3 tsp. \\
    Lard \dotfill ¼ cup \\
    Corn or flour tortillas \dotfill 24 \\
    Tomatillo salsa \dotfill 1½ cups \\
    White onion, chopped \dotfill 1½ cups \\
    Fresh cilantro, chopped \dotfill 1½ cups \\
\end{multicols}

\section*{Directions}

\noindent
Preheat oven to \textit{325°F} ---
Trim \textbf{pork~shoulder} of about half the visible fat ---
Cut \textbf{pork} into 1-inch cubes ---
Stem and seed \textbf{ancho~chiles} ---
Tear \textbf{chiles} into large pieces ---
Peel and roughly chop \textbf{garlic} ---
Juice fresh \textbf{oranges} ---
Chop \textbf{onion} and \textbf{cilantro} for serving

\begin{enumerate}
    \item Heat a dry cast-iron skillet or comal over medium heat. Working in batches, toast \textbf{ancho~chile} pieces for \textit{15-30~seconds} per side until fragrant and slightly darkened. Be vigilant—burned \textbf{chiles} turn bitter. Transfer to a bowl.
    
    \item In a blender, combine \textbf{orange~juice}, toasted \textbf{ancho~chiles}, \textbf{garlic}, \textbf{oregano}, \textbf{black~pepper}, \textbf{cumin}, \textbf{vinegar}, and 3~tsp. \textbf{kosher~salt}. Blend on high until completely smooth, about \textit{1-2~minutes}. Strain through a medium-mesh sieve into a bowl, pressing solids to extract maximum flavor. Discard solids.
    
    \item Heat a large Dutch oven (at least 6-quart capacity) over medium-high heat. Working in batches to avoid crowding, sear \textbf{pork~cubes} on at least two sides until deeply golden brown, about \textit{3-4~minutes} per side. Remove to a plate. The rendered \textbf{pork~fat} left in the pot will contribute to the final dish—do not discard.
    
    \item Pour the strained \textbf{chile~mixture} into the Dutch oven. Return all \textbf{pork} to the pot, stirring to coat evenly. The liquid should come about halfway up the \textbf{pork}—add water if needed. Bring to a simmer over medium-high heat.
    
    \item Cover tightly with lid and transfer to \textit{325°F} oven. Braise for \textit{2½-3~hours}, checking occasionally, until \textbf{pork} is completely tender and easily pulls apart with a fork. Internal temperature should reach \textit{195-205°F}. The braising time allows tough shoulder fibers to break down into tender strands.
    
    \item Transfer \textbf{pork} to a large bowl using a slotted spoon. Let the \textbf{braising~liquid} settle for \textit{5~minutes}, then skim excess fat from the surface (reserve this fat—it's flavorful). If more than 1~cup of \textbf{braising~liquid} remains, return pot to stovetop and reduce over high heat to approximately 1~cup.
    
    \item Using two forks or your fingers, shred the \textbf{pork} into coarse pieces. Discard any large pieces of fat or connective tissue.
    
    \item Heat \textbf{reserved~fat} and/or \textbf{lard} in a very large (12-inch) skillet over medium-high heat. When shimmering hot, add the shredded \textbf{pork} in a single layer (work in batches if necessary). Let it sear undisturbed for \textit{2-3~minutes}, then stir and continue cooking until the \textbf{pork} develops brown, crispy edges, about \textit{3-4~minutes} total.
    
    \item Add 1~cup of the reduced \textbf{braising~liquid} to the skillet. Cook, stirring frequently, until the liquid reduces to a thick glaze that coats the \textbf{pork}, about \textit{5-7~minutes}. The \textbf{chilorio} should appear glossy and concentrated, not soupy.
    
    \item Taste and adjust seasoning with additional \textbf{salt} if needed—usually about ½~tsp. more. The finished \textbf{chilorio} should be intensely flavored, with pronounced smoky-sweet notes from the \textbf{chiles} and caramelized \textbf{pork} edges.
    
    \item Serve immediately with warm \textbf{tortillas}, \textbf{salsa}, and the \textbf{onion-cilantro} mixture. \textbf{Chilorio} is traditionally served in flour tortillas, though corn tortillas are equally appropriate.
\end{enumerate}

\newpage

% Begin compact two-column layout
{\small
\setlength{\columnsep}{20pt}
\setlength{\multicolsep}{6pt}
\begin{multicols}{2}
\setlength{\parindent}{0pt}
\setlength{\parskip}{4pt}

\subsection*{Equipment Required}
\begin{itemize}
    \item 6-quart (or larger) Dutch oven with tight-fitting lid
    \item Cast-iron skillet or comal for toasting chiles
    \item High-powered blender (at least 500 watts)
    \item Medium-mesh strainer or china cap
    \item Large (12-inch) skillet for finishing
    \item Slotted spoon or spider
    \item Two forks for shredding
    \item Large mixing bowl
    \item Sharp chef's knife
    \item Cutting board
    \item Kitchen tongs
    \item Instant-read thermometer
    \item Fat separator (optional, for braising liquid)
\end{itemize}

\subsection*{Mise en Place}
\begin{itemize}
    \item Trim \textbf{pork~shoulder} the day before and refrigerate overnight—this allows surface moisture to evaporate, promoting better searing
    \item Bring \textbf{pork} to room temperature \textit{30-45~minutes} before cooking
    \item Toast and blend \textbf{chile~mixture} up to \textit{24~hours} ahead; refrigerate until needed
    \item Prepare all garnishes before beginning the finishing step
    \item Have all equipment and ingredients measured and ready—the finishing stage moves quickly
\end{itemize}

\subsection*{Ingredient Tips}
\begin{itemize}
    \item \textbf{Pork~shoulder}: Select well-marbled meat with visible fat striations. Avoid pre-trimmed "lean" cuts—fat equals flavor and moisture. Bone-in shoulder works but requires longer cooking; boneless is easier to cube uniformly
    \item \textbf{Ancho~chiles}: Choose pliable, leathery \textbf{chiles} that bend without cracking. Avoid brittle, dusty specimens—they're stale and bitter. Fresh \textbf{anchos} have a slightly fruity aroma. Store in cool, dark pantry up to \textit{6~months}
    \item \textbf{Mexican~oregano}: Distinct from Mediterranean oregano—more citrusy and less minty. If unavailable, substitute half the amount of dried marjoram plus a pinch of dried thyme
    \item \textbf{Orange~juice}: Fresh-squeezed is vastly superior. The natural sugars caramelize during reduction, adding depth. Avoid juice with added sugar or preservatives
    \item \textbf{Lard}: Provides authentic flavor and high smoke point. Leaf lard (from kidney region) is highest quality. Vegetable oil works but lacks the richness traditional \textbf{chilorio} demands
    \item \textbf{Cumin}: Toast whole seeds in a dry pan, then grind fresh for maximum aromatics. Pre-ground \textbf{cumin} loses potency rapidly
\end{itemize}

\subsection*{Preparation Tips}
\begin{itemize}
    \item \textbf{Chile~toasting}: Watch constantly—\textbf{anchos} toast quickly. You'll smell a pronounced aroma when done. Dark spots are acceptable; black, acrid smoke means burned
    \item \textbf{Searing}: Don't crowd the pot. \textbf{Pork} releases moisture initially; give space for evaporation. The fond (browned bits) stuck to the pot adds critical flavor—the \textbf{braising~liquid} will deglaze it
    \item \textbf{Braising}: Resist the urge to check frequently. Each lid removal drops temperature \textit{25-30°F}, extending cooking time. Check once at \textit{2~hours}, then at \textit{30-minute} intervals
    \item \textbf{Shredding}: Don't over-shred. Coarse chunks hold texture better during the final sear. Aim for rustic, irregular pieces
    \item \textbf{Finishing}: High heat is essential. The \textbf{pork} should sizzle audibly. This Maillard browning develops the characteristic crispy edges and concentrated flavor
    \item \textbf{Glaze~consistency}: The \textbf{braising~liquid} should coat the back of a spoon. Too thin means insufficient reduction; too thick becomes sticky rather than glossy
\end{itemize}

\subsection*{Make Ahead \& Storage}
\begin{itemize}
    \item Complete through step 7 (braising and shredding) up to \textit{3~days} ahead. Refrigerate \textbf{shredded~pork} and \textbf{braising~liquid} separately
    \item For longer storage, freeze \textbf{shredded~pork} in \textbf{braising~liquid} up to \textit{3~months}. Thaw overnight in refrigerator
    \item Perform the final searing and glazing just before serving for optimal texture
    \item Leftover finished \textbf{chilorio} keeps \textit{5~days} refrigerated. Reheat gently in a skillet with a splash of water to restore glaze
    \item Do not freeze finished \textbf{chilorio}—the crispy texture degrades
    \item \textbf{Chile~adobo}: The strained sauce freezes beautifully for \textit{6~months}. Portion into ice cube trays for convenient use in other dishes
\end{itemize}

\subsection*{Serving Suggestions}
\begin{itemize}
    \item Traditional Sinaloan service: \textbf{Flour~tortillas}, \textbf{arbol-tomatillo~salsa}, diced \textbf{white~onion}, \textbf{cilantro}, and \textbf{lime} wedges
    \item Alternative applications: Burritos, tortas (Mexican sandwiches), quesadillas, tostadas, or served over rice with \textbf{refried~beans}
    \item Beverage pairing: Mexican lager, hibiscus agua fresca, or a crisp white wine (Albariño or unoaked Chardonnay) to balance the rich fat and chile heat
    \item Complementary sides: Pickled jalapeños, curtido (fermented cabbage slaw), or grilled scallions
    \item For breakfast: Serve with scrambled eggs, fried potatoes, and warm tortillas
    \item Garnish finished tacos with crumbled queso fresco or Cotija for salty contrast
    \item The rendered \textbf{pork~fat} skimmed from \textbf{braising~liquid} can be saved and used to fry tortillas for chilaquiles or to enrich \textbf{refried~beans}
\end{itemize}


\end{multicols}
}

\end{document}
