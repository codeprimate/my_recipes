\documentclass[11pt,letterpaper]{article}
\usepackage{fontspec}
\usepackage{tocloft} % For dotted lines
\usepackage{multicol}
\usepackage{nicefrac}
\usepackage[left=1in,right=1in,top=1in,bottom=1in]{geometry}
\setmainfont[Scale=1.2,AutoFakeBold=1.5,AutoFakeSlant=0.3]{Doves Type}

\widowpenalty=10000
\clubpenalty=10000
\interlinepenalty=500

\title{Chicken Broccoli Rice Casserole •}
\author{}
\date{}

\begin{document}

\maketitle
\thispagestyle{empty}

\begin{quote}
\textit{Rendered and pressure-cooked chicken is diced and combined with al-dente rice, blanched broccoli, and sautéed vegetables (corn caramelized in chicken fat, then onion, garlic, and mushrooms). A spiced roux-based cream sauce enriched with reserved chicken broth binds everything together, and the casserole is baked until bubbly and topped with crispy fried onions for texture. Serves 8--10.}
\end{quote}

\section*{Ingredients}
\setlength{\columnsep}{20pt}
\begin{multicols}{2}
\noindent
    Chicken thighs, bone-in, skin-on \dotfill 8 (3~lb.) \\
    Rice, long-grain white \dotfill 3 cups \\
    Water \dotfill 6 cups \\
    Chicken Better Than Bouillon \dotfill 4 Tbsp. \\
    Broccoli \dotfill 1 head \\
    Cream cheese \dotfill 8 oz. \\
    Onion, medium \dotfill 1 \\
    Garlic cloves \dotfill 6-8 \\
    Butter \dotfill 6~\nicefrac{1}{2} Tbsp. \\
    Flour \dotfill 3 Tbsp. \\
    Milk \dotfill 2 cups \\
    \columnbreak
    Mushroom slices, canned \dotfill 8 oz. \\
    Sweet corn, canned \dotfill 8 oz. \\
    Bay leaf \dotfill 2 \\
    Thyme, dried \dotfill 1 tsp. \\
    Rosemary, dried \dotfill 1 tsp. \\
    Sweet paprika \dotfill 1 tsp. \\
    MSG \dotfill \nicefrac{1}{4} tsp. \\
    Garlic powder \dotfill \nicefrac{1}{2} tsp. \\
    Onion powder \dotfill \nicefrac{1}{2} tsp. \\
    Nutmeg powder \dotfill pinch \\
    Salt \dotfill 1 tsp. \\
    Black pepper \dotfill \nicefrac{1}{2} tsp. \\
    Red pepper, crushed \dotfill \nicefrac{1}{4} tsp. \\
    Crispy fried onions \dotfill \nicefrac{1}{4} cup
\end{multicols}

\section*{Directions}

\noindent
Fill a large pot with salted water and start heating for blanching ---
Prepare \textbf{chicken~broth}: whisk 4~Tbsp. \textbf{Better~Than~Bouillon} into 6~cups \textbf{water} until dissolved; set aside in \textit{Medium Bowl~\#1} (broth) ---
Chop \textbf{broccoli}; set aside in \textit{Medium Bowl~\#2} ---
Drain \textbf{corn} and set aside; dice \textbf{onion} and set aside; mince \textbf{garlic} and drain \textbf{mushrooms}, combine these two in \textit{Small Bowl~\#1} (garlic and mushrooms for step 5) ---
Soften \textbf{cream cheese} at room temp and cube; set aside in \textit{Small Bowl~\#2} (cream cheese)

\begin{enumerate}
    \item Start cooking \textbf{rice} al-dente using 3~cups \textbf{chicken~broth} (\textit{Medium Bowl~\#1}) and a \textbf{bay leaf}. Rice is done when grains are separate (not mushy), tender but with a slight bite in the center, and most liquid has been absorbed but rice appears slightly wet. Stop your rice cooker \textit{10~minutes} early, or check manually. While rice cooks, proceed with steps below.

	\item In a deep skillet over medium heat, place \textbf{chicken} skin-side down. Render for \textit{6-8~minutes} until skin releases easily from pan and appears golden brown and crispy. Turn chicken and cook \textit{5~minutes} more until second side is lightly browned. Transfer \textbf{chicken} to Instant Pot. Pour off all but 2~Tbsp. of rendered fat from skillet and reserve skillet for step 5.
	
	\item Add 3~cups \textbf{chicken~broth} (\textit{Medium Bowl~\#1}), 1~tsp. \textbf{thyme}, and one \textbf{bay~leaf} to Instant Pot with \textbf{chicken}. Cook on \textbf{high pressure} for \textit{15~minutes}, then allow natural release for \textit{10~minutes} before manually releasing remaining pressure. Chicken is done when meat is tender and easily pulls away from bones, and internal temperature reaches \textit{165°F} on an instant-read thermometer. While chicken pressure cooks, proceed to step 5.
	
	\item Transfer cooked \textbf{chicken} to a plate and let rest for \textit{10-15~minutes} until cool enough to handle (meat should feel warm but not hot to the touch). Strain cooked broth through a fine-mesh strainer and reserve in \textit{Medium Bowl~\#4} (reserved broth); you should have approximately 2~\nicefrac{1}{2}--3~cups of reserved broth. While chicken rests, proceed to step 6 to make the sauce.
    
    \item While chicken pressure cooks (step 3), prepare vegetables using the reserved skillet with 2~Tbsp. rendered fat from step 2. Heat skillet over medium-high heat and add drained \textbf{corn}; caramelize for \textit{5~minutes} until kernels appear golden brown with some darker spots and smell sweet and nutty. Reduce heat to medium, add 2~Tbsp. \textbf{butter}, and add diced \textbf{onion}. Cook until translucent and softened, \textit{5-10~minutes}: onion should appear clear and glossy, edges should be slightly golden, and pieces should feel soft when pressed with a spatula. Add \textbf{garlic} and \textbf{mushrooms} (\textit{Small Bowl~\#1}), cooking for another \textit{5~minutes} until garlic is fragrant (aromatic but not browned) and mushrooms are tender and have released their liquid. Remove from heat and transfer to \textit{Large Bowl~\#1} (final mixture).
    
    \item Preheat oven to \textit{375°F}. While chicken cools (requires strained broth from step 4), make the sauce: In a large saucepan, melt 4~Tbsp. \textbf{butter} over medium heat until foaming subsides. Add 1~tsp. \textbf{rosemary}, 1~tsp. \textbf{paprika}, \nicefrac{1}{4}~tsp. \textbf{MSG}, \nicefrac{1}{2}~tsp. \textbf{garlic powder}, \nicefrac{1}{2}~tsp. \textbf{onion powder}, pinch \textbf{nutmeg}, 1~tsp. \textbf{salt}, \nicefrac{1}{2}~tsp. \textbf{black pepper}, and \nicefrac{1}{4}~tsp. \textbf{red pepper}. Stir constantly for \textit{30~seconds} until spices are fragrant and evenly distributed.
    
    \item Sprinkle 3~Tbsp. \textbf{flour} over the butter/spice mixture, whisking to combine until no dry flour remains. Cook, stirring constantly for \textit{2~minutes} until roux smells nutty and toasted (not burnt) and has a smooth, paste-like consistency with no raw flour taste. The spices will color the roux dark, so rely on time, smell, and texture rather than color. If roux begins to smell burnt, reduce heat immediately. If lumps form, continue whisking vigorously until smooth.
    
    \item Gradually whisk in 2~cups of \textbf{reserved~broth} (\textit{Medium Bowl~\#4}), then 2~cups \textbf{milk}, adding liquid in a steady stream while whisking constantly to prevent lumps. Heat and whisk constantly until sauce reaches a gentle boil and thickens noticeably, about \textit{5-8~minutes}. Sauce is done when it coats the back of a spoon thickly (a line drawn through it with your finger should hold clearly), appears smooth and creamy (not thin or watery), and bubbles gently throughout. If sauce doesn't thicken after \textit{8~minutes}, continue cooking and whisking; it may need an additional \textit{2-3~minutes}.
    
    \item Add cubes of \textbf{cream cheese} (\textit{Small Bowl~\#2}) to the hot sauce and whisk until melted and smooth with no visible chunks remaining. If cream cheese doesn't melt smoothly, remove from heat and continue whisking off heat until smooth. Remove from heat and transfer to \textit{Large Bowl~\#1} (final mixture).
    
    \item While sauce thickens (step 8), blanch \textbf{broccoli}: bring the pot of salted water to a rolling boil if not already boiling. Blanch \textbf{broccoli} (\textit{Medium Bowl~\#2}) for \textit{2~minutes} until bright green and crisp-tender: florets should appear vibrant green (not dull or olive-colored), stems should feel slightly softened when pierced with a knife but still have resistance, and broccoli should maintain its structure without becoming mushy. Immediately strain in a metal colander, rinsing with cold water to stop cooking. Drain well and transfer to \textit{Large Bowl~\#1} (final mixture).
    
    \item After sauce is complete, grease a 9~inch~×~13~inch baking dish with \nicefrac{1}{2}~Tbsp. \textbf{butter}. Debone and dice the rested \textbf{chicken}, discarding bones and skin. Place diced \textbf{chicken} in \textit{Large Bowl~\#1} (final mixture).
    
    \item Add cooked \textbf{rice} to \textit{Large Bowl~\#1} (final mixture) and mix all components until all ingredients are evenly distributed and coated with sauce. Add \nicefrac{1}{2}--1~cup of remaining \textbf{reserved~broth} (\textit{Medium Bowl~\#4}) to achieve a creamy, moist consistency (mixture should be cohesive but not soupy). Mixture should appear uniform with no dry spots. Taste and adjust salt and pepper if needed.
    
    \item Transfer mixture to prepared baking dish, spreading evenly. Bake at \textit{375°F} for \textit{20~minutes} until edges are bubbly and beginning to brown, and center is hot throughout.
    
    \item Remove from oven and sprinkle liberally with \nicefrac{1}{4}~cup \textbf{crispy fried onions}. Return to oven and bake another \textit{10~minutes} until top is golden brown, edges are bubbly and slightly crisp, and casserole is hot throughout (internal temperature should reach \textit{165°F} if checked). Continue baking in \textit{3~minute} increments if center is not hot or top is not golden. Rest \textit{5-10~minutes} before serving to allow casserole to set slightly.
\end{enumerate}

\newpage

{\footnotesize
\setlength{\columnsep}{20pt}
\setlength{\multicolsep}{6pt}
\begin{multicols}{2}
\setlength{\parindent}{0pt}
\setlength{\parskip}{2pt}
\setlength{\itemsep}{0pt}
\setlength{\parsep}{0pt}

\section*{Equipment Required}
\begin{itemize}
    \item Instant Pot or pressure cooker
    \item Rice cooker (or stovetop method)
    \item 9~inch~×~13~inch baking dish
    \item Large saucepan (for roux and sauce)
    \item Deep skillet (12-inch preferred, for rendering chicken and cooking vegetables)
    \item Large pot (for blanching broccoli)
    \item Metal colander
    \item Fine-mesh strainer
    \item Large prep bowl (1)
    \item Medium prep bowls (3)
    \item Small prep bowls (2)
    \item Instant-read thermometer
    \item Whisk
    \item Slotted spoon or tongs
    \item Measuring cups and spoons
    \item Cutting board and chef's knife
\end{itemize}

\section*{Hints and Notes}
\subsection*{Yield}
\begin{itemize}
    \item Serves 8-10 as main dish
    \item Makes one 9~inch~×~13~inch casserole
\end{itemize}

\subsection*{Mise en Place}
\begin{itemize}
    \item \textit{Medium Bowl~\#1} --- \textbf{chicken~broth}: 6~cups \textbf{water} with 4~Tbsp. \textbf{Better~Than~Bouillon} (about 6~cups total, 3~cups used in step 1 for rice, 3~cups used in step 3 for pressure cooking chicken)
    \item \textit{Medium Bowl~\#2} --- chopped \textbf{broccoli} (whole head, about 3-4~cups, blanched in step 10 during sauce thickening, then transferred to Large Bowl)
    \item Drained \textbf{corn} (8~oz., about 1~cup) measured and set aside (not in bowl) for step 5
    \item Diced \textbf{onion} (1 medium, about 1~cup) measured and set aside (not in bowl) for step 5
    \item \textit{Small Bowl~\#1} --- minced \textbf{garlic} (6-8 cloves) and drained \textbf{mushrooms} (8~oz., about \nicefrac{1}{2}~cup) — combined in prep, added together in step 5
    \item \textit{Small Bowl~\#2} --- cubed \textbf{cream cheese} (8~oz., softened at room temperature, about 1~cup, used in step 9)
    \item \textit{Medium Bowl~\#4} --- reserved \textbf{chicken~broth} (created in step 4, about 2~\nicefrac{1}{2}--3~cups total: 2~cups used in step 8 for sauce, \nicefrac{1}{2}--1~cup added to mixture in step 12)
    \item \textit{Large Bowl~\#1} --- final \textbf{mixture}: accumulates finished components as they're completed (cooked \textbf{vegetables} from step 5, \textbf{sauce} from step 9, blanched \textbf{broccoli} from step 10, diced \textbf{chicken} from step 11, cooked \textbf{rice} and additional \textbf{broth} added in step 12; about 14-16~cups total when complete)
    \item Soften \textbf{cream cheese} at room temperature \textit{30~minutes} before cooking
    \item Prep sequence: broth first, then vegetables (broccoli, corn, onion, garlic/mushrooms), then cream cheese
\end{itemize}

\subsection*{Ingredient Tips}
\begin{itemize}
    \item Use bone-in, skin-on \textbf{chicken thighs} for best flavor; rendering the skin adds richness, and bones contribute to the broth
    \item \textbf{Long-grain white rice} works best; avoid short-grain or sticky rice varieties
    \item Fresh \textbf{broccoli} with tight florets and firm stems is ideal; avoid yellowing or limp broccoli
    \item Canned \textbf{mushrooms} are convenient, but fresh sliced mushrooms can be substituted (sauté until tender before adding to vegetables)
    \item Quality \textbf{Better~Than~Bouillon} provides depth; adjust amount to taste preference
    \item Full-fat \textbf{cream cheese} provides best texture and flavor
    \item \textbf{Crispy fried onions} add essential texture; don't skip or substitute
\end{itemize}

\subsection*{Preparation Tips}
\begin{itemize}
    \item Rendering \textbf{chicken} skin-side down builds flavor and produces fat for caramelizing vegetables; don't rush this step
    \item Pressure cooking \textbf{chicken} after rendering produces tender meat and rich broth; natural release for \textit{10~minutes} prevents toughening
    \item Reserve the cooked \textbf{chicken~broth} carefully—it's used for the sauce and final mixture (approximately 2~\nicefrac{1}{2}--3~cups)
    \item Cook \textbf{rice} al-dente (slightly undercooked) as it will finish in the oven; fully cooked rice will become mushy
    \item Blanching \textbf{broccoli} maintains bright color and crisp-tender texture; don't overcook or it will become mushy in the final bake
    \item Use reserved chicken fat to caramelize \textbf{corn}; the rendered fat adds depth to the vegetables
    \item Building the \textbf{roux} slowly prevents burning; watch for light golden color and nutty aroma
    \item Whisk \textbf{sauce} constantly while adding liquid to prevent lumps; if lumps form, strain before proceeding
    \item \textbf{Cream cheese} should be at room temperature and cubed small for easy melting
    \item Add \nicefrac{1}{2}--1~cup of \textbf{reserved~broth} to the final mixture for creamy consistency; start with less and add more if needed
    \item Combine all components while warm for easier mixing and even distribution
    \item Add \textbf{crispy fried onions} near the end to preserve their crunch; they'll burn if added too early
    \item Rest casserole before serving to allow sauce to set slightly for cleaner slices
\end{itemize}

\subsection*{Make Ahead \& Storage}
\begin{itemize}
    \item Prep \textbf{chicken} and \textbf{broth} up to \textit{2~days} ahead; store separately in refrigerator
    \item Cook \textbf{rice} up to \textit{1~day} ahead; store covered in refrigerator
    \item Blanch \textbf{broccoli} up to \textit{1~day} ahead; store covered in refrigerator
    \item Prepare \textbf{vegetables} (corn, onion, garlic, mushrooms) up to \textit{1~day} ahead; store covered in refrigerator
    \item Make \textbf{sauce} up to \textit{2~days} ahead; store covered in refrigerator (may need to thin with additional milk when reheating)
    \item Assemble entire casserole up to \textit{1~day} ahead; cover and refrigerate, then add \textit{5~minutes} to initial bake time
    \item Leftovers keep \textit{3-4~days} refrigerated, covered tightly
    \item Reheat individual portions in microwave or reheat entire casserole at \textit{350°F} for \textit{20-25~minutes} until hot throughout
    \item Freezes well for up to \textit{3~months}; thaw in refrigerator overnight before reheating
\end{itemize}

\subsection*{Serving Suggestions}
\begin{itemize}
    \item Serve hot from the oven while \textbf{crispy fried onions} are still crunchy
    \item Excellent as a complete meal with a simple green salad
    \item Pairs well with roasted vegetables or steamed green beans
    \item Can be served as a side dish for larger gatherings (serves 12-14 as side)
    \item Leftovers make excellent packed lunches; reheat thoroughly
    \item Garnish with additional fresh herbs (parsley, chives) if desired
\end{itemize}

\end{multicols}
}

\end{document}
