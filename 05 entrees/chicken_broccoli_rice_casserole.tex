\documentclass[11pt,letterpaper]{article}
\usepackage{fontspec}
\usepackage{tocloft} % For dotted lines
\usepackage{multicol}
\usepackage{nicefrac}
\usepackage[left=1in,right=1in,top=1in,bottom=1in]{geometry}
\setmainfont[Scale=1.1,AutoFakeBold=1.5,AutoFakeSlant=0.3]{Doves Type}

\title{Chicken Broccoli Rice Casserole •}
\author{}
\date{}

\begin{document}

\maketitle
\thispagestyle{empty}

\begin{quote}
\textit{Pressure-cooked chicken thighs are diced and combined with al-dente rice, blanched broccoli, and sautéed vegetables (caramelized corn, onion, garlic, and mushrooms). A spiced roux-based cream sauce binds everything together, and the casserole is baked until bubbly and topped with crispy fried onions for texture.}
\end{quote}

\section*{Ingredients}
\setlength{\columnsep}{20pt}
\begin{multicols}{2}
\noindent
    Chicken thighs \dotfill 2-3lb. \\
    Rice, long-grain white \dotfill 2 cups \\
    Neutral oil \dotfill 2 tsp. \\
    Water \dotfill 4 cups \\
    Chicken Better Than Bouillon \dotfill 3 Tbsp. \\
    Broccoli \dotfill 1 head \\
    Cream cheese \dotfill 8 oz. \\
    Onion, medium \dotfill 1 \\
    Garlic cloves \dotfill 6-8 \\
    Butter \dotfill 6~\nicefrac{1}{2} Tbsp. \\
    Flour \dotfill 3 Tbsp. \\
    Milk \dotfill 2 cups \\
    \columnbreak
    Mushroom slices, canned \dotfill 8 oz. \\
    Sweet corn, canned \dotfill 8 oz. \\
    Bay leaf \dotfill 2 \\
    Thyme, dried \dotfill 1 tsp. \\
    Rosemary, dried \dotfill 1 tsp. \\
    Sweet paprika \dotfill 1 tsp. \\
    MSG \dotfill \nicefrac{1}{4} tsp. \\
    Garlic powder \dotfill \nicefrac{1}{2} tsp. \\
    Onion powder \dotfill \nicefrac{1}{2} tsp. \\
    Nutmeg powder \dotfill pinch \\
    Salt \dotfill 1 tsp. \\
    Black pepper \dotfill \nicefrac{1}{2} tsp. \\
    Red pepper, crushed \dotfill \nicefrac{1}{4} tsp. \\
    Crispy fried onions \dotfill \nicefrac{1}{4} cup
\end{multicols}

\section*{Directions}

\noindent
Preheat oven to \textit{375°F} ---
Prepare \textbf{chicken~broth}: whisk 3~Tbsp. \textbf{Better~Than~Bouillon} into 4~cups \textbf{water} until dissolved; set aside in \textit{Medium Bowl~\#1} (broth) ---
Chop \textbf{broccoli}; set aside in \textit{Medium Bowl~\#2} ---
Drain \textbf{corn}, dice \textbf{onion}, mince \textbf{garlic}, and drain \textbf{mushrooms}; combine in \textit{Medium Bowl~\#3} (vegetables for step 5) ---
Soften \textbf{cream cheese} at room temp and cube; set aside in \textit{Medium Bowl~\#4} (cream cheese) ---
Grease a 9~inch~×~13~inch baking dish with \nicefrac{1}{2}~Tbsp. \textbf{butter}

\begin{enumerate}
    \item Start cooking \textbf{rice} al-dente using 2~cups \textbf{chicken~broth} (\textit{Medium Bowl~\#1}) and a \textbf{bay leaf}. Rice is done when grains are tender but still have a slight bite in the center (not fully soft), and most liquid has been absorbed but rice appears slightly wet. Stop your rice cooker \textit{5-10~minutes} early, or check rice manually: grains should be separate, not mushy, and have a firm center when bitten. While rice cooks, proceed with steps 4 and 5.

	\item Place \textbf{chicken~thighs} in Instant Pot. Add 2~cups \textbf{chicken~broth} (\textit{Medium Bowl~\#1}), 1~tsp. \textbf{thyme}, and one \textbf{bay~leaf}. Cook on \textbf{high pressure} for \textit{15~minutes}, then allow natural release for \textit{5~minutes} before manually releasing remaining pressure. Chicken is done when meat is tender and easily pulls away from bones, and internal temperature reaches \textit{165°F} on an instant-read thermometer.
	
	\item Set cooked \textbf{chicken} aside on a plate; let rest for \textit{5-10~minutes} until cool enough to handle (meat should feel warm but not hot to the touch). Pick and dice meat, discarding bones and skin. Place diced \textbf{chicken} in \textit{Large Bowl~\#1} (final mixture). Strain cooked broth through a fine-mesh strainer and reserve in \textit{Medium Bowl~\#5} (reserved broth); you should have approximately 2~cups of reserved broth.
    
    \item While rice and chicken cooks, bring a pot of salted water to a rolling boil. Blanch \textbf{broccoli} (\textit{Medium Bowl~\#2}) for \textit{2~minutes} until bright green and crisp-tender: florets should appear vibrant green (not dull or olive-colored), stems should feel slightly softened when pierced with a knife but still have resistance, and broccoli should maintain its structure without becoming mushy. Immediately strain in a metal colander, rinsing with cold water to stop cooking. Drain well and transfer to \textit{Large Bowl~\#1} (final mixture).
    
    \item While rice continues cooking, prepare vegetables. In a large skillet, melt 4~Tbsp. \textbf{butter} on medium heat. Add 2~tsp. \textbf{oil} and \textbf{corn} from \textit{Medium Bowl~\#3} and increase heat to medium-high to brown and caramelize the corn, about \textit{5~minutes} until kernels appear golden brown with some darker spots and smell sweet and nutty. Add diced \textbf{onion} from \textit{Medium Bowl~\#3} and cook until translucent and softened, \textit{5-10~minutes}: onion should appear clear and glossy, edges should be slightly golden, and pieces should feel soft when pressed with a spatula. Add \textbf{garlic} and \textbf{mushrooms} from \textit{Medium Bowl~\#3}, cooking for another \textit{5~minutes} until garlic is fragrant (aromatic but not browned) and mushrooms are tender and have released their liquid. Remove from heat and transfer to \textit{Large Bowl~\#1} (final mixture).
    
    \item In a large saucepan, melt 2~Tbsp. \textbf{butter} over medium heat until foaming subsides. Add 1~tsp. \textbf{rosemary}, 1~tsp. \textbf{paprika}, \nicefrac{1}{4}~tsp. \textbf{MSG}, \nicefrac{1}{2}~tsp. \textbf{garlic powder}, \nicefrac{1}{2}~tsp. \textbf{onion powder}, pinch \textbf{nutmeg}, 1~tsp. \textbf{salt}, \nicefrac{1}{2}~tsp. \textbf{black pepper}, and \nicefrac{1}{4}~tsp. \textbf{red pepper}. Heat on medium, stirring constantly for \textit{30~seconds} until spices are fragrant and evenly distributed. Sprinkle 3~Tbsp. \textbf{flour} over the butter/spice mixture using a whisk to combine until no dry flour remains. Cook, stirring constantly for \textit{2~minutes} until roux appears light golden brown (blond roux), smells nutty, and has a smooth, paste-like consistency with no raw flour taste. If roux begins to darken too quickly or smell burnt, reduce heat immediately. If lumps form, continue whisking vigorously until smooth.
    
    \item Gradually whisk in 2~cups of \textbf{reserved~broth} (\textit{Medium Bowl~\#5}), then 2~cups \textbf{milk}, adding liquid in a steady stream while whisking constantly to prevent lumps. Heat and whisk constantly until sauce reaches a gentle boil and thickens noticeably, about \textit{5-8~minutes}. Sauce is done when it coats the back of a spoon thickly (a line drawn through it with your finger should hold clearly), appears smooth and creamy (not thin or watery), and bubbles gently throughout. If sauce doesn't thicken after \textit{8~minutes}, continue cooking and whisking; it may need an additional \textit{2-3~minutes}. If lumps form, strain sauce through a fine-mesh strainer before proceeding. Add cubes of \textbf{cream cheese} (\textit{Medium Bowl~\#4}) and whisk until melted and smooth with no visible chunks remaining. If cream cheese doesn't melt smoothly, remove from heat and continue whisking off heat until smooth. Remove from heat and transfer to \textit{Large Bowl~\#1} (final mixture).
    
    \item Add cooked \textbf{rice} to \textit{Large Bowl~\#1} (final mixture) and mix all components until all ingredients are evenly distributed and coated with sauce. Mixture should appear uniform with no dry spots.
    
    \item Transfer mixture to prepared baking dish, spreading evenly. Bake at \textit{375°F} for \textit{20~minutes} until edges are bubbly and beginning to brown, and center is hot throughout. Remove from oven to sprinkle liberally with \nicefrac{1}{4}~cup \textbf{crispy fried onions}, and bake another \textit{10~minutes} until top is golden brown, edges are bubbly and slightly crisp, and casserole is hot throughout (internal temperature should reach \textit{165°F} if checked). Continue baking in \textit{3~minute} increments if center is not hot or top is not golden. Rest \textit{5-10~minutes} before serving to allow casserole to set slightly.
\end{enumerate}

\newpage

{\footnotesize
\setlength{\columnsep}{20pt}
\setlength{\multicolsep}{6pt}
\begin{multicols}{2}
\setlength{\parindent}{0pt}
\setlength{\parskip}{2pt}
\setlength{\itemsep}{0pt}
\setlength{\parsep}{0pt}

\section*{Equipment Required}
\begin{itemize}
    \item Instant Pot or pressure cooker
    \item Rice cooker (or stovetop method)
    \item 9~inch~×~13~inch baking dish
    \item Large saucepan (for roux and sauce)
    \item Large skillet (12-inch preferred)
    \item Large pot (for blanching broccoli)
    \item Metal colander
    \item Fine-mesh strainer
    \item Large prep bowl (1)
    \item Medium prep bowls (5)
    \item Instant-read thermometer
    \item Whisk
    \item Slotted spoon
    \item Measuring cups and spoons
    \item Cutting board and chef's knife
\end{itemize}

\subsection*{Yield}
\begin{itemize}
    \item Serves 6-8 as main dish
    \item Makes one 9~inch~×~13~inch casserole
\end{itemize}

\subsection*{Mise en Place}
\begin{itemize}
    \item \textit{Large Bowl~\#1} --- final \textbf{mixture}: accumulates finished components as they're completed (diced \textbf{chicken} from step 2, blanched \textbf{broccoli} from step 4, cooked \textbf{vegetables} from step 5, \textbf{sauce} from step 7, cooked \textbf{rice} added in step 8; about 12-14~cups total when complete)
    \item \textit{Medium Bowl~\#1} --- \textbf{chicken~broth}: 4~cups \textbf{water} with 3~Tbsp. \textbf{Better~Than~Bouillon} (about 4~cups total, used in steps 1 and 3)
    \item \textit{Medium Bowl~\#2} --- chopped \textbf{broccoli} (whole head, about 3-4~cups, blanched in step 4 then transferred to Large Bowl)
    \item \textit{Medium Bowl~\#3} --- vegetables for step 5: drained \textbf{corn} (8~oz., about 1~cup), diced \textbf{onion} (1 medium, about 1~cup), minced \textbf{garlic} (6-8 cloves), and drained \textbf{mushrooms} (8~oz., about \nicefrac{1}{2}~cup) — combined in prep, added sequentially to skillet in step 5, then transferred to Large Bowl
    \item \textit{Medium Bowl~\#4} --- cubed \textbf{cream cheese} (8~oz., softened at room temperature, about 1~cup, used in step 7)
    \item \textit{Medium Bowl~\#5} --- reserved \textbf{chicken~broth} (created in step 2, about 2~cups, used in step 7)
    \item Soften \textbf{cream cheese} at room temperature \textit{30~minutes} before cooking
    \item Prep sequence: broth first, then vegetables (broccoli, corn, onion, garlic/mushrooms), then cream cheese
\end{itemize}

\subsection*{Ingredient Tips}
\begin{itemize}
    \item Use bone-in, skin-on \textbf{chicken~thighs} for best flavor; the bones and skin contribute to the rich broth
    \item \textbf{Long-grain white rice} works best; avoid short-grain or sticky rice varieties
    \item Fresh \textbf{broccoli} with tight florets and firm stems is ideal; avoid yellowing or limp broccoli
    \item Canned \textbf{mushrooms} are convenient, but fresh sliced mushrooms can be substituted (sauté until tender before adding to vegetables)
    \item Quality \textbf{Better~Than~Bouillon} provides depth; adjust amount to taste preference
    \item Full-fat \textbf{cream cheese} provides best texture and flavor
    \item \textbf{Crispy fried onions} add essential texture; don't skip or substitute
\end{itemize}

\subsection*{Preparation Tips}
\begin{itemize}
    \item Pressure cooking \textbf{chicken} produces tender meat and flavorful broth; don't skip the natural release period
    \item Reserve the cooked \textbf{chicken~broth} carefully—it's essential for the sauce and should yield approximately 2~cups
    \item Cook \textbf{rice} al-dente (slightly undercooked) as it will finish in the oven; fully cooked rice will become mushy
    \item Blanching \textbf{broccoli} maintains bright color and crisp-tender texture; don't overcook or it will become mushy in the final bake
    \item Caramelizing \textbf{corn} brings out natural sweetness; take time to develop golden brown color
    \item Building the \textbf{roux} slowly prevents burning; watch for light golden color and nutty aroma
    \item Whisk \textbf{sauce} constantly while adding liquid to prevent lumps; if lumps form, strain before proceeding
    \item \textbf{Cream cheese} should be at room temperature and cubed small for easy melting
    \item Combine all components while warm for easier mixing and even distribution
    \item Add \textbf{crispy fried onions} near the end to preserve their crunch; they'll burn if added too early
    \item Rest casserole before serving to allow sauce to set slightly for cleaner slices
\end{itemize}

\subsection*{Make Ahead \& Storage}
\begin{itemize}
    \item Prep \textbf{chicken} and \textbf{broth} up to \textit{2~days} ahead; store separately in refrigerator
    \item Cook \textbf{rice} up to \textit{1~day} ahead; store covered in refrigerator
    \item Blanch \textbf{broccoli} up to \textit{1~day} ahead; store covered in refrigerator
    \item Prepare \textbf{vegetables} (corn, onion, garlic, mushrooms) up to \textit{1~day} ahead; store covered in refrigerator
    \item Make \textbf{sauce} up to \textit{2~days} ahead; store covered in refrigerator (may need to thin with additional milk when reheating)
    \item Assemble entire casserole up to \textit{1~day} ahead; cover and refrigerate, then add \textit{5~minutes} to initial bake time
    \item Leftovers keep \textit{3-4~days} refrigerated, covered tightly
    \item Reheat individual portions in microwave or reheat entire casserole at \textit{350°F} for \textit{20-25~minutes} until hot throughout
    \item Freezes well for up to \textit{3~months}; thaw in refrigerator overnight before reheating
\end{itemize}

\subsection*{Serving Suggestions}
\begin{itemize}
    \item Serve hot from the oven while \textbf{crispy fried onions} are still crunchy
    \item Excellent as a complete meal with a simple green salad
    \item Pairs well with roasted vegetables or steamed green beans
    \item Can be served as a side dish for larger gatherings (serves 8-10 as side)
    \item Leftovers make excellent packed lunches; reheat thoroughly
    \item Garnish with additional fresh herbs (parsley, chives) if desired
\end{itemize}

\end{multicols}
}

\end{document}
