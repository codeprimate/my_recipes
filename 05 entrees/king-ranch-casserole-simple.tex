\documentclass[11pt,letterpaper]{article}
\usepackage{fontspec}
\usepackage{tocloft}
\usepackage{multicol}
\usepackage{fancyhdr}
\usepackage{nicefrac}
\usepackage[left=1.3in,right=1.3in,top=1in,bottom=1in]{geometry}
\setmainfont[Scale=1.3,AutoFakeBold=1.5,AutoFakeSlant=0.3]{Doves Type}

\title{King Ranch Casserole - Simple •}
\author{}
\date{}

% Configure fancy header
\pagestyle{fancy}
\fancyhf{} % Clear all header and footer fields
\fancyhead[C]{\textit{King Ranch Casserole - Simple}} % Center header with recipe title
\fancyfoot[C]{\thepage} % Center page number at bottom
\renewcommand{\headrulewidth}{0pt} % Remove header line
\renewcommand{\footrulewidth}{0pt} % Remove footer line

\begin{document}

\maketitle
\thispagestyle{empty}

\begin{quote}
\textit{Shredded rotisserie chicken is combined with a spiced sauce of caramelized corn, sautéed vegetables, cream soups, and Ro-Tel. The mixture is layered with quartered corn tortillas and sharp cheddar cheese, then baked until bubbly and golden.}
\end{quote}

\section*{Ingredients}
\setlength{\columnsep}{20pt}
\begin{multicols}{2}
\noindent
    Rotisserie chicken \dotfill 1 (about 3 lbs.) \\
    Cream of mushroom soup \dotfill 10 oz. can \\
    Cream of chicken soup \dotfill 10 oz. can \\
    Sweet corn, canned \dotfill 10 oz. can \\
    Black beans, canned \dotfill 15 oz. can \\
    Garlic cloves \dotfill 4-6 \\
	\columnbreak
    Onion powder \dotfill 1 tsp. \\
    MSG \dotfill \nicefrac{1}{2} tsp. \\
    Mexican spice mix \dotfill 2 Tbsp. \\
    Lard (or bacon fat) \dotfill 2 Tbsp. + 1 tsp. \\
    Onion \dotfill 1 large \\
    Bell pepper \dotfill 1 \\
    Ro-Tel \dotfill 10-14 oz. can \\
    Corn tortillas, medium \dotfill 16 \\
    Sharp cheddar cheese \dotfill 16 oz. \\
\end{multicols}

\section*{Directions}

\noindent
Preheat oven to \textit{375°F} ---
Drain and rinse \textbf{black beans}; set aside in \textit{Medium Bowl~\#2} ---
Drain \textbf{sweet corn}; set aside in \textit{Medium Bowl~\#1} ---
Combine \textbf{Mexican spice mix} and \textbf{MSG} in \textit{Small Bowl~\#1} (spice blend) ---
Shred \textbf{cheese}; set aside in \textit{Large Bowl~\#3} ---
Quarter 16 \textbf{corn tortillas}; set aside on a plate ---
Dice \textbf{onion} and \textbf{bell pepper}; mince \textbf{garlic}; combine in \textit{Medium Bowl~\#3} (aromatics) ---
Shred \textbf{rotisserie chicken} into \textit{Large Bowl~\#1}, discarding skin and bones ---
Grease a 3~quart baking dish with 1~tsp. \textbf{lard}

\begin{enumerate}
    \item In a large skillet, melt 2~Tbsp. \textbf{lard} over medium heat. Add \textbf{sweet corn} (\textit{Medium Bowl~\#1}) and sauté until well browned, about \textit{4-5~minutes}. Corn is done when kernels appear golden brown with darker spots, smell sweet and nutty, and have a slightly crisp texture when tasted.
    
    \item Add \textbf{onion}, \textbf{bell pepper}, and \textbf{garlic} (\textit{Medium Bowl~\#3}). Sauté over medium heat for \textit{8~minutes} until vegetables are softened and translucent: onion should appear clear and glossy, bell pepper should be tender, and garlic should be fragrant without browning. Raise heat to \textit{high} and cook undisturbed for \textit{1~minute}, then stir and cook \textit{1~minute} more until vegetables have dark caramelized spots on edges and smell sweet and nutty.
    
    \item Reduce heat to \textit{medium-low}. Add undrained \textbf{cream of mushroom soup}, undrained \textbf{cream of chicken soup}, undrained \textbf{Ro-Tel}, and \textbf{spice blend} (\textit{Small Bowl~\#1}). Stir to combine.
    
    \item Bring to a simmer, stirring occasionally, and cook for \textit{3~minutes} until well combined and heated through. Sauce is done when it appears uniform in color, bubbles gently throughout, and feels hot when a small amount is tested on the back of a spoon.
    
    \item Pour \textbf{vegetable and sauce mixture} from the skillet over the \textbf{shredded chicken} (\textit{Large Bowl~\#1}) and stir to coat evenly.
    
    \item Layer ingredients in the prepared baking dish:
    \begin{itemize}
        \item First layer: \nicefrac{1}{3}~of the \textbf{tortilla quarters}, \nicefrac{1}{2}~of the \textbf{chicken mixture}, \nicefrac{1}{3}~of the \textbf{shredded cheese} (\textit{Large Bowl~\#3})
        \item Second layer: \nicefrac{1}{3}~of the \textbf{tortilla quarters}, remaining \textbf{chicken mixture}, \nicefrac{1}{3}~of the \textbf{shredded cheese} (\textit{Large Bowl~\#3})
        \item Final layer: remaining \textbf{tortilla quarters}, remaining \textbf{shredded cheese} (\textit{Large Bowl~\#3})
    \end{itemize}
    
    \item Bake uncovered at \textit{375°F} for \textit{30-35~minutes} until bubbly and cheese is melted. Casserole is done when edges are bubbly and beginning to brown, cheese is fully melted and golden brown on top, and center is hot throughout (internal temperature should reach \textit{165°F} if checked). Continue baking in \textit{3~minute} increments if center is not hot or cheese is not golden.
    
    \item Let stand for \textit{10~minutes} before serving.
\end{enumerate}

\newpage

{\footnotesize
\setlength{\columnsep}{20pt}
\setlength{\multicolsep}{6pt}
\begin{multicols}{2}
\setlength{\parindent}{0pt}
\setlength{\parskip}{2pt}
\setlength{\itemsep}{0pt}
\setlength{\parsep}{0pt}

\section*{Equipment Required}
\begin{itemize}
    \item 3~quart baking dish
    \item Large skillet (12-inch preferred)
    \item Large prep bowls (2)
    \item Medium prep bowls (3)
    \item Small prep bowl (1)
    \item Measuring cups and spoons
    \item Mixing spoon or spatula
    \item Instant-read thermometer (optional but recommended)
    \item Cutting board and chef's knife
\end{itemize}

\subsection*{Yield}
\begin{itemize}
    \item Serves 6-8 as main dish
    \item Makes one 3~quart casserole
\end{itemize}

\subsection*{Mise en Place}
\begin{itemize}
    \item \textit{Large Bowl~\#1} --- shredded \textbf{rotisserie chicken} (about 3~lbs., discarding skin and bones; about 4-5~cups shredded)
    \item \textit{Large Bowl~\#3} --- shredded \textbf{sharp cheddar cheese} (16~oz., about 4~cups)
    \item \textit{Medium Bowl~\#1} --- drained \textbf{sweet corn} (10~oz. can, about 1~\nicefrac{1}{4}~cups)
    \item \textit{Medium Bowl~\#2} --- drained and rinsed \textbf{black beans} (15~oz. can, about 1~\nicefrac{1}{2}~cups; note: not used in this simple version)
    \item \textit{Medium Bowl~\#3} --- aromatics: diced \textbf{onion} (1 large, about 1~\nicefrac{1}{2}~cups), diced \textbf{bell pepper} (1 pepper, about \nicefrac{3}{4}~cup), minced \textbf{garlic} (4-6 cloves)
    \item \textit{Small Bowl~\#1} --- spice blend: 2~Tbsp. \textbf{Mexican spice mix}, \nicefrac{1}{2}~tsp. \textbf{MSG}
    \item 16 quartered \textbf{corn tortillas} set aside on a plate
    \item Prep sequence: drain and rinse beans, drain corn, combine spices, shred cheese, quarter tortillas, dice aromatics, shred chicken, grease baking dish
\end{itemize}

\subsection*{Ingredient Tips}
\begin{itemize}
    \item Use a quality \textbf{rotisserie chicken} for best flavor; remove all skin and bones carefully
    \item \textbf{Corn tortillas} should be fresh and pliable; stale tortillas may become too brittle when quartered
    \item \textbf{Sharp cheddar cheese} provides the best flavor; pre-shredded works but freshly shredded melts more evenly
    \item \textbf{Mexican spice mix} can be store-bought or homemade; adjust quantity to taste preference
    \item \textbf{Ro-Tel} adds heat and acidity; use mild or original depending on heat preference
    \item \textbf{Lard} or \textbf{bacon fat} adds authentic flavor; butter can be substituted if needed
\end{itemize}

\subsection*{Preparation Tips}
\begin{itemize}
    \item Caramelizing \textbf{corn} properly is key—take time to develop golden brown color and nutty aroma
    \item Don't rush the vegetable sauté—allowing vegetables to soften before high-heat caramelization prevents burning
    \item The high-heat finish on vegetables creates dark caramelized spots that add depth of flavor
    \item Stir \textbf{sauce} mixture well to fully incorporate cream soups and Ro-Tel
    \item Layer \textbf{tortillas} evenly to ensure even distribution throughout the casserole
    \item Don't overbake—cheese should be golden and bubbly, not browned or crispy
    \item Letting casserole stand before serving allows it to set slightly for cleaner slices
    \item If casserole appears too dry, the sauce may need more liquid; if too wet, bake longer
\end{itemize}

\subsection*{Make Ahead \& Storage}
\begin{itemize}
    \item Shred \textbf{chicken} up to \textit{2~days} ahead; store covered in refrigerator
    \item Prepare \textbf{vegetables} (corn, onion, bell pepper, garlic) up to \textit{1~day} ahead; store covered in refrigerator
    \item Make \textbf{sauce mixture} up to \textit{1~day} ahead; store covered in refrigerator
    \item Assemble entire casserole up to \textit{1~day} ahead; cover and refrigerate, then add \textit{5~minutes} to bake time
    \item Leftovers keep \textit{3-4~days} refrigerated, covered tightly
    \item Reheat individual portions in microwave or reheat entire casserole at \textit{350°F} for \textit{20-25~minutes} until hot throughout
    \item Freezes well for up to \textit{3~months}; thaw in refrigerator overnight before reheating
\end{itemize}

\subsection*{Serving Suggestions}
\begin{itemize}
    \item Serve hot from the oven while cheese is still bubbly
    \item Excellent as a complete meal with a simple green salad
    \item Pairs well with Mexican rice or refried beans
    \item Garnish with fresh cilantro, diced tomatoes, or sliced jalapeños if desired
    \item Serve with sour cream or guacamole on the side
    \item Leftovers make excellent packed lunches; reheat thoroughly
\end{itemize}

\end{multicols}
}

\end{document}

