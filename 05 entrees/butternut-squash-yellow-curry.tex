\documentclass[11pt,letterpaper]{article}
\usepackage{fontspec}
\usepackage{tocloft}
\usepackage{multicol}
\usepackage{fancyhdr}
\usepackage{nicefrac}
\usepackage[left=1.3in,right=1.3in,top=1in,bottom=1in]{geometry}
\setmainfont[Scale=1.2,AutoFakeBold=1.5,AutoFakeSlant=0.3]{Doves Type}

% Prevent page breaks within list items
\widowpenalty=10000
\clubpenalty=10000
\interlinepenalty=500

\title{Butternut Squash Yellow Curry}
\author{}
\date{}

% Configure fancy header
\pagestyle{fancy}
\fancyhf{} % Clear all header and footer fields
\fancyhead[C]{\textit{Butternut Squash Yellow Curry}} % Center header with recipe title
\fancyfoot[C]{\thepage} % Center page number at bottom
\renewcommand{\headrulewidth}{0pt} % Remove header line
\renewcommand{\footrulewidth}{0pt} % Remove footer line

\begin{document}

\maketitle
\thispagestyle{empty}


\section*{Ingredients}

\setlength{\columnsep}{20pt}
\begin{multicols}{2}
\noindent
    Neutral oil \dotfill 3 Tbsp. \\
    Boneless skinless chicken thighs \dotfill 2~lb. \\
    Butternut squash \dotfill 3~lb. \\
    Yellow onion, medium \dotfill 1 \\
    Garlic cloves \dotfill 4 \\
    Fresh ginger, minced \dotfill 1 Tbsp. \\
    \columnbreak
    Mae Ploy yellow curry paste \dotfill 3--4 Tbsp. \\
    Coconut milk \dotfill 2 (13.5~oz.) cans \\
    Chicken stock \dotfill 1\nicefrac{1}{2}~cups \\
    Palm or brown sugar \dotfill 1--2 Tbsp. \\
    Fish sauce \dotfill 1 Tbsp. \\
    Limes \dotfill 2 \\
    Fresh cilantro \dotfill for garnish \\
    Salt \dotfill to taste \\
\end{multicols}

\section*{Directions}

\noindent
Cut 2~lb. \textbf{chicken~thighs} into 1--1\nicefrac{1}{2}" chunks ---
Peel, seed, and cut \textbf{butternut~squash} into 1" chunks; set aside in \textit{Large Bowl~\#1} ---
Dice 1 medium \textbf{onion}; mince 4 \textbf{garlic~cloves} and 1~Tbsp. \textbf{ginger}; combine in \textit{Small Bowl~\#1} (aromatics) ---
Open 2 cans \textbf{coconut~milk}; have 1\nicefrac{1}{2}~cups \textbf{chicken~stock}, \textbf{curry~paste}, \textbf{sugar}, and \textbf{fish~sauce} ready ---
Juice 2 \textbf{limes} and cut into wedges; chop \textbf{cilantro} for garnish

\begin{enumerate}
    \item Heat 1\nicefrac{1}{2}~Tbsp. \textbf{neutral~oil} in a large Dutch oven over medium-high heat. Brown \textbf{chicken~thighs} in two batches, turning as needed, until golden on multiple sides and no longer pink on the surface, about \textit{5--7~minutes} per batch. Thighs are done when they release easily from the pan and show golden browning. Transfer \textbf{chicken~thighs} to \textit{Large Bowl~\#2} and set aside.

    \item Add remaining 1\nicefrac{1}{2}~Tbsp. \textbf{oil} to the pot. Brown \textbf{squash} (\textit{Large Bowl~\#1}) in two batches over medium-high heat, stirring occasionally, until edges are golden and surfaces take on color, about \textit{4--6~minutes} per batch. Squash should not be fully tender. Transfer to \textit{Large Bowl~\#1} and set aside.

    \item Reduce heat to medium. Add \textbf{onion}, \textbf{garlic}, and \textbf{ginger} (\textit{Small Bowl~\#1}) and cook, stirring, until \textbf{onion} is translucent and aromatics are fragrant, about \textit{3--4~minutes}. Do not brown.

    \item Add 3--4~Tbsp. \textbf{Mae~Ploy~yellow~curry~paste} and cook, stirring constantly, until the paste darkens slightly and smells toasted and fragrant, about \textit{2--3~minutes}.

    \item Pour in about half of the first can of \textbf{coconut~milk} and stir to combine with the paste until smooth. Add remaining \textbf{coconut~milk} from both cans, 1\nicefrac{1}{2}~cups \textbf{chicken~stock}, 1--2~Tbsp. \textbf{sugar}, and 1~Tbsp. \textbf{fish~sauce}. Stir and bring to a simmer.

    \item Add \textbf{squash} (\textit{Large Bowl~\#1}) and return to a gentle simmer. Cook uncovered, stirring occasionally, for \textit{18--22~minutes} until \textbf{squash} is tender: a paring knife inserted into the center of a chunk meets no resistance, and chunks hold their shape but are soft. Sauce should coat the back of a spoon. Continue simmering in \textit{2--3~minute} increments until \textbf{squash} is tender.

    \item Return \textbf{chicken~thighs} and any accumulated juices (\textit{Large Bowl~\#2}) to the pot. Simmer for \textit{5--8~minutes} until \textbf{chicken~thighs} reach \textit{165°F} internally and \textbf{squash} is fully tender. Thighs are done when no pink remains and juices run clear.

    \item Taste and adjust seasoning with \textbf{salt} if needed. Stir in \textbf{lime~juice} to taste. Serve over steamed jasmine rice, garnished with \textbf{cilantro} and \textbf{lime~wedges}.
\end{enumerate}

\newpage

{\footnotesize
\setlength{\columnsep}{20pt}
\setlength{\multicolsep}{6pt}
\begin{multicols}{2}
\setlength{\parindent}{0pt}
\setlength{\parskip}{2pt}
\setlength{\itemsep}{0pt}
\setlength{\parsep}{0pt}

\subsection*{Yield}
\begin{itemize}
    \item Serves 8
\end{itemize}

\subsection*{Equipment Required}
\begin{itemize}
    \item Large Dutch oven or heavy pot (6--7~quart capacity)
    \item Cutting board and chef's knife
    \item Measuring cups and spoons
    \item Wooden spoon or silicone spatula
    \item Instant-read thermometer (for chicken)
    \item Small prep bowls (1)
    \item Large prep bowls (2)
\end{itemize}

\subsection*{Mise en Place}
\begin{itemize}
    \item Small Bowl \#1 --- aromatics: diced \textbf{onion}, minced \textbf{garlic}, minced \textbf{ginger}
    \item Large Bowl \#1 --- \textbf{butternut~squash} chunks (about 8--10~cups); after browning, return to this bowl
    \item Large Bowl \#2 --- browned \textbf{chicken~thighs} (set aside until final step)
    \item Have \textbf{coconut~milk} opened, \textbf{chicken~stock} measured, and \textbf{curry~paste}, \textbf{sugar}, \textbf{fish~sauce} ready before building the curry
\end{itemize}

\subsection*{Ingredient Tips}
\begin{itemize}
    \item Mae Ploy \textbf{yellow~curry~paste} is recommended; other brands may differ in salt and heat---taste and adjust
    \item \textbf{Fish~sauce} is used lightly (1~Tbsp.) for subtle umami without dominating
    \item Full-fat \textbf{coconut~milk} gives the best body and flavor
    \item \textbf{Palm~sugar} is traditional; brown sugar works well
\end{itemize}

\subsection*{Preparation Tips}
\begin{itemize}
    \item Browning \textbf{chicken} and \textbf{squash} in sequence builds fond and sweetness; do not skip
    \item Bloom the \textbf{curry~paste} until it darkens and smells toasted---raw paste tastes flat
    \item Uniform 1" \textbf{squash} chunks cook evenly; larger chunks need a few more minutes
    \item Adding \textbf{chicken} only at the end keeps it from overcooking and keeps \textbf{squash} as the focus
    \item Taste before serving; \textbf{lime~juice} and \textbf{salt} balance the curry
\end{itemize}

\subsection*{Make Ahead \& Storage}
\begin{itemize}
    \item Curry can be made \textit{1--2~days} ahead; flavors improve
    \item Store refrigerated up to \textit{4~days}
    \item Reheat gently; thin with \textbf{chicken~stock} or \textbf{coconut~milk} if sauce has thickened
    \item Add fresh \textbf{lime~juice} and \textbf{cilantro} after reheating
\end{itemize}

\subsection*{Serving Suggestions}
\begin{itemize}
    \item Serve over steamed jasmine rice
    \item \textbf{Lime~wedges} and extra \textbf{cilantro} at the table
    \item Thai basil or sliced fresh chile for optional heat and aroma
\end{itemize}

\end{multicols}
}

\end{document}
