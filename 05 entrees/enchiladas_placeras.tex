\documentclass[11pt,letterpaper]{article}
\usepackage{fontspec}
\usepackage{tocloft}
\usepackage{multicol}
\usepackage{fancyhdr}
\usepackage{nicefrac}
\usepackage[left=1.3in,right=1.3in,top=1in,bottom=1in]{geometry}
\setmainfont[Scale=1.3,AutoFakeBold=1.5,AutoFakeSlant=0.3]{Doves Type}

\title{Enchiladas Placeras}
\author{}
\date{}

% Configure fancy header
\pagestyle{fancy}
\fancyhf{} % Clear all header and footer fields
\fancyhead[C]{\textit{Enchiladas Placeras}} % Center header with recipe title
\fancyfoot[C]{\thepage} % Center page number at bottom
\renewcommand{\headrulewidth}{0pt} % Remove header line
\renewcommand{\footrulewidth}{0pt} % Remove footer line

\begin{document}

\maketitle
\thispagestyle{empty}

\begin{quote}
\textit{Michoacán-Style Market Enchiladas} --- Traditional plaza-style enchiladas from Morelia, featuring corn tortillas dipped in guajillo chile sauce, lightly fried, filled with cheese and chicken, then topped with fried potatoes, carrots, and fresh garnishes. This is authentic street food at its finest.
\end{quote}

\section*{Ingredients}
\setlength{\columnsep}{20pt}

\begin{multicols}{2}
\noindent
Bone-in chicken thighs \dotfill 3~lb. \\
White onion, medium \dotfill 2 \\
Garlic cloves \dotfill 6 (divided) \\
Bay leaf \dotfill 1 \\
Salt \dotfill 1~tsp. \\
Chicken stock \dotfill 6 cups \\
Dried guajillo chiles \dotfill 16 \\
Dried ancho chiles \dotfill 4 \\
Chiles de árbol (optional) \dotfill 2--4 \\
Reserved chicken broth \dotfill 4--5~cups \\
Ground cumin \dotfill 2~tsp. \\
Mexican oregano \dotfill 2~tsp. \\
White vinegar \dotfill 2~Tbsp. \\
\columnbreak
\noindent
Salt \dotfill 2~tsp. \\
Vegetable oil \dotfill 3~Tbsp. \\
Yukon gold potatoes \dotfill 1~lb. \\
Carrots \dotfill \nicefrac{1}{2}~lb. \\
Salt \dotfill to taste \\
Vegetable oil \dotfill 3~Tbsp. \\
Corn tortillas (GF) \dotfill 12 \\
White onion, finely diced \dotfill \nicefrac{1}{2}~cup \\
Queso fresco, crumbled \dotfill 8~oz. \\
Vegetable oil \dotfill 1~cup \\
Romaine lettuce or cabbage \dotfill 1~small head \\
Mexican crema OR sour cream \dotfill \nicefrac{1}{2}~cup \\
Pickled jalapeños \dotfill \nicefrac{1}{4}~cup \\
Queso fresco \dotfill \nicefrac{1}{4}~cup \\
\end{multicols}

\section*{Directions}

\noindent
Halve 1 medium \textbf{onion} ---
Peel and dice 1~lb. \textbf{potatoes} into \nicefrac{1}{2}-inch cubes; set aside in \textit{Medium Bowl~\#1} ---
Peel and dice \nicefrac{1}{2}~lb. \textbf{carrots} into \nicefrac{1}{2}-inch cubes; add to \textit{Medium Bowl~\#1} ---
Quarter \textbf{onions} --- 
Finely dice \nicefrac{1}{2}~cup \textbf{white onion}; set aside in \textit{Small Bowl~\#1} ---
Crumble 6~oz. \textbf{queso fresco} into \textit{Small Bowl~\#1} with diced onion (filling) ---
Reserve 2~oz. \textbf{queso fresco} in \textit{Small Bowl~\#2} (topping) ---
Finely shred \textbf{lettuce} or \textbf{cabbage}; set aside in \textit{Medium Bowl~\#2}

\newpage

\begin{enumerate}
    \item Place 3~lb. \textbf{chicken thighs} in a large pot with a quartered \textbf{onion}, 4~\textbf{garlic cloves}, 1~\textbf{bay leaf}, and 1~tsp. \textbf{salt}. Add enough \textbf{water} to cover by 1~inch (approximately 6--8~cups). Bring to a boil over high heat, then reduce to a gentle simmer. Skim any foam that rises to the surface.
    
    \item Simmer for \textit{35--40~minutes} until \textbf{chicken} is fully cooked and tender. Remove \textbf{chicken} to a plate and let cool. Strain the broth through a fine-mesh sieve into a large bowl, discarding solids. Reserve approximately 5~cups of broth (you'll need 4~cups for the sauce).
    
    \item Once \textbf{chicken} is cool enough to handle, remove and discard skin and bones. Shred meat with two forks into bite-sized pieces. Set aside in \textit{Large Bowl~\#1}.
    
    \item Using kitchen scissors, cut open each \textbf{guajillo chile}, \textbf{ancho chile}, and \textbf{chile de árbol} (if using) lengthwise. Shake out and discard most seeds. Remove stems.
    
    \item Heat a dry cast-iron skillet or comal over medium heat. Toast the \textbf{chiles} in batches, pressing flat with a spatula, about \textit{10--15~seconds per side} until fragrant and slightly darkened. Do not burn.
    
    \item Transfer toasted \textbf{chiles} to a large heatproof bowl. Bring 4~cups of the reserved \textbf{chicken broth} to a boil and pour over the \textbf{chiles}. Weight \textbf{chiles} down with a small plate to keep submerged. Let soak for \textit{15--20~minutes} until completely softened.
    
    \item Working in two batches, transfer half of the softened \textbf{chiles} to a blender along with 1\nicefrac{1}{2}~cups of the soaking liquid, half of the remaining \textbf{onion} (raw), 1~\textbf{garlic clove}, 1~tsp. \textbf{cumin}, 1~tsp. \textbf{oregano}, \nicefrac{1}{2}~Tbsp. \textbf{vinegar}, and 1~tsp. \textbf{salt}. Blend on high for \textit{2--3~minutes} until completely smooth. Transfer to a large bowl. Repeat with remaining \textbf{chiles} and aromatics. Combine both batches. Add additional soaking liquid or reserved broth if needed to reach a pourable consistency like heavy cream.
    
    \item Strain the sauce through a fine-mesh sieve into a bowl, pressing on solids with a spatula to extract maximum liquid. Discard solids.
    
    \item Heat 3~Tbsp. \textbf{vegetable oil} in a large skillet or saucepan over medium-high heat until shimmering. Carefully pour in all of the strained sauce (it will sputter and steam). Reduce heat to medium and cook, stirring frequently, for \textit{10--12~minutes} until sauce darkens slightly, thickens to coat the back of a spoon, and loses its raw flavor. Taste and adjust salt. Transfer approximately 2~cups to a wide, shallow bowl for dipping tortillas. Reserve remaining sauce in a separate bowl for drizzling. Keep both warm.
    
    \item Bring a large pot of salted water to a boil. Add diced \textbf{potatoes} and \textbf{carrots} (\textit{Medium Bowl~\#1}). Boil for \textit{5--8~minutes} until just tender but still firm (a knife should pierce with slight resistance). Drain well and pat dry with paper towels.
    
    \item Heat 3~Tbsp. \textbf{vegetable oil} in a large skillet over medium-high heat. Add the parboiled \textbf{potatoes} and \textbf{carrots}. Fry, stirring occasionally, for \textit{5--7~minutes} until golden brown on edges. Add 3~Tbsp. of the prepared \textbf{guajillo sauce} and toss to coat. Season with \textbf{salt} to taste. Remove from heat and keep warm in \textit{Medium Bowl~\#3}.
    
    \item Heat \nicefrac{1}{4}~cup \textbf{vegetable oil} in a large skillet over medium-high heat. Add the shredded \textbf{chicken} (\textit{Large Bowl~\#1}) in a single layer. Fry without stirring for \textit{3--4~minutes} until bottom is golden and crispy. Flip and fry another \textit{2--3~minutes}. Season lightly with \textbf{salt}. Transfer to \textit{Large Bowl~\#2} and keep warm.
    
    \item Set up assembly station: Wide shallow bowl with warm \textbf{guajillo sauce} for dipping, large skillet with \nicefrac{1}{4}~inch \textbf{vegetable oil} heated to \textit{350°F} (medium heat), stack of \textbf{corn tortillas}, bowl with \textbf{cheese-onion filling} (\textit{Small Bowl~\#1}), platter for finished enchiladas.
    
    \item Working one at a time: Dip a \textbf{tortilla} completely in the \textbf{sauce}, coating both sides (about 2--3~seconds total). Let excess drip off briefly.
    
    \item Immediately place the sauce-coated \textbf{tortilla} in the hot oil. Fry for \textit{10--15~seconds per side} --- just until the sauce sets and edges firm slightly but tortilla remains pliable. Remove with tongs to a plate.
    
    \item While still hot and pliable, place 2--3~Tbsp. of the \textbf{cheese-onion mixture} (\textit{Small Bowl~\#1}) and a small handful of \textbf{fried chicken} (\textit{Large Bowl~\#2}) down the center of the tortilla. Roll tightly and place seam-side down on serving plate. Repeat with remaining tortillas, adding more oil to skillet as needed between batches.
    
    \item Arrange 3 enchiladas per plate. Spoon \textbf{fried potatoes and carrots} (\textit{Medium Bowl~\#3}) generously over the top of the enchiladas.
    
    \item Optional: Drizzle additional warm \textbf{guajillo sauce} over the enchiladas before adding final toppings.
    
    \item Drizzle with \textbf{Mexican crema}, scatter shredded \textbf{lettuce} or \textbf{cabbage} (\textit{Medium Bowl~\#2}) over the top, and finish with crumbled \textbf{queso fresco} (\textit{Small Bowl~\#2}) and \textbf{pickled jalapeños} to taste. Serve immediately.
\end{enumerate}

\newpage

{\footnotesize
\setlength{\columnsep}{20pt}
\setlength{\multicolsep}{6pt}
\begin{multicols}{2}
\setlength{\parindent}{0pt}
\setlength{\parskip}{2pt}
\setlength{\itemsep}{0pt}
\setlength{\parsep}{0pt}

\subsection*{Equipment Required}
\begin{itemize}
    \item Large stockpot (for poaching chicken)
    \item Large cast-iron skillet or comal
    \item 2--3 large skillets (for vegetables, chicken, and frying tortillas)
    \item High-powered blender
    \item Fine-mesh strainer
    \item Kitchen scissors
    \item Wide shallow bowl (for dipping tortillas)
    \item Tongs
    \item Instant-read thermometer
    \item Small prep bowls (2)
    \item Medium prep bowls (3)
    \item Large prep bowls (2)
    \item Sharp knife
    \item Measuring cups and spoons
\end{itemize}

\subsection*{Mise en Place}
\begin{itemize}
    \item Small Bowl \#1 --- filling: \nicefrac{1}{2}~cup finely diced \textbf{white onion} mixed with 6~oz. crumbled \textbf{queso fresco}
    \item Small Bowl \#2 --- 2~oz. crumbled \textbf{queso fresco} (topping)
    \item Medium Bowl \#1 --- raw diced \textbf{potatoes} and \textbf{carrots} (about 3~cups total)
    \item Medium Bowl \#2 --- shredded \textbf{lettuce} or \textbf{cabbage}
    \item Medium Bowl \#3 --- fried \textbf{potatoes} and \textbf{carrots} with sauce (after step 11)
    \item Large Bowl \#1 --- shredded poached \textbf{chicken} (after step 3, about 3--4~cups)
    \item Large Bowl \#2 --- fried \textbf{chicken} (after step 12)
\end{itemize}

\subsection*{Ingredient Tips}
\begin{itemize}
    \item \textbf{Chicken}: Bone-in thighs provide the richest broth; drumsticks also work well
    \item \textbf{Guajillo chiles}: Should be pliable, not brittle; look for glossy, deep red color
    \item \textbf{Ancho chiles}: Add depth and mild fruity notes; darker and wrinkled compared to guajillos
    \item \textbf{Chiles de árbol}: Optional for heat; add 1--2 to sauce if you want spicier enchiladas
    \item \textbf{Corn tortillas}: Certified gluten-free if needed; slightly day-old tortillas absorb less oil
    \item \textbf{Queso fresco}: Look for authentic Mexican queso fresco; feta can substitute in a pinch
    \item \textbf{Mexican oregano}: Has different flavor than Mediterranean oregano; worth seeking out
    \item \textbf{Potatoes}: Yukon gold hold their shape better than russets when fried
\end{itemize}

\subsection*{Preparation Tips}
\begin{itemize}
    \item Poaching \textbf{chicken} with aromatics creates flavorful broth that becomes the sauce base
    \item Reserve extra \textbf{chicken broth} for thinning sauce if needed
    \item Toast \textbf{chiles} briefly --- burnt chiles create bitter sauce
    \item Blending in two batches prevents overloading blender and ensures smooth consistency
    \item Strain sauce well to remove tough chile skins for smooth texture
    \item Frying the blended sauce concentrates flavor and cooks out raw taste
    \item Recipe makes approximately 5--6~cups sauce; use 3~cups for dipping tortillas, reserve remainder for drizzling
    \item Parboil \textbf{potatoes} and \textbf{carrots} before frying to ensure they're cooked through
    \item Adding sauce to fried vegetables colors them and adds flavor
    \item Frying shredded \textbf{chicken} adds texture contrast and caramelized flavor
    \item Work quickly during tortilla assembly --- they must stay hot to roll without breaking
    \item Keep oil temperature steady at \textit{350°F} --- too hot burns sauce, too cool makes greasy enchiladas
    \item Tortilla should be pliable after frying, not crispy
    \item Assemble enchiladas immediately before serving for best texture
    \item Drizzling extra sauce over enchiladas before final toppings is optional but traditional
\end{itemize}

\subsection*{Make Ahead \& Storage}
\begin{itemize}
    \item \textbf{Chicken} can be poached and shredded \textit{1~day} ahead; store broth and chicken separately
    \item \textbf{Guajillo sauce} can be made \textit{1--2~days} ahead; refrigerate and reheat gently before using
    \item \textbf{Vegetables} can be parboiled \textit{4~hours} ahead; fry just before serving
    \item Assembly must be done immediately before serving --- tortillas become soggy if held
    \item Leftovers keep \textit{2--3~days} refrigerated but texture will soften
    \item Reheat individual portions at \textit{350°F} for \textit{10--12~minutes}
    \item Not recommended for freezing --- corn tortillas become mealy
\end{itemize}

\subsection*{Serving Suggestions}
\begin{itemize}
    \item Traditional serving is 3 enchiladas per person as a main course
    \item Serve immediately while enchiladas are hot and vegetables are crispy
    \item Provide extra \textbf{pickled jalapeños} and \textbf{crema} at the table for customization
    \item Can serve with refried beans on the side, though not traditional
    \item \textbf{Lime wedges} make an excellent garnish for brightness
    \item Presentation matters: layer components in proper order for visual appeal
\end{itemize}

\subsection*{Heat Level Options}
\begin{itemize}
    \item Mild: Use only \textbf{guajillo} and \textbf{ancho chiles} as written (remove all seeds)
    \item Medium: Add 2--3 \textbf{chiles de árbol} to the sauce blend
    \item Spicy: Add 4--6 \textbf{chiles de árbol} and include some seeds when blending
    \item Very Spicy: Add 8--10 \textbf{chiles de árbol} with seeds
    \item Heat comes from chiles in sauce, not the toppings
    \item \textbf{Chiles de árbol} add clean heat without changing the fundamental flavor profile
\end{itemize}

\end{multicols}
}

\end{document}
