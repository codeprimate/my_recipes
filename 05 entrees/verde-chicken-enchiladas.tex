\documentclass[11pt,letterpaper]{article}
\usepackage{fontspec}
\usepackage{tocloft}
\usepackage{multicol}
\usepackage{fancyhdr}
\usepackage{nicefrac}
\usepackage[left=1.3in,right=1.3in,top=1in,bottom=1in]{geometry}
\setmainfont[Scale=1.3,AutoFakeBold=1.5,AutoFakeSlant=0.3]{Doves Type}

\title{Verde Chicken Enchiladas}
\author{}
\date{}

% Configure fancy header
\pagestyle{fancy}
\fancyhf{} % Clear all header and footer fields
\fancyhead[C]{\textit{Verde Chicken Enchiladas}} % Center header with recipe title
\fancyfoot[C]{\thepage} % Center page number at bottom
\renewcommand{\headrulewidth}{0pt} % Remove header line
\renewcommand{\footrulewidth}{0pt} % Remove footer line

\begin{document}

\maketitle
\thispagestyle{empty}

\begin{quote}
\textit{Pressure-cooked chicken thighs are shredded and combined with Monterey Jack cheese, then rolled in corn tortillas that have been dipped in a vibrant verde sauce. The sauce is built from roasted jalapeños, onion, and garlic blended with tomatillos, then simmered with reserved chicken broth until concentrated. The enchiladas are topped with more sauce and a combination of Monterey Jack and queso fresco, then baked until bubbly and golden.}
\end{quote}

\section*{Ingredients}
\setlength{\columnsep}{20pt}

\begin{multicols}{2}
\noindent
Chicken thighs, bone-in \dotfill 2\nicefrac{1}{2}~lb. \\
Mexican spice blend \dotfill 2~Tbsp. \\
Water \dotfill 2~cups \\
Better Than Bouillon, chicken \dotfill 1~Tbsp. \\
Tomatillos, canned \dotfill 28~oz. \\
Jalapeños, medium \dotfill 2 (about 2~Tbsp. minced) \\
White onion, medium \dotfill 1 \\
Garlic cloves \dotfill 4 \\
Fresh cilantro, chopped \dotfill \nicefrac{1}{2}~cup \\
Lime juice \dotfill 2~Tbsp. \\
Ground cumin \dotfill 1~tsp. \\
Mexican oregano \dotfill 1~tsp. \\
\columnbreak
\noindent
Salt \dotfill 1\nicefrac{1}{2}~tsp. (divided) \\
Black pepper \dotfill \nicefrac{1}{2}~tsp. \\
Vegetable oil \dotfill 3~Tbsp. (divided) \\
Corn tortillas (6-inch) \dotfill 12--14 \\
Monterey Jack cheese, shredded \dotfill 12~oz. \\
Queso fresco, crumbled \dotfill 4~oz. \\
Reserved chicken broth \dotfill 1\nicefrac{1}{2}~cups \\
Chicken broth (if needed) \dotfill \nicefrac{1}{2}~cup \\
\end{multicols}

\section*{Directions}

\noindent
Preheat oven to \textit{400°F} ---
Grease a 9~inch×13~inch baking dish ---
Drain \textbf{tomatillos}, reserving liquid; set aside in \textit{Medium Bowl~\#1} ---
Halve \textbf{jalapeños} lengthwise and remove seeds and ribs; set aside on a small plate ---
Quarter \textbf{onion}; set aside on plate with jalapeños ---
Peel \textbf{garlic cloves}; set aside on plate ---
Chop \textbf{cilantro}; set aside in \textit{Small Bowl~\#1} ---
Juice \textbf{lime}; set aside in \textit{Small Bowl~\#2} ---
Shred \textbf{Monterey Jack cheese}; set aside in \textit{Large Bowl~\#1} ---
Crumble \textbf{queso fresco}; set aside in \textit{Small Bowl~\#3} ---
Combine 1~tsp. \textbf{cumin} and 1~tsp. \textbf{oregano} in \textit{Small Bowl~\#4} (sauce spices)

\begin{enumerate}
    \item Prepare \textbf{chicken broth}: Combine 2~cups \textbf{water} and 1~Tbsp. \textbf{Better Than Bouillon} in a measuring cup, stirring until dissolved.
    
    \item Place 2\nicefrac{1}{2}~lb. \textbf{chicken thighs} in Instant Pot and coat thoroughly with 2~Tbsp. \textbf{Mexican spice blend}, rubbing it in well. Add prepared \textbf{broth} to the pot. Seal Instant Pot and cook on \textit{high} pressure for \textit{15~minutes}, then allow natural release for \textit{5~minutes}.
    
    \item While \textbf{chicken} cooks, roast the aromatics: Arrange \textbf{jalapeños} (cut-side down), \textbf{onion} quarters, and \textbf{garlic cloves} on a rimmed baking sheet. Drizzle with 1~Tbsp. \textbf{vegetable oil} and toss to coat. Roast at \textit{400°F} for \textit{15-20~minutes} until \textbf{jalapeños} are blistered and slightly charred, \textbf{onion} is softened and browned at edges, and \textbf{garlic} is golden and tender. \textbf{Jalapeños} are done when skin is blistered and peppers feel soft when pressed. Remove from oven and let cool slightly.
    
    \item When Instant Pot is ready, quick release remaining pressure. Transfer \textbf{chicken} to a plate and let rest for \textit{5-10~minutes} until cool enough to handle. Strain the \textbf{cooking liquid} through a fine-mesh sieve and measure out 1\nicefrac{1}{2}~cups; reserve in \textit{Medium Bowl~\#2} for the sauce. Discard any excess liquid.
    
    \item Discard skin and bones from \textbf{chicken}, then shred into bite-sized pieces; set aside in \textit{Large Bowl~\#2}.
    
    \item Transfer roasted \textbf{jalapeños}, \textbf{onion}, and \textbf{garlic} to a blender. Add drained \textbf{tomatillos} (\textit{Medium Bowl~\#1}), 1~tsp. \textbf{cumin} and 1~tsp. \textbf{oregano} (\textit{Small Bowl~\#4}), 1~tsp. \textbf{salt}, and \nicefrac{1}{2}~tsp. \textbf{black pepper}. Blend on high for \textit{1-2~minutes} until completely smooth.
    
    \item Add \nicefrac{1}{4}~cup of reserved \textbf{tomatillo liquid} or \textbf{chicken broth} if needed to reach a smooth, pourable consistency. Blend briefly to incorporate.
    
    \item Heat 2~Tbsp. \textbf{vegetable oil} in a large saucepan over medium-high heat until shimmering. Carefully pour in the blended \textbf{sauce} (it will sputter). Reduce heat to medium and cook, stirring frequently, for \textit{15-20~minutes} until sauce darkens slightly, thickens to coat the back of a spoon, and loses its raw flavor. The sauce is done when it has reduced by about one-third, appears darker green, and coats a spoon without running off immediately.
    
    \item Stir in 1\nicefrac{1}{2}~cups reserved \textbf{chicken broth} (\textit{Medium Bowl~\#2}) and simmer for \textit{2-3~minutes} more to integrate. Remove from heat and stir in \textbf{cilantro} (\textit{Small Bowl~\#1}) and \textbf{lime juice} (\textit{Small Bowl~\#2}). Taste and adjust \textbf{salt} (add remaining \nicefrac{1}{2}~tsp. if needed).
    
    \item Transfer approximately 2~cups of warm \textbf{sauce} to a wide, shallow bowl for dipping tortillas; keep remaining sauce warm in the saucepan.
    
    \item Reduce oven temperature to \textit{350°F}.
    
    \item Warm \textbf{corn tortillas} in the microwave wrapped in a damp paper towel for \textit{30~seconds}, or heat briefly on a griddle until pliable. Working one at a time, dip each \textbf{tortilla} completely in the warm \textbf{sauce} from the bowl, coating both sides (about 2-3~seconds total). Let excess drip off briefly.
    
    \item Place sauce-coated \textbf{tortilla} on a plate. Spoon 2-3~Tbsp. shredded \textbf{chicken} (\textit{Large Bowl~\#2}) down the center, then sprinkle with 1~Tbsp. \textbf{Monterey Jack cheese} (\textit{Large Bowl~\#1}). Roll tightly and place seam-side down in the prepared baking dish. Repeat with remaining \textbf{tortillas}, arranging them snugly in the pan.
    
    \item Pour remaining warm \textbf{sauce} from the saucepan evenly over the \textbf{enchiladas}, covering them completely. Sprinkle remaining \textbf{Monterey Jack cheese} (\textit{Large Bowl~\#1}) evenly over the top, then scatter \textbf{queso fresco} (\textit{Small Bowl~\#3}) over the cheese.
    
    \item Bake at \textit{350°F} for \textit{25-30~minutes} until \textbf{cheese} is melted and bubbly, sauce is bubbling around edges, and \textbf{enchiladas} are heated through. \textbf{Enchiladas} are done when \textbf{cheese} is golden brown in spots, sauce is actively bubbling, and center of pan feels hot when tested with a knife. Continue baking in \textit{2~minute} increments if needed.
    
    \item Let rest for \textit{5~minutes} before serving. Garnish with additional \textbf{cilantro} if desired. Serve hot.
\end{enumerate}

\newpage

{\footnotesize
\setlength{\columnsep}{20pt}
\setlength{\multicolsep}{6pt}
\begin{multicols}{2}
\setlength{\parindent}{0pt}
\setlength{\parskip}{2pt}
\setlength{\itemsep}{0pt}
\setlength{\parsep}{0pt}

\subsection*{Yield}
\begin{itemize}
    \item Serves 4-6 as main course
    \item Makes 12-14 enchiladas in a 9~inch×13~inch baking dish
\end{itemize}

\subsection*{Equipment Required}
\begin{itemize}
    \item Instant Pot (6-quart or larger)
    \item Rimmed baking sheet
    \item Large saucepan (3-4 quart)
    \item High-powered blender
    \item Fine-mesh strainer
    \item 9~inch×13~inch baking dish
    \item Wide shallow bowl (for dipping tortillas)
    \item Small prep bowls (4)
    \item Medium prep bowls (2)
    \item Large prep bowls (2)
    \item Tongs or slotted spoon
    \item Measuring cups and spoons
\end{itemize}

\subsection*{Mise en Place}
\begin{itemize}
    \item Small Bowl \#1 --- chopped \textbf{cilantro} (\nicefrac{1}{2}~cup)
    \item Small Bowl \#2 --- \textbf{lime juice} (2~Tbsp.)
    \item Small Bowl \#3 --- crumbled \textbf{queso fresco} (4~oz.)
    \item Small Bowl \#4 --- sauce spices: 1~tsp. \textbf{cumin}, 1~tsp. \textbf{oregano}
    \item Medium Bowl \#1 --- drained \textbf{tomatillos} (reserve liquid)
    \item Medium Bowl \#2 --- reserved \textbf{chicken broth} (1\nicefrac{1}{2}~cups, after step 4)
    \item Large Bowl \#1 --- shredded \textbf{Monterey Jack cheese} (12~oz.)
    \item Large Bowl \#2 --- shredded \textbf{chicken} (after step 5, about 2\nicefrac{1}{2}~cups)
    \item Prepare \textbf{chicken broth} with Better Than Bouillon before starting
    \item Roast aromatics while \textbf{chicken} cooks in Instant Pot
\end{itemize}

\subsection*{Ingredient Tips}
\begin{itemize}
    \item \textbf{Chicken}: Bone-in thighs provide richer broth and more flavor; boneless work but reduce broth amount
    \item \textbf{Mexican spice blend}: Your prepared blend adds depth; adjust amount based on heat preference
    \item \textbf{Tomatillos, canned}: Look for firm tomatillos in water (not heavy brine); 28~oz. is standard can size
    \item \textbf{Jalapeños}: 2 medium jalapeños with seeds removed yields mild-medium heat; add seeds for more heat
    \item \textbf{Monterey Jack}: Melts beautifully and provides creamy texture; can substitute with Colby Jack
    \item \textbf{Queso fresco}: Adds texture contrast and mild saltiness; feta can substitute but is saltier
    \item \textbf{Corn tortillas}: Slightly day-old tortillas absorb sauce better; warm thoroughly before dipping
    \item \textbf{Cilantro}: Add at the end to preserve bright, fresh flavor
    \item \textbf{Lime juice}: Fresh is essential for brightness; bottled lacks complexity
\end{itemize}

\subsection*{Preparation Tips}
\begin{itemize}
    \item Roasting \textbf{jalapeños}, \textbf{onion}, and \textbf{garlic} creates depth and mellows raw flavors; blistered skin on peppers indicates proper roasting
    \item Blending sauce until completely smooth ensures even texture; no chunks should remain
    \item Frying the blended sauce concentrates flavor and cooks out raw taste; sauce darkens and thickens as it reduces
    \item Simmering with \textbf{chicken broth} integrates flavors and adds body; sauce should coat a spoon without being too thick
    \item Adding \textbf{cilantro} and \textbf{lime juice} at the end preserves brightness; adding too early causes them to lose vibrancy
    \item Dipping \textbf{tortillas} in warm sauce makes them pliable and prevents cracking during rolling
    \item Don't overfill \textbf{enchiladas}---2-3~Tbsp. \textbf{chicken} per tortilla prevents bursting
    \item Arranging \textbf{enchiladas} snugly in the pan prevents them from unrolling during baking
    \item Sauce should cover \textbf{enchiladas} completely but not pool excessively; adjust amount if needed
    \item Resting after baking allows sauce to set slightly and makes cutting easier
    \item Recipe makes approximately 4-5~cups sauce; use 2~cups for dipping, remainder for topping
\end{itemize}

\subsection*{Make Ahead \& Storage}
\begin{itemize}
    \item \textbf{Chicken} can be cooked and shredded \textit{1~day} ahead; store separately from broth
    \item \textbf{Verde sauce} can be made \textit{1-2~days} ahead; refrigerate and reheat gently before using
    \item Add \textbf{cilantro} and \textbf{lime juice} to sauce just before using if made ahead
    \item Assembled \textbf{enchiladas} can be refrigerated \textit{4~hours} before baking; add \textit{5~minutes} to baking time
    \item Leftovers keep \textit{3-4~days} refrigerated; reheat individual portions at \textit{350°F} for \textit{10-15~minutes}
    \item Sauce can be frozen for \textit{3~months}; thaw and reheat gently, adding fresh \textbf{cilantro} and \textbf{lime juice}
    \item \textbf{Enchiladas} don't freeze well---corn tortillas become mealy when frozen and thawed
\end{itemize}

\subsection*{Serving Suggestions}
\begin{itemize}
    \item Serve immediately while \textbf{cheese} is bubbly and \textbf{enchiladas} are hot
    \item Traditional serving is 2-3 \textbf{enchiladas} per person as a main course
    \item Pair with \textbf{Mexican rice} and \textbf{refried beans} for a complete meal
    \item Garnish with additional \textbf{cilantro}, \textbf{lime wedges}, and \textbf{diced white onion}
    \item \textbf{Sour cream} or \textbf{Mexican crema} adds richness and cools heat
    \item \textbf{Avocado} slices or \textbf{guacamole} complement the tangy sauce
    \item \textbf{Pickled jalapeños} add heat and acidity
    \item Serve with a crisp green salad with lime vinaigrette for contrast
\end{itemize}

\subsection*{Heat Level Options}
\begin{itemize}
    \item Mild: Use 2 \textbf{jalapeños} with all seeds and ribs removed (as written)
    \item Medium: Use 2 \textbf{jalapeños} with seeds from 1 pepper included
    \item Medium-Hot: Use 2 \textbf{jalapeños} with all seeds included, or add 1 \textbf{serrano pepper}
    \item Hot: Use 3 \textbf{jalapeños} with seeds, or substitute 2 \textbf{serrano peppers}
    \item Heat comes primarily from \textbf{jalapeños}; roasting mellows heat slightly
    \item Taste sauce after blending and add more heat if desired before simmering
\end{itemize}

\end{multicols}
}

\end{document}
