\documentclass[11pt,letterpaper]{article}
\usepackage{fontspec}
\usepackage{tocloft}
\usepackage{multicol}
\usepackage{nicefrac}
\usepackage[left=1.4in,right=1.5in,top=1in,bottom=1in]{geometry}
\setmainfont[Scale=1.2,AutoFakeBold=1.5,AutoFakeSlant=0.3]{Doves Type}

\title{Hong Shao Rou •}
\author{}
\date{}

\begin{document}

\maketitle
\thispagestyle{empty}

\section*{Ingredients}
\setlength{\columnsep}{20pt}
\begin{multicols}{2}
\noindent
    Skin-on pork belly \dotfill 2~lbs. \\
    Rock or granulated sugar \dotfill 5~Tbsp. \\
    Shaoxing wine \dotfill 3~Tbsp. \\
    Light soy sauce \dotfill 2~Tbsp. \\
    Dark soy sauce \dotfill 1~Tbsp. \\
    Ginger, fresh \dotfill 1~inch piece \\
    Green onions \dotfill 6 \\
    Star anise \dotfill 2 \\
    \columnbreak
    Cinnamon stick \dotfill 1 \\
    Bay leaves \dotfill 1 \\
    Dried red chilis \dotfill 3 \\
    Tangerine \dotfill 1 \\
    Garlic cloves \dotfill 4 \\
    Daikon radish \dotfill 1~lb. \\
    Water or stock \dotfill 1-2 cups \\
    Jasmine rice \dotfill 3 cups \\
    Baby bok choy \dotfill 2~lbs. \\
    Sesame seeds, toasted \dotfill 2~Tbsp. 
\end{multicols}

\section*{Directions}

\noindent
Cook rice ---
Cut \textbf{pork belly} into 1\nicefrac{1}{2}--2~inch cubes ---
Measure 5~Tbsp. \textbf{granulated sugar} and set aside ---
Measure 3~Tbsp. \textbf{Shaoxing wine} and set aside ---
Peel and slice \textbf{ginger} (1~inch piece) into \nicefrac{1}{2}~inch thick slices and smash ---
Cut \textbf{green onions} into 2--inch pieces (white parts) for braising; thinly slice green parts for garnish and set aside in \textit{Small Bowl~\#2} ---
Smash \textbf{garlic} cloves lightly ---
Peel and slice \textbf{daikon radish} into \nicefrac{1}{2}~inch thick rounds; set aside in \textit{Medium Bowl~\#1} ---
Trim \textbf{baby bok choy}: remove any damaged outer leaves, trim the base, and quarter lengthwise through the core; set aside in \textit{Medium Bowl~\#2} ---
Toast \textbf{sesame seeds} in a dry pan over medium heat until golden and fragrant, approximately \textit{2--3~minutes}; set aside in \textit{Small Bowl~\#2} with sliced green onion tops
---
Peel \textbf{tangerine} and cut the peel into \nicefrac{1}{2}~inch wide strips, avoiding the white pith as much as possible ---
Combine \textbf{ginger}, \textbf{green onion whites}, \textbf{garlic}, 2~Tbsp. \textbf{light soy sauce}, 1~Tbsp. \textbf{dark soy sauce}, 2 \textbf{star anise}, 1 \textbf{cinnamon stick}, 1 \textbf{bay leaf}, 3 \textbf{dried red chilis}, and \textbf{tangerine peel} strips in \textit{Small Bowl~\#1} 

\begin{enumerate}
    \item Heat a heavy pot or Dutch oven over medium heat. Add \textbf{pork belly} cubes (no oil needed; fat will render). Sear until golden on all sides, approximately \textit{5--7~minutes}, turning pieces as needed. The fat should render and pool in the pot, and the meat should appear golden brown, not deeply browned. Remove \textbf{pork belly} and transfer to \textit{Large Bowl~\#1}; set aside. Reserve 2~Tbsp. rendered fat in the pot.
    
    \item Reduce heat to low. Add measured \textbf{granulated sugar} to the rendered fat in the pot. Stir constantly as the sugar melts and caramelizes. The sugar will transition from clear liquid to golden, then to amber, and finally to deep red-brown, approximately \textit{3--5~minutes}. Stop immediately when the caramel reaches deep amber/red-brown color; do not allow it to turn black or smoke. If the caramel darkens too quickly, remove the pot from heat briefly---residual heat will continue cooking. The caramel should appear glossy and deep red-brown, not blackened.
    
    \item Return \textbf{pork belly} (\textit{Large Bowl~\#1}) to the pot with the caramel. Toss to coat each piece evenly with the caramelized sugar. The pork should appear uniformly coated with the glossy caramel.
    
    \item Add measured \textbf{Shaoxing wine} and allow it to bubble for \textit{30~seconds}, scraping up any caramelized bits from the bottom of the pot. This deglazes the pot and adds depth of flavor.
    
    \item Add all contents from \textit{Small Bowl~\#1} (\textbf{ginger}, \textbf{green onions}, \textbf{garlic}, \textbf{light soy sauce}, \textbf{dark soy sauce}, \textbf{star anise}, \textbf{cinnamon stick}, \textbf{bay leaf}, \textbf{dried red chilis}, \textbf{tangerine peel}).
    
    \item Add \textbf{daikon radish} (\textit{Medium Bowl~\#1}) to the pot, then add hot \textbf{water or stock} to just cover the \textbf{pork belly} and \textbf{daikon}. Bring to a boil, then reduce heat to a low simmer. Cover and braise for \textit{1\nicefrac{1}{2}--2~hours} until the pork is very tender and the \textbf{daikon} is tender and translucent. Check at \textit{1~hour}: the pork should be yielding but not falling apart, and the \textbf{daikon} should be beginning to soften. The fat should appear gelatinous and translucent, not chewy. Check occasionally and add hot water if needed to maintain coverage of the meat and vegetables.
    
    \item Remove and discard the aromatics: \textbf{ginger}, \textbf{green onions}, \textbf{star anise}, \textbf{cinnamon stick}, \textbf{bay leaf}, \textbf{dried red chilis}, \textbf{tangerine peel}, and \textbf{garlic} cloves.
    
    \item Uncover the pot and increase heat to medium-high. Reduce the braising liquid until the sauce thickens and becomes glossy, approximately \textit{10--15~minutes}. The sauce is ready when it coats the back of a spoon thickly and the pork has a deep red-brown sheen. The sauce should cling to the meat, not pool or appear runny. Bubbles will become larger and slower as the sauce thickens.
    
    \item While the sauce thickens, bring a large pot of well-salted water to a rolling boil. Add the quartered \textbf{baby bok choy} (\textit{Medium Bowl~\#2}) and blanch until bright green and tender-crisp, approximately \textit{1\nicefrac{1}{2}--2~minutes}. The stems should be tender but still have a slight crunch, and the leaves should be wilted but not mushy. Drain immediately and keep warm.
    
    \item To serve, assemble in individual bowls: place a layer of blanched \textbf{baby bok choy} on the bottom, add a generous portion of steamed rice on top, then arrange the \textbf{pork belly} pieces and \textbf{daikon} on top of the rice. Spoon the reduced braising sauce over the pork. Garnish with sliced \textbf{green onion} tops and toasted \textbf{sesame seeds} from \textit{Small Bowl~\#2}. Serve immediately. The pork should be fork-tender with gelatinous, melt-in-your-mouth fat and a glossy, deep red-brown exterior. The \textbf{daikon} should be tender and translucent, having absorbed the rich braising liquid.
\end{enumerate}

\newpage

% Begin compact two-column layout
{\small
\setlength{\columnsep}{20pt}
\setlength{\multicolsep}{6pt}
\begin{multicols}{2}
\setlength{\parindent}{0pt}
\setlength{\parskip}{4pt}

\subsection*{Equipment Required}
\begin{itemize}
    \item 5--6 quart heavy-bottomed pot or Dutch oven with lid
    \item Wooden spoon or silicone spatula for stirring
    \item Tongs or slotted spoon for handling pork
    \item Sharp knife and cutting board
    \item Measuring cups and spoons
    \item Large mixing bowls (2--3)
\end{itemize}

\subsection*{Mise en Place}
\begin{itemize}
    \item Cut \textbf{pork belly} into uniform 1\nicefrac{1}{2}--2~inch cubes before beginning
    \item Measure and set aside: \textbf{granulated sugar} (5~Tbsp.) and \textbf{Shaoxing wine} (3~Tbsp.) --- these are needed during time-sensitive steps
    \item \textit{Small Bowl~\#1} --- all ingredients added together in step 5: \textbf{ginger} (1~inch piece, sliced and smashed), \textbf{green onions} (2--inch pieces), smashed \textbf{garlic} cloves, \textbf{light soy sauce} (2~Tbsp.), \textbf{dark soy sauce} (1~Tbsp.), \textbf{star anise} (2), \textbf{cinnamon stick} (1), \textbf{bay leaf} (1), \textbf{dried red chilis} (3), and \textbf{tangerine peel} (strips from 1 tangerine)
    \item \textit{Medium Bowl~\#1} --- \textbf{daikon radish} sliced into \nicefrac{1}{2}~inch rounds
    \item \textit{Medium Bowl~\#2} --- \textbf{baby bok choy} trimmed and quartered lengthwise
    \item \textit{Small Bowl~\#2} --- garnish: thinly sliced \textbf{green onion} tops and toasted \textbf{sesame seeds}
    \item Have hot \textbf{water or stock} ready for braising liquid
\end{itemize}

\subsection*{Ingredient Tips}
\begin{itemize}
    \item \textbf{Skin-on pork belly} is essential for authentic texture---the skin becomes gelatinous during braising
    \item \textbf{Rock sugar (bing tang)} can be substituted for granulated sugar for a glossier finish; crush if using large pieces
    \item \textbf{Shaoxing wine} adds depth and neutralizes gamey flavors; dry sherry can substitute if unavailable
    \item \textbf{Dark soy sauce} provides the characteristic deep color; do not omit
    \item Fresh aromatics are preferred; dried star anise and bay leaves work but fresh ginger and green onions are essential
    \item \textbf{Dried red chilis} add a subtle heat and depth; adjust quantity to taste preference, or remove seeds for milder heat
    \item \textbf{Fresh tangerine peel} adds a bright, citrusy aroma; use only the colored outer peel, avoiding the bitter white pith
    \item \textbf{Daikon radish} adds a mild, slightly sweet flavor and absorbs the rich braising liquid beautifully
    \item \textbf{Baby bok choy} provides a fresh, crisp contrast to the rich pork; look for firm, bright green heads with no yellowing
    \item \textbf{Sesame seeds} should be toasted until golden and fragrant for best flavor; store-bought toasted sesame seeds can be used for convenience
    \item Use homemade or high-quality stock if available; water works but stock adds depth
\end{itemize}

\subsection*{Preparation Tips}
\begin{itemize}
    \item Pat \textbf{pork belly} thoroughly dry before searing; moisture prevents proper browning
    \item Render fat slowly during searing---the goal is golden color, not deep browning
    \item The sugar caramelization step (\textit{chao tang se}) is critical---this creates the signature glossy, deep red-brown color and complex flavor
    \item Watch the caramel carefully; it can burn quickly. Remove from heat if it darkens too fast
    \item The caramel should be deep amber/red-brown, not black. Black caramel will taste bitter
    \item Coat each piece of pork evenly with caramel before adding liquid
    \item Braise low and slow---the fat should become gelatinous and translucent, not chewy
    \item Check liquid level periodically; add hot water if needed to maintain coverage
    \item Remove aromatics before reducing sauce for cleaner presentation
    \item Reduce sauce until it coats the meat thickly; it should not be runny
    \item Blanch \textbf{baby bok choy} while the sauce reduces to save time; do not overcook---it should be tender-crisp, not mushy
    \item The finished dish should have a glossy sheen and deep red-brown color
\end{itemize}

\subsection*{Make Ahead \& Storage}
\begin{itemize}
    \item The braise can be prepared up to \textit{2~days} ahead; flavors deepen overnight
    \item Cool completely before refrigerating
    \item Reheat gently in a covered pot over low heat, adding a splash of water if sauce has reduced too much
    \item The dish does not freeze well due to the delicate texture of the gelatinous fat
    \item Leftovers keep refrigerated for up to \textit{4~days}
\end{itemize}

\subsection*{Troubleshooting}
\begin{itemize}
    \item If caramel burns: start over with fresh sugar; burned caramel cannot be salvaged
    \item If caramel darkens too quickly: remove from heat immediately and stir; residual heat will continue cooking
    \item If pork is tough: continue braising, checking every \textit{15~minutes} until fork-tender
    \item If sauce is too thin: continue reducing over medium-high heat until it coats the back of a spoon
    \item If sauce is too thick: add a splash of hot water and stir to combine
    \item If color is too light: the caramelization step may have been insufficient; ensure sugar reaches deep amber/red-brown
    \item If fat is chewy: continue braising until it becomes gelatinous and translucent
\end{itemize}

\subsection*{Serving Suggestions}
\begin{itemize}
    \item Assemble in individual bowls: \textbf{baby bok choy} on the bottom, steamed rice in the middle, and the braised \textbf{pork belly} and \textbf{daikon} on top
    \item The layered presentation allows the rice to absorb the rich braising sauce while the bok choy provides a fresh, crisp base
    \item Garnish with sliced \textbf{green onion} tops and toasted \textbf{sesame seeds} for color, texture, and flavor
    \item The dish is rich; moderate portions are recommended
    \item Serve hot; the gelatinous fat should be warm and yielding, and the bok choy should be bright green and tender-crisp
\end{itemize}

\end{multicols}
}

\end{document}
