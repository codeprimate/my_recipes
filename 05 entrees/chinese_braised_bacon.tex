\documentclass[11pt,letterpaper]{article}
\usepackage{fontspec}
\usepackage{tocloft}
\usepackage{multicol}
\usepackage{fancyhdr}
\usepackage{nicefrac}
\usepackage[left=1.3in,right=1.3in,top=1in,bottom=1in]{geometry}
\setmainfont[Scale=1.2,AutoFakeBold=1.5,AutoFakeSlant=0.3]{Doves Type}

\title{Chinese-Style Red-Braised Hickory Bacon}
\author{}
\date{}

% Configure fancy header
\pagestyle{fancy}
\fancyhf{} % Clear all header and footer fields
\fancyhead[C]{\textit{Chinese-Style Red-Braised Hickory Bacon}} % Center header
\fancyfoot[C]{\thepage} % Center page number at bottom
\renewcommand{\headrulewidth}{0pt} % Remove header line
\renewcommand{\footrulewidth}{0pt} % Remove footer line

\begin{document}

\maketitle
\thispagestyle{empty}

\begin{quote}
\textit{Thick-cut hickory bacon is caramelized with rock sugar, then braised Chinese-style with Shaoxing wine, aromatics, and daikon radish until tender. The braised bacon is finished in the oven for caramelized edges, then combined with mushrooms and bamboo shoots in a thickened hong shao sauce. Served with blanched bok choy and jasmine rice.}
\end{quote}

\section*{Ingredients}
\setlength{\columnsep}{20pt}
\begin{multicols}{2}
\noindent
    Hickory bacon, \nicefrac{1}{4}" thick-cut \dotfill 1.5 lbs. \\
    Rock sugar, crushed \dotfill 4 Tbsp. \\
    Vegetable oil \dotfill 3 Tbsp. \\
    Shaoxing rice wine \dotfill 1\nicefrac{1}{4} cups \\
    Light soy sauce \dotfill 2\nicefrac{1}{2} Tbsp. \\
    Dark soy sauce \dotfill 1\nicefrac{1}{2} Tbsp. \\
    Water OR chicken stock \dotfill 2\nicefrac{1}{2} cups \\
    Ginger, sliced \nicefrac{1}{4}" thick \dotfill 5 slices \\
    \columnbreak
    Scallions, white parts \dotfill 3-4 \\
    Star anise \dotfill 2-3 pieces \\
    Cinnamon stick, 3" \dotfill 1 \\
    Dried red chilies, whole \dotfill 2-3 \\
    Straw mushrooms, canned (15~oz.) \dotfill 1 can \\
    Bamboo shoots, canned (8~oz.) \dotfill 1 can \\
    Daikon radish, peeled \dotfill 1-1.5 lbs. \\
    Baby bok choy \dotfill 1-1.5 lbs. \\
    Cornstarch \dotfill 2\nicefrac{1}{2} Tbsp. \\
    Jasmine rice, uncooked \dotfill 2 cups \\
    Scallion greens, sliced thin \dotfill \nicefrac{1}{4} cup \\
\end{multicols}

\section*{Directions}

\noindent
Crush \textbf{rock sugar} into smaller chunks; set aside in \textit{Small Bowl~\#1} ---
Slice \textbf{ginger} and cut 3-4 \textbf{scallion whites} into 2-3" pieces; combine in \textit{Small Bowl~\#2} (aromatics) ---
Lightly crack 2-3 \textbf{dried chilies} with side of knife; combine with 2-3 \textbf{star anise} and 1 \textbf{cinnamon stick} in \textit{Small Bowl~\#3} (whole spices) ---
Drain and rinse \textbf{straw mushrooms} and \textbf{bamboo shoots}; if \textbf{bamboo shoots} are whole, slice into bite-size pieces; combine in \textit{Medium Bowl~\#1} (vegetables) ---
Peel 1-1.5~lbs. \textbf{daikon radish} and cut into 1-1\nicefrac{1}{2}" chunks; set aside in \textit{Medium Bowl~\#2} (daikon) ---
Cut \nicefrac{1}{4}~cup \textbf{scallion greens}; set aside in \textit{Small Bowl~\#4} (garnish) ---
Combine 2\nicefrac{1}{2}~Tbsp. \textbf{cornstarch} with 3~Tbsp. cold water in \textit{Small Bowl~\#5} (slurry) ---
Start cooking 2~cups \textbf{jasmine rice} according to package directions

\begin{enumerate}
    \item Cut 1.5~lbs. thick-cut \textbf{hickory bacon} into 2" lengths. Pat \textbf{bacon} pieces completely dry with paper towels. \textbf{Bacon} should feel dry to the touch with no moisture remaining.
    
    \item In a Dutch oven over medium-low heat, add 2~Tbsp. \textbf{vegetable oil} and 4~Tbsp. crushed \textbf{rock sugar} (\textit{Small Bowl~\#1}). Heat without stirring, swirling the pot occasionally, until sugar melts and turns amber brown, about \textit{3-5~minutes}. Sugar is ready when it appears uniformly amber brown (like honey) and bubbles actively, with a rich caramel aroma. Watch carefully to prevent burning—if edges darken too quickly, reduce heat slightly. Continue swirling until entire surface is amber brown.
    
    \item Carefully add \textbf{bacon} pieces to the caramelized sugar (sugar may sputter—stand back). Increase heat to medium and cook, turning pieces gently with tongs, until well-coated with caramel and lightly browned, about \textit{3-4~minutes}. \textbf{Bacon} is ready when pieces appear evenly coated with caramel glaze, edges show light golden brown color, and caramel has adhered to the surface.
    
    \item Add \textbf{ginger} and \textbf{scallion whites} (\textit{Small Bowl~\#2}), then deglaze with 1\nicefrac{1}{4}~cups \textbf{Shaoxing wine}. Cook for \textit{2~minutes}, stirring constantly to scrape up caramelized bits from bottom of pot, until liquid has reduced slightly and aromatics are fragrant. The bottom of the pot should appear clean with no stuck-on caramel remaining.
    
    \item Add 2\nicefrac{1}{2}~Tbsp. \textbf{light soy sauce}, 1\nicefrac{1}{2}~Tbsp. \textbf{dark soy sauce}, \textbf{star anise}, \textbf{cinnamon stick}, and \textbf{dried chilies} (\textit{Small Bowl~\#3}). Add \textbf{daikon radish} chunks (\textit{Medium Bowl~\#2}), then add 2\nicefrac{1}{2}~cups \textbf{water} or \textbf{stock}. Liquid should come about halfway up the \textbf{bacon} pieces (approximately to the midpoint of the pieces when viewed from the side). If needed, add additional \textbf{water} or \textbf{stock} to reach this level.
    
    \item Bring to a boil, then reduce heat to low. Cover and simmer for \textit{25-30~minutes}, stirring occasionally, until \textbf{bacon} is tender and \textbf{daikon radish} is soft and translucent. \textbf{Bacon} is done when pieces feel tender when pierced with a fork (fork should slide in easily), meat appears slightly shrunken from edges, and braising liquid has reduced and darkened. \textbf{Daikon radish} is done when chunks appear translucent around edges, feel tender when pierced with a fork (fork should slide in easily), and have absorbed the braising liquid color. Check liquid level at \textit{15~minutes}; add \textbf{water} if liquid drops below halfway up the \textbf{bacon} pieces to prevent scorching. Continue simmering until pieces feel tender when pierced with a fork.
    
    \item While \textbf{bacon} simmers, prepare \textbf{bok choy}: Bring a large pot of water to a rolling boil. Halve or quarter 1-1.5~lbs. \textbf{baby bok choy} lengthwise depending on size (small heads in half, larger heads in quarters). Blanch in boiling water for \textit{2~minutes} until bright green and tender-crisp. \textbf{Bok choy} is done when leaves appear bright green (not dull), stems feel tender-crisp when pierced with a knife (not mushy), and leaves have wilted slightly. Drain and immediately shock in ice water until completely cool, about \textit{1~minute}. Drain again and arrange on serving platter. Set aside.
    
    \item After \textit{25-30~minutes}, \textbf{bacon} should be tender (pieces feel tender when pierced with a fork) and \textbf{daikon radish} should be soft and translucent. Using a slotted spoon, transfer \textbf{bacon} pieces and \textbf{daikon radish} chunks to \textit{Large Bowl~\#1}, leaving braising liquid in pot. Remove and discard \textbf{ginger}, \textbf{scallion whites}, \textbf{star anise}, \textbf{cinnamon stick}, and \textbf{chilies} from liquid. Strain liquid through a fine-mesh strainer if needed to remove any remaining aromatics.
    
    \item Preheat oven to \textit{420°F}. Place a wire rack on a rimmed baking sheet and set aside.
    
    \item Increase heat under Dutch oven with braising liquid to medium-high. Boil uncovered, stirring occasionally, until liquid reduces slightly, about \textit{3-5~minutes}. Liquid should appear slightly thickened but still fluid. Reserve \textit{half} of the reduced liquid to \textit{Medium Bowl~\#3} (reserved sauce) for later use.
    
    \item Stir \textbf{cornstarch slurry} (\textit{Small Bowl~\#5}) to recombine (cornstarch settles quickly), then slowly drizzle into remaining liquid in Dutch oven while stirring constantly with a whisk or fork. Cook for \textit{1-2~minutes} until sauce thickens to a glossy, coating consistency. Sauce is ready when it appears glossy and smooth, coats the back of a spoon thickly (leaving a clear trail when you draw your finger through it), and has the consistency of thick gravy. If sauce becomes too thick, add a splash of \textbf{water} or \textbf{stock} and stir to thin. If too thin, mix another \nicefrac{1}{2}~Tbsp. \textbf{cornstarch} with 1~Tbsp. cold \textbf{water} and add gradually.
    
    \item Arrange \textbf{bacon} pieces (\textit{Large Bowl~\#1}) in a single layer on the prepared wire rack. Bake at \textit{420°F} for \textit{5-7~minutes}, until edges are caramelized and slightly crispy. \textbf{Bacon} is done when edges appear dark golden brown and slightly crispy, surface appears glossy, and pieces feel firm but not hard when gently pressed. Set \textbf{daikon radish} chunks aside (they will be added back to the pot later).
    
    \item While \textbf{bacon} bakes, add \textbf{straw mushrooms} and \textbf{bamboo shoots} (\textit{Medium Bowl~\#1}) to the thickened sauce in the Dutch oven. Add some of the reserved sauce (\textit{Medium Bowl~\#3}) so the mixture is saucy but not soupy. Heat over medium heat for \textit{5~minutes}, stirring occasionally, until \textbf{mushrooms} and \textbf{bamboo shoots} appear heated through (steam rises from the surface) and have absorbed some of the sauce color.
    
    \item Add the finished \textbf{bacon} pieces and \textbf{daikon radish} chunks to the pot with \textbf{mushrooms} and \textbf{bamboo shoots}. Gently stir to combine and heat through, about \textit{1~minute}. Taste and adjust seasoning if needed.
    
    \item To serve: Dish can be served either portioned directly, or with a bed of blanched \textbf{bok choy} and cooked \textbf{jasmine rice}. Garnish with \textbf{scallion greens} (\textit{Small Bowl~\#4}). Serve immediately while warm.
\end{enumerate}

\newpage

{\footnotesize
\setlength{\columnsep}{20pt}
\setlength{\multicolsep}{6pt}
\begin{multicols}{2}
\setlength{\parindent}{0pt}
\setlength{\parskip}{2pt}
\setlength{\itemsep}{0pt}
\setlength{\parsep}{0pt}

\subsection*{Yield}
\begin{itemize}
    \item Serves 4-6 as main course
    \item Makes about 1.5~lbs. braised \textbf{bacon} with accompaniments
\end{itemize}

\subsection*{Equipment Required}
\begin{itemize}
    \item Dutch oven (5-6 quart)
    \item Large pot (for blanching bok choy)
    \item Wire rack and rimmed baking sheet
    \item Large slotted spoon
    \item Fine-mesh strainer (optional, for removing aromatics)
    \item Small prep bowls (5)
    \item Medium prep bowls (3)
    \item Large prep bowls (1)
    \item Sharp knife
    \item Cutting board
    \item Paper towels
    \item Ice bath (bowl with ice water)
    \item Measuring cups and spoons
    \item Whisk or fork (for slurry)
    \item Tongs (for turning \textbf{bacon})
    \item Serving platters and bowls
\end{itemize}

\subsection*{Mise en Place}
\begin{itemize}
    \item Small Bowl \#1 — caramelization: 4~Tbsp. crushed \textbf{rock sugar}
    \item Small Bowl \#2 — aromatics: 5 slices \textbf{ginger}, 3-4 \textbf{scallion whites} cut into 2-3" pieces
    \item Small Bowl \#3 — whole spices: 2-3 \textbf{star anise}, 1 \textbf{cinnamon stick}, 2-3 cracked \textbf{dried chilies}
    \item Small Bowl \#4 — garnish: \nicefrac{1}{4}~cup sliced \textbf{scallion greens}
    \item Small Bowl \#5 — slurry: 2\nicefrac{1}{2}~Tbsp. \textbf{cornstarch} with 3~Tbsp. cold water (stir before using)
    \item Medium Bowl \#1 — vegetables: 1 can drained \textbf{straw mushrooms}, 1 can drained and sliced \textbf{bamboo shoots} (about 3~cups combined)
    \item Medium Bowl \#2 — daikon: 1-1.5~lbs. peeled and cut \textbf{daikon radish} chunks (about 3-4~cups)
    \item Medium Bowl \#3 — reserved sauce: set aside for reserved braising liquid (about 1~cup)
    \item Large Bowl \#1 — holding: set aside for cooked \textbf{bacon} and \textbf{daikon radish} temporarily (about 3-4~cups after braising)
    \item Start \textbf{jasmine rice} cooking when beginning step 1 (timing should align with braising)
\end{itemize}

\subsection*{Ingredient Tips}
\begin{itemize}
    \item Ultra-thick \textbf{hickory bacon} (\nicefrac{1}{4}") is essential --- thinner bacon will dry out
    \item \textbf{Rock sugar} creates a cleaner, glossier finish than granulated; if unavailable, use brown sugar
    \item Quality \textbf{Shaoxing rice wine} should have minimal salt content (check label)
    \item \textbf{Dark soy sauce} provides color; don't substitute with regular soy sauce
    \item Canned \textbf{straw mushrooms} have better texture than fresh for braising
    \item Canned \textbf{bamboo shoots} work perfectly; rinse well to remove tinny flavor
    \item \textbf{Daikon radish} should be firm and heavy for its size; avoid spongy or hollow-feeling radishes
    \item \textbf{Baby bok choy} (5-6" heads) are ideal; regular bok choy works if quartered
    \item \textbf{Jasmine rice} absorbs sauce better than other long-grain varieties
\end{itemize}

\subsection*{Preparation Tips}
\begin{itemize}
    \item Pat \textbf{bacon} completely dry before caramelization to prevent splattering; any moisture will cause dangerous sputtering
    \item Don't stir sugar while melting --- swirl pot instead to prevent crystallization; sugar should appear uniformly amber brown
    \item Watch caramelization carefully --- amber color develops quickly and can burn; reduce heat if edges darken too fast
    \item Braising liquid should stay at a gentle simmer, not a rolling boil; adjust heat to maintain steady, gentle bubbles
    \item Check liquid level during braising; add \textbf{water} if it drops below halfway up the \textbf{bacon} to prevent scorching
    \item \textbf{Daikon radish} will become translucent and tender during braising; it absorbs the braising liquid beautifully and adds a sweet, mild flavor
    \item Reduce braising liquid only briefly before adding slurry; the thickening agent allows for shorter reduction time
    \item Reserve half the sauce before thickening to adjust consistency later when adding vegetables
    \item Arrange \textbf{bacon} in single layer on wire rack for even browning in oven; edges should crisp without burning
    \item Oven temperature of \textit{420°F} provides quick caramelization; watch carefully to prevent burning
    \item Add reserved sauce to vegetables as needed to achieve saucy but not soupy consistency
    \item Cornstarch slurry must be stirred before adding to prevent lumps; drizzle slowly while stirring constantly
    \item Remove whole spices before finishing to prevent biting into them; strain if needed
    \item Shocking \textbf{bok choy} in ice water preserves bright green color and stops cooking immediately
\end{itemize}

\subsection*{Make Ahead \& Storage}
\begin{itemize}
    \item Complete braising up to \textit{24~hours} ahead; refrigerate \textbf{bacon} and \textbf{daikon radish} in its liquid
    \item Reheat gently before reducing liquid and finishing in oven
    \item Do not finish in oven ahead --- texture deteriorates when reheated
    \item Blanch \textbf{bok choy} up to \textit{4~hours} ahead; refrigerate after shocking
    \item Leftover braised \textbf{bacon} and \textbf{daikon radish} keep \textit{3-4~days} refrigerated
    \item Reheat leftovers gently in a covered pan over low heat, or in oven at \textit{350°F} for \textit{10~minutes}
    \item Not recommended for freezing --- texture of canned vegetables suffers
\end{itemize}

\subsection*{Serving Suggestions}
\begin{itemize}
    \item Serve either portioned directly, or with a bed of blanched \textbf{bok choy} and \textbf{jasmine rice}
    \item For family-style service, present the saucy main in a large serving bowl
    \item Arrange blanched \textbf{bok choy} as a bed on individual plates or serving platter
    \item Spoon the saucy main over \textbf{rice} and \textbf{bok choy}
    \item Excellent as part of larger Chinese meal with 2-3 other dishes
    \item Pairs well with simple stir-fried greens or cucumber salad
    \item Garnish generously with fresh \textbf{scallion greens} for color and brightness
    \item Serve immediately while \textbf{bacon} is warm and edges are crispy
\end{itemize}


\end{multicols}
}

\end{document}
