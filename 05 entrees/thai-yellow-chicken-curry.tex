\documentclass[11pt,letterpaper]{article}
\usepackage{fontspec}
\usepackage{tocloft}
\usepackage{multicol}
\usepackage{fancyhdr}
\usepackage{nicefrac}
\usepackage[left=1.3in,right=1.3in,top=1in,bottom=1in]{geometry}
\setmainfont[Scale=1.2,AutoFakeBold=1.5,AutoFakeSlant=0.3]{Doves Type}

\title{Thai Yellow Chicken Curry}
\author{}
\date{}

% Configure fancy header
\pagestyle{fancy}
\fancyhf{} % Clear all header and footer fields
\fancyhead[C]{\textit{Thai Yellow Chicken Curry}} % Center header with recipe title
\fancyfoot[C]{\thepage} % Center page number at bottom
\renewcommand{\headrulewidth}{0pt} % Remove header line
\renewcommand{\footrulewidth}{0pt} % Remove footer line

\begin{document}

\maketitle
\thispagestyle{empty}

\section*{Ingredients}

\setlength{\columnsep}{20pt}
\begin{multicols}{2}
\noindent
    Bone-in chicken thighs \dotfill 3 lbs \\
    Coconut cream \dotfill 2 (13.5 oz.) cans \\
    Yellow curry paste \dotfill 5 Tbsp. \\
    Yukon gold potatoes \dotfill 1\nicefrac{1}{2}~lbs \\
    Carrots, medium \dotfill 3 \\
    Yellow onion, large \dotfill 1 \\
    Shallots \dotfill 3 \\
    Garlic cloves \dotfill 8 \\
    Fresh ginger \dotfill 3 Tbsp. \\
    Lemongrass stalks \dotfill 2 \\
    \columnbreak
    Canned pineapple slices \dotfill 2 cups \\
    Fish sauce \dotfill 3 Tbsp. \\
    Brown or turbinado sugar \dotfill 3 Tbsp. \\
    Chicken stock \dotfill 2 cups \\
    Neutral oil \dotfill 2 Tbsp. + \nicefrac{1}{2}~cup \\
    Limes \dotfill 2 \\
    Fresh cilantro \dotfill for garnish \\
    Thai basil \dotfill for garnish \\
    Salt \dotfill to taste \\
    Dried árbol chilies \dotfill 15-20 \\
    Fresh serrano peppers \dotfill 3-4 \\
\end{multicols}

\section*{Directions}

\noindent
Cut 1\nicefrac{1}{2}~lbs \textbf{potatoes} into 1\nicefrac{1}{2}" uniform chunks ---
Peel and roll-cut 3 medium \textbf{carrots} into 1" pieces ---
Combine \textbf{potatoes} and \textbf{carrots} in \textit{Medium Bowl~\#1} (vegetables) ---
Cut 1 large \textbf{onion} into 1" wedges ---
Slice 3 \textbf{shallots} ---
Mince 6 \textbf{garlic~cloves} ---
Mince 3~Tbsp. \textbf{ginger} ---
Combine \textbf{shallots}, \textbf{ginger}, and \textbf{garlic} in \textit{Small Bowl~\#1} (aromatics) ---
Bruise and cut 2 \textbf{lemongrass~stalks} into 3" pieces ---
Drain and pat dry 2~cups \textbf{canned~pineapple~slices} and place in \textit{Large Bowl~\#1} ---
Remove stems from 15-20 \textbf{árbol~chilies} ---
Slice 3-4 \textbf{serrano~peppers} thin (seeds in) ---
Slice 2 \textbf{garlic~cloves} thin for oil ---
Zest and juice 2 \textbf{limes}

\subsection*{Chili Oil}

\begin{enumerate}
    \item In a small saucepan, combine \nicefrac{1}{2}~cup \textbf{neutral~oil}, 15-20 \textbf{dried~árbol~chilies}, 3-4 sliced \textbf{serrano~peppers}, and 2 sliced \textbf{garlic~cloves}. Place over medium-low heat.
    
    \item Heat gently, watching carefully. When \textbf{chilies} darken and become fragrant (about \textit{2-3~minutes}), immediately remove from heat.
    
    \item Add pinch of \textbf{salt} to the hot oil off-heat. Let steep for at least \textit{20-30~minutes}. Strain or leave \textbf{chilies} in for presentation. Set aside.
\end{enumerate}

\newpage

\subsection*{Curry}

\begin{enumerate}

    \item Render 3~lbs bone-in \textbf{chicken~thighs} in dutch oven over medium-high heat for about \textit{3-4~minutes} per side, until deeply golden-brown and skin releases easily from the pan. Transfer to \textit{Large Bowl~\#2} and set aside.

    \item Open 2 (13.5~oz.) cans of \textbf{coconut~cream} without shaking. Scoop the thick cream from the top of one can into a large Dutch oven or heavy-bottomed pot (should yield about \nicefrac{3}{4}-1~cup thick cream). Reserve the remaining coconut liquid and the second can in \textit{Medium Bowl~\#2}.
    
    \item Heat the thick \textbf{coconut~cream} over medium-high heat, stirring occasionally. It will begin to separate and the fat will "crack" (you'll see oil pooling). This takes \textit{3-5~minutes}. When you see clear oil separating, you're ready.
    
    \item Add 5~Tbsp. \textbf{yellow~curry~paste} to the coconut fat. Fry the \textbf{paste}, stirring constantly, until it darkens, becomes very fragrant, and the oil takes on the \textbf{paste's} color, about \textit{3-4~minutes}. The mixture should smell toasted and complex, not raw.
    
    \item Add 3 sliced \textbf{shallots}, 3~Tbsp. minced \textbf{ginger}, and 6 minced \textbf{garlic~cloves} (\textit{Small Bowl~\#1}) to the \textbf{paste} and cook, stirring, for \textit{2~minutes} until \textbf{shallots} are softened. Continue cooking for \textit{1~minute} more until fragrant.

    \item Add seared \textbf{chicken~thighs}, remaining \textbf{coconut~cream} (\textit{Medium Bowl~\#2}), 2~cups \textbf{chicken~stock}, 2 bruised \textbf{lemongrass~stalks}, \textbf{potatoes} and \textbf{carrots} (\textit{Medium Bowl~\#1}), and 3~Tbsp. \textbf{fish~sauce}. Stir to combine.
    
    \item Bring to a boil, then reduce heat to maintain a gentle simmer. Cover partially and cook for \textit{35-40~minutes}, stirring occasionally, until \textbf{chicken} is very tender and \textbf{potatoes} have softened and begun to break down slightly, thickening the sauce.
    
    \item Add 1 large \textbf{onion} cut into 1" wedges and continue simmering for \textit{10~minutes} until \textbf{onions} are tender but still hold their shape.
    
    \item Remove \textbf{chicken~thighs} to a cutting board. The bones should pull out easily at this point. Discard skin if desired. Shred or chop the meat into bite-sized pieces and return to the curry.
    
    \item While curry continues to simmer, heat a skillet over high heat. Sear \textbf{pineapple~slices} (\textit{Large Bowl~\#1}) quickly, about \textit{1-2~minutes} per side, until lightly caramelized. Roughly chop seared \textbf{pineapple} and add to curry.
    
    \item Taste the curry broth. Add \textbf{brown~or~turbinado~sugar} 1~Tbsp. at a time (up to 3~Tbsp. total), tasting as you go. The curry should be balanced: rich, slightly sweet, savory, with subtle sour notes from the \textbf{pineapple}.
    
    \item Simmer for final \textit{5~minutes} to meld flavors. Taste and adjust seasoning with \textbf{salt} if needed (\textbf{fish~sauce} usually provides enough).
    
    \item Remove \textbf{lemongrass~stalks}. Turn off heat and stir in \textbf{juice} and \textbf{zest} from 2 \textbf{limes}.
    
    \item Ladle curry over steamed jasmine rice. Garnish with fresh \textbf{cilantro} and \textbf{Thai~basil}. Provide \textbf{lime~wedges} and the \textbf{chili~oil} on the side. For heat, drizzle \textit{1-2~tsp.} of \textbf{chili~oil} over individual portions.
\end{enumerate}

\newpage

% Begin compact two-column layout
{\small
\setlength{\columnsep}{20pt}
\setlength{\multicolsep}{6pt}
\begin{multicols}{2}
\setlength{\parindent}{0pt}
\setlength{\parskip}{4pt}

\subsection*{Equipment Required}
\begin{itemize}
    \item Large Dutch oven or heavy-bottomed pot (6-8 quart capacity)
    \item Small saucepan (for chili oil)
    \item Large skillet (for searing pineapple)
    \item Cutting board and sharp knife
    \item Measuring cups and spoons
    \item Wooden spoon or silicone spatula
    \item Fine-mesh strainer (optional, for chili oil)
    \item Vegetable peeler
    \item Citrus zester or microplane
    \item Ladle
    \item Tongs (for removing chicken thighs)
\end{itemize}

\subsection*{Mise en Place}
\begin{itemize}
    \item Chill \textbf{coconut~cream} cans overnight or for several hours to ensure proper fat separation
    \item Cut all vegetables before starting
    \item Cut \textbf{potatoes} into uniform 1\nicefrac{1}{2}" chunks for consistent cooking
    \item Prepare \textbf{chili~oil} while curry simmers or up to several days ahead
    \item Have all aromatics prepped and ready before beginning to cook
    \item Bruise \textbf{lemongrass} by smashing with the flat of a knife to release oils
    \item \textbf{Small Bowl~\#1} --- aromatics: sliced \textbf{shallots}, minced \textbf{ginger}, and minced \textbf{garlic} (about \nicefrac{1}{2}~cup total)
    \item \textbf{Medium Bowl~\#1} --- vegetables: \textbf{potatoes} cut into 1\nicefrac{1}{2}" chunks and roll-cut \textbf{carrots} (about 3~cups total)
    \item \textbf{Medium Bowl~\#2} --- reserved \textbf{coconut~cream} liquid from both cans (about 2~cups)
    \item \textbf{Large Bowl~\#1} --- drained and patted dry \textbf{canned~pineapple~slices} (2~cups)
    \item \textbf{Large Bowl~\#2} --- seared \textbf{chicken~thighs} (set aside after initial searing)
\end{itemize}

\subsection*{Ingredient Tips}
\begin{itemize}
    \item Bone-in \textbf{chicken~thighs} provide superior flavor - the bones contribute gelatin and depth during cooking
    \item Russet \textbf{potatoes} are essential for their high starch content which naturally thickens the curry as they break down
    \item Cut \textbf{potatoes} into uniform 1\nicefrac{1}{2}" chunks - consistent size ensures even cooking and proper breakdown
    \item Mae Ploy brand \textbf{curry~paste} is recommended for consistent results
    \item \textbf{Fish~sauce} brands vary in saltiness - Red Boat and Three Crabs are more concentrated than Squid or Tiparos brands
    \item Drain and thoroughly pat dry \textbf{canned~pineapple} before searing to ensure proper caramelization
    \item If \textbf{lemongrass} is unavailable, substitute with lemon zest added at the finish
    \item \textbf{Palm~sugar} can replace brown sugar for more authentic flavor
\end{itemize}

\subsection*{Preparation Tips}
\begin{itemize}
    \item The "cracking" of \textbf{coconut~cream} is critical - don't rush this step. You must see clear oil separating before adding \textbf{curry~paste}
    \item Properly blooming the \textbf{curry~paste} in the fat until darkened and fragrant is what elevates this above takeout quality
    \item Sear \textbf{chicken~thighs} for only \textit{3-4~minutes} - they will finish cooking during the simmer
    \item The curry should simmer gently, not boil hard, or the \textbf{chicken} will toughen
    \item After \textit{35-40~minutes}, the \textbf{chicken} bones will pull out easily - this is the ideal time to debone
    \item Keep 2-3 bones in the curry during final simmer for additional body and flavor
    \item Don't skip searing the \textbf{pineapple} - the caramelization adds complexity. Rough chopping after searing helps distribute the flavor throughout the curry
    \item Taste and adjust seasoning multiple times - the balance of sweet, salty, sour is crucial
    \item Add \textbf{lime~juice} and zest at the very end to preserve bright citrus notes
\end{itemize}

\subsection*{Make Ahead \& Storage}
\begin{itemize}
    \item \textbf{Chili~oil} can be made up to \textit{2~weeks} ahead and stored at room temperature
    \item Curry can be made \textit{1-2~days} ahead and often tastes better as flavors meld
    \item Store curry in refrigerator for up to \textit{4~days}
    \item Curry thickens significantly when refrigerated - thin with \textbf{stock} or \textbf{coconut~milk} when reheating
    \item Freezes well for up to \textit{3~months} - thaw overnight in refrigerator
    \item Reheat gently over medium-low heat, stirring frequently
    \item Add fresh \textbf{lime~juice} and herbs after reheating for brightness
\end{itemize}

\subsection*{Serving Suggestions}
\begin{itemize}
    \item Serve over steamed jasmine rice (traditional) or sticky rice
    \item Accompany with Thai cucumber salad for cooling contrast
    \item \textbf{Chili~oil} can be drizzled on individual portions for customizable heat
    \item Garnish generously with fresh \textbf{cilantro}, \textbf{Thai~basil}, and \textbf{lime~wedges}
    \item Crispy fried shallots make an excellent textural garnish
    \item Leftover curry can be used as a filling for savory crepes or over noodles
    \item For a complete meal, serve with spring rolls or satay as appetizers
\end{itemize}

\end{multicols}
}

\end{document}
