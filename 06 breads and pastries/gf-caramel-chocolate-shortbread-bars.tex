\documentclass[11pt,letterpaper]{article}
\usepackage{fontspec}
\usepackage{tocloft}
\usepackage{multicol}
\usepackage{fancyhdr}
\usepackage{nicefrac}
\usepackage[left=1.3in,right=1.3in,top=1in,bottom=1in]{geometry}
\setmainfont[Scale=1.3,AutoFakeBold=1.5,AutoFakeSlant=0.3]{Doves Type}

% Prevent page breaks within list items
\widowpenalty=10000
\clubpenalty=10000
\interlinepenalty=500

\title{Gluten-Free Caramel Cookie Bars}
\author{}
\date{}

% Configure fancy header
\pagestyle{fancy}
\fancyhf{}
\fancyhead[C]{\textit{Gluten-Free Caramel Cookie Bars}}
\fancyfoot[C]{\thepage}
\renewcommand{\headrulewidth}{0pt}
\renewcommand{\footrulewidth}{0pt}

\begin{document}

\maketitle
\thispagestyle{empty}

% Quote intentionally omitted for brevity

\section*{Ingredients}
\setlength{\columnsep}{20pt}
\begin{multicols}{2}
\noindent
    Unsalted butter \dotfill 1\nicefrac{1}{2} cups (3 sticks) \\
    Powdered sugar \dotfill \nicefrac{1}{2} cup \\
    BRM 1:1 G.F. flour \dotfill 2 cups \\
    Fine salt \dotfill \nicefrac{1}{2} tsp. \\
    \columnbreak
    Mexican vanilla \dotfill 2 tsp. \\
    Brown sugar, packed \dotfill 1 cup \\
    Sweetened condensed milk \dotfill 1 can (14 oz.) \\
    Light corn syrup \dotfill \nicefrac{1}{4} cup \\
    Semisweet chocolate, chopped \dotfill 12 oz. \\
    Hazelnuts or Pecans \dotfill \nicefrac{3}{4} cup
\end{multicols}

\section*{Directions}

\noindent
Preheat oven to \textit{350°F} ---
Line a \textit{9×13-inch} pan with parchment, leaving overhang on two long sides ---
Bring \textbf{butter} for shortbread to cool room temperature ---
Measure and combine \textbf{brown sugar}, \textbf{sweetened condensed milk}, \textbf{corn syrup}, and \nicefrac{1}{4}~tsp. \textbf{salt} in \textit{Medium Bowl~\#1} (caramel ingredients) ---
Chop \textbf{semisweet chocolate}: place \textit{9~oz.} in \textit{Medium Bowl~\#2} (to melt) and \textit{3~oz.} in \textit{Small Bowl~\#1} (seed for tempering) ---


\begin{enumerate}

    \item Prepare nuts:
    \begin{enumerate}
        \item Toast \textbf{nuts} at \textit{350°F} until fragrant and golden: \textit{10--12~minutes} for hazelnuts (rub off skins in a towel if desired) or \textit{8--10~minutes} for pecans
        \item Coarsely chop and place in \textit{Small Bowl~\#2}
    \end{enumerate}

    \item Make the shortbread base in a medium bowl:
    \begin{enumerate}
        \item Add 1~cup \textbf{butter}, \nicefrac{1}{2}~cup \textbf{powdered sugar}, and 1~tsp. \textbf{vanilla}. Beat with a hand mixer on medium until smooth and fluffy, about \textit{2~minutes}. Scrape bowl as needed.
        \item Add 2~cups \textbf{GF flour} and \nicefrac{1}{4}~tsp. \textbf{salt}; mix on low until no dry flour remains and dough holds together when pressed.
        \item Press dough evenly into the prepared pan in a uniform layer. Dock all over with a fork.
    \end{enumerate}

    \item Bake at \textit{350°F} for \textit{18--22~minutes} until shortbread is set and golden at the edges: center is light golden and dry to the touch, and the surface feels firm when gently pressed. Cool in pan on a rack until completely cool to the touch, about \textit{45~minutes}. The base must be fully cool before adding caramel or it will melt and slide.

    \item Make the caramel in a heavy 3--4~qt. saucepan:
    \begin{enumerate}
        \item Combine \nicefrac{1}{2}~cup \textbf{butter} and the contents of \textit{Medium Bowl~\#1} (\textbf{brown sugar}, \textbf{sweetened condensed milk}, \textbf{corn syrup}, \nicefrac{1}{4}~tsp. \textbf{salt}). Clip a candy thermometer to the side so the tip is immersed and not touching the bottom.
        \item Cook over \textit{medium heat}, stirring constantly with a heat-safe spatula and scraping the bottom and sides to prevent scorching. The mixture will bubble and thicken; cook until the thermometer reads \textit{235--240°F} (soft-ball stage), about \textit{12--18~minutes}. At this temperature the caramel will set into a chewy (not runny or hard) layer.
        \item Optional cold-water test: drop a small bit into a cup of cold water; it should form a soft, pliable ball that flattens when removed. If still below \textit{235°F}, continue cooking in \textit{1~minute} increments. If you have passed \textit{240°F}, use as is but expect a firmer chew.
    \end{enumerate}

    \item Remove the caramel from heat and stir in remaining 1~tsp. \textbf{vanilla}. Fold in the \textbf{nuts} (\textit{Small Bowl~\#2}). Pour immediately over the cooled shortbread and spread evenly to the edges with a heat-safe spatula.

    \item Cool the caramel at room temperature until it is no longer warm and feels set when gently touched, about \textit{1--2~hours}, or refrigerate for \textit{45--60~minutes} until firm. The caramel layer must be set before adding chocolate.

    \item Melt \textbf{chocolate} (\textit{Medium Bowl~\#2}, 9~oz.) in a double boiler over barely simmering water, or in a microwave in \textit{20--30~second} bursts, stirring after each, until smooth and no lumps remain. Do not exceed \textit{120°F}; if using a thermometer, remove from heat when melted and around \textit{115--118°F}.

    \item Add the reserved \textbf{chocolate} (\textit{Small Bowl~\#1}, 3~oz.) to the melted chocolate in two or three additions, stirring constantly after each until fully melted. Continue stirring until the mixture cools to \textit{88--90°F} on a candy or instant-read thermometer, about \textit{5--10~minutes}. The chocolate is in temper when it reaches this range and looks smooth and glossy. To test: spread a thin layer on a cool plate or the back of a spoon and refrigerate for \textit{2--3~minutes}; it should set shiny and firm with a crisp snap when broken.

    \item Working quickly, pour the tempered \textbf{chocolate} over the cooled caramel layer and spread evenly to the edges with an offset spatula. The chocolate will set with a glossy finish and clean snap when in temper. If it thickens or loses shine before you finish, it has gone out of temper; you can still use it---the bars will taste the same but the coating may look dull or feel slightly soft. Refrigerate until the chocolate is fully set, about \textit{30--45~minutes}.

    \item Use the parchment overhang to lift the slab onto a cutting board. Run a sharp knife under hot water, wipe dry, and cut into 24 bars (6 columns × 4 rows). Wipe the blade clean between cuts for neat edges. Store in an airtight container in the refrigerator.
\end{enumerate}

\newpage

{\footnotesize
\setlength{\columnsep}{20pt}
\setlength{\multicolsep}{6pt}
\begin{multicols}{2}
\setlength{\parindent}{0pt}
\setlength{\parskip}{2pt}
\setlength{\itemsep}{0pt}
\setlength{\parsep}{0pt}

\section*{Equipment Required}
\begin{itemize}
    \item 9×13-inch baking pan
    \item Parchment paper
    \item Candy thermometer (clip-on) or instant-read thermometer
    \item Heavy 3--4~qt. saucepan (for caramel; large size prevents boil-over when adding ingredients)
    \item Double boiler or heatproof bowl plus saucepan (for chocolate), or microwave-safe bowl
    \item Medium bowls (2), Small bowls (2)
    \item Hand mixer
    \item Heat-safe spatula, offset spatula
    \item Cutting board and sharp knife
    \item Measuring cups and spoons
\end{itemize}

\section*{Hints and Notes}

\subsection*{Yield}
\begin{itemize}
    \item Makes 24 small bars (6×4 grid from 9×13 pan)
\end{itemize}

\subsection*{Mise en Place}
\begin{itemize}
    \item \textit{Medium Bowl~\#1} --- caramel wet mix: 1~cup packed \textbf{brown sugar}, 1 can \textbf{sweetened condensed milk}, \nicefrac{1}{4}~cup \textbf{corn syrup}, \nicefrac{1}{4}~tsp. \textbf{salt}
    \item \textit{Medium Bowl~\#2} --- 9~oz. chopped \textbf{semisweet chocolate} (to melt)
    \item \textit{Small Bowl~\#1} --- 3~oz. chopped \textbf{semisweet chocolate} (seed for tempering)
    \item \textit{Small Bowl~\#2} --- \nicefrac{3}{4}~cup toasted, coarsely chopped \textbf{nuts}
    \item Bring 1~cup \textbf{butter} (for shortbread) to cool room temperature; have \nicefrac{1}{2}~cup \textbf{butter} (for caramel) ready
    \item Line pan and preheat oven before starting shortbread
\end{itemize}

\subsection*{Ingredient Tips}
\begin{itemize}
    \item \textbf{1:1 GF flour} (e.g. Bob's Red Mill) gives a tender shortbread; do not use a bread-style GF blend
    \item \textbf{Sweetened condensed milk} is not interchangeable with evaporated milk
    \item \textbf{Light corn syrup} keeps the caramel smooth and chewy; honey can be used in equal amount but may change flavor slightly
    \item \textbf{Semisweet chocolate} (bar or chips) should be real chocolate (cocoa butter); compound coating does not temper
    \item \textbf{Nuts}: toasting brings out flavor---hazelnuts \textit{10--12~minutes} (rub off skins in a towel to reduce bitterness) or pecans \textit{8--10~minutes}
\end{itemize}

\subsection*{Preparation Tips}
\begin{itemize}
    \item \textbf{Caramel}: Use a large, heavy pot so the mixture does not boil over. Stir constantly over medium heat to avoid scorching. Humidity can affect set---if the day is very humid, the caramel may stay slightly softer.
    \item \textbf{Soft-ball stage} (\textit{235--240°F}) is critical: under \textit{235°F} the caramel stays runny; over \textit{248°F} it becomes hard. A candy thermometer is the most reliable guide; the cold-water test is a backup.
    \item \textbf{Tempering}: Keep chocolate dry (no water or steam in the bowl). Stir frequently while cooling to \textit{88--90°F}. If the chocolate goes above \textit{92°F} after seeding, add a bit more chopped chocolate and stir until it returns to range.
    \item Cool shortbread and caramel completely before adding chocolate; warm layers will cause the coating to bloom or fail to set properly.
\end{itemize}

\subsection*{Make Ahead \& Storage}
\begin{itemize}
    \item Bars keep in an airtight container in the refrigerator for up to \textit{1~week}
    \item Serve cold or let stand at room temperature \textit{10--15~minutes} for a slightly softer bite
    \item Layer bars between parchment if stacking to avoid sticking
\end{itemize}

\subsection*{Serving Suggestions}
\begin{itemize}
    \item Serve as a sweet snack or dessert with milk or coffee
    \item Best eaten within a few days for optimal texture and shine
\end{itemize}

\end{multicols}
}

\end{document}
